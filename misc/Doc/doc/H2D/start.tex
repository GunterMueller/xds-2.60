\chapter{Getting Started}
\label{start}

In this chapter we assume that H2D is properly installed and configured
(See Chapter \ref{config}).

\section{Creating a working directory}
\label{start:dirs}

\See{Redirection files}{}{config:red} give you complete freedom
over where you keep your header files and any files which H2D
itself creates for further use. It is recommended to work in a project
oriented fashion --- i.e. to have a separate directory
hierarchy for each set of header files you wish to translate.

In this case, each project shall have a main working directory.
The script called {\tt h2dwork} may be used to create the required
subdirectories and a redirection file. For example,
to create a directory structure for a project called {\tt myproj}
in the current directory, issue the following commands:

\verb'    mkdir myproj'\\
\verb'    cd myproj'\\
\verb'    h2dwork'

{\bf Note:} Since H2D preserves directory hierarchies of original
header files. you may also need to create additional subdirectories.
See \ref{rules:pp:include} for more information.

\section{Invoking H2D}
\label{start:invoking}

H2D is implemented as a command line utility called \verb'h2d'.
To translate a header file (or a set of header files), type

\verb'    h2d { HeaderFile } { Option } [ -prj=ProjectFile ] '

at the command prompt, where \verb'HeaderFile' is a header
file name.

The syntax for \verb'Option' is described in \ref{config:cfg}.

If you specify the \verb'-prj' option, each header will be translated
as if it was specified in a \ProjDir{module} in \verb'ProjectFile'.

To process a \ProjectFile{}, type

\verb'    h2d =p ProjectFile { Option }'

To view the default option values, type

\verb'    h2d =o'

If invoked without parameters, the utility prints a brief
help information.

\section{H2D usage example}

Copy the H2D sample included in your XDS or H2D distribution
to a working directory and type

\verb'    h2d =p example.h2d'

at the command prompt. The H2D banner line will appear:

\verb'    H2D v1.30 (c) XDS 1996-1997' \\
\verb'    File example.h'

After translation the following lines will be displayed:

\verb'    no errors, lines 23.'\\
\verb'    ----------------------------------------------'\\
\verb'    Files 1, lines 23, no errors, time 0:3.'

showing the number of errors, the number of source lines in the file,
and some statistics. The following files will be generated:

\begin{flushleft}
\begin{tabular}{ll}
\tt h2d\_example.def & basic definition module \\
\tt h2d\_example.h   & definitions of \See{types generated by H2D}{}{using:fit:c} \\
\tt mac\_example.def & \See{macro definition module}{}{using:fit:native} \\
\tt mac\_example.mod & \See{prototype macro implementation module}{}{using:fit:native}
\end{tabular}
\end{flushleft}

% !!! More files?

\section{Error reporting}

When H2D detects an error in the input file, it displays an error report.
It contains the file name and position (line and column numbers) where the
error occurred:

\verb"    Error [ example.h 16:44 ] ** Duplicate identifier 'insert'"

The error which is often encountered is

\verb'    Error [...] ** Expected  , or ;'

In most cases it means that an identifier is undefined for some reason.
Try to put "," or ";" at the specified position to find out what is the
problem source.


