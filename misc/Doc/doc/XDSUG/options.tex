% !!! check "inline" and "header" attributes
% !!! options/equation list not full
% !!! platform-specific options (like PM and VIO)
% !!! explicitly specify BE instead of '\ifgenc'/'\ifgencode'

% Modes list

\newcommand{\MLBegin}{\ifonline {\bf Modes:} \else [\fi}
\newcommand{\MLEnd}{\ifonline \else ]\fi}
\newcommand{\ModeC}{\ifonline \ref[COMPILE]{xc:modes:compile}\else compile\fi}
\newcommand{\ModeM}{\ifonline \ref[MAKE]{xc:modes:make}\else make\fi}
\newcommand{\ModeP}{\ifonline \ref[PROJECT]{xc:modes:project}\else project\fi}
\newcommand{\ModeB}{\ifonline \ref[BROWSE]{xc:modes:browse}\else browse\fi}
\newcommand{\ModeG}{\ifonline \ref[GEN]{xc:modes:gen}\else gen\fi}

\chapter{Compiler options and equations}\label{options}

A rich set of \xds{} compiler options allows you to control the source language,
the generated code, and the internal limits and settings.  We distinguish
between boolean options (or just options) and equations. An option can
be set ON (TRUE) or OFF (FALSE), while an equation value is a string.

\section{Options}\label{opt:bool}
\index{options}

Options control the process of compilation, including  language
extensions, run-time checks and code generation. An option can be
set ON (TRUE) or OFF (FALSE).

A compiler setup directive (See \ref{config:options}) is used
to set the option value or to declare a new option.

Options may be set in a \See{configuration file}{}{config:cfg},
\See{on the command line}{}{xc:modes}, in a \See{project file}{}{xc:project}).
or in the source text (See \ref{m2:pragmas}). At any
point of operation, the last value of an option is in effect.

Alphabetical list of all options along with their descriptions
may be found in the section \ref{opt:bool:list}.
\ifonline\else
See also tables
\ref{table:opt:check} (page \pageref{table:opt:check}),
\ref{table:opt:ext}   (page \pageref{table:opt:ext}),
\ref{table:opt:code}  (page \pageref{table:opt:code})
and
\ref{table:opt:misc}  (page \pageref{table:opt:misc}).
\fi

\begin{table}[htbp]
\begin{center}
\begin{tabular}{|l|l|}
\hline
\bf Option  & \bf Meaning \\
\hline
\OERef{ASSERT}       & enable ASSERT generation             \\
\OERef{CHECKDINDEX}  & check of dynamic array bounds        \\
\OERef{CHECKDIV}     & check for a positive divisor         \\
                      & (DIV and MOD)                        \\
\OERef{CHECKINDEX}   & check of static array bounds         \\
\OERef{CHECKNIL}     & NIL pointer dereference check        \\
\OERef{CHECKPROC}    & check of a formal procedure call     \\
\OERef{CHECKRANGE}   & range checks                         \\
                      & (range types and enumerations)       \\
\OERef{CHECKSET}     & range check of set operations        \\
\OERef{CHECKTYPE}    & dynamic type guards (\ot{} only)     \\
\ifgencode
\OERef{COVERFLOW}    & cardinal overflow check              \\
\fi
\ifgencode
\OERef{IOVERFLOW}    & integer overflow check               \\
\fi
\hline
\end{tabular}
\end{center}
\caption{Run-time checks}
\label{table:opt:check}
\index{options!run-time checks}
\end{table}

\begin{table}[htbp]
\begin{center}
\begin{tabular}{|l|l|}
\hline
\bf Option        & \bf Meaning \\
\hline
\OERef{M2ADDTYPES}    & add SHORT and LONG types  \\
\OERef{M2BASE16}      & use 16-bits basic types in Modula-2 \\
\OERef{M2CMPSYM}      & compare symbol files in Modula-2 \\
\OERef{M2EXTENSIONS}  & enable \mt{} extensions  \\
\OERef{O2ADDKWD}      & enable additonal keywords in \ot \\
\OERef{O2EXTENSIONS}  & enable \ot{} extensions  \\
\OERef{O2ISOPRAGMA}   & enable ISO \mt{} pragmas in Oberon \\
\OERef{O2NUMEXT}      & enable \ot{} scientific extensions  \\
\OERef{STORAGE}       & enable default memory management in \mt{} \\
\OERef{TOPSPEED}      & enable Topspeed \mt{}-compatible extensions \\
\hline
\end{tabular}
\end{center}
\caption{Source language control options}\label{table:opt:ext}
\index{options!language control}
\end{table}

\begin{table}[htbp]
\begin{center}
\begin{tabular}{|l|l|}
\hline
\bf Option  &\bf Meaning \\
\hline
\OERef{\_\_GEN\_C\_\_}   & ANSI C code generation      \\
\OERef{\_\_GEN\_X86\_\_} & code generation for 386/486/Pentium/PentiumPro \\
\ifgenc
  \OERef{COMMENT}        & copy comments into a generated C code  \\
  \OERef{CONVHDRNAME}    & use file name in the \verb|#include| directive \\
\fi
\ifgenc
  \OERef{CSTDLIB}     & definition of the C standard library \\
\fi
\ifgencode
  \OERef{DBGNESTEDPROC} & generate information about procedure nesting \\
  \OERef{DBGQUALIDS}  & generate qualified identifiers in debug info \\
  \OERef{DEFLIBS}     & put the default library names into object files \\
\fi
\ifgenc
  \OERef{DIFADR16}    & SYSTEM.DIFADR returns 16-bits value   \\
\fi
\ifgencode
\ifdll
  \OERef{DLLEXPORT}   & show exported objects to DLL clients \\
\fi
  \OERef{DOREORDER}   & perform instruction scheduling \\
  \OERef{GENASM}      & generate assembly text \\
\fi
\ifgenc
  \OERef{GENCDIV}     & generate C division operators \\
  \OERef{GENCPP}      & generate C++                     \\
  \OERef{GENCONSTENUM} & generate enumeration as constants   \\
\fi
\ifgencode
  \OERef{GENCPREF}    & generate underscore prefixes \\
\fi
\ifgenc
  \OERef{GENCTYPES}   & generate C types                     \\
  \OERef{GENDATE}     & insert a date in a C file  \\
\fi
\OERef{GENDEBUG}      & generate code in the debug mode        \\
\ifgencode
\ifdll
  \OERef{GENDLL}      & generate code for dynamic linking      \\
\fi
  \OERef{GENFRAME}    & always generate a procedure frame      \\
\fi
\ifgenc
  \OERef{GENFULLFNAME} & generate full name in \verb'#lineno' directive \\
\fi
  \OERef{GENHISTORY}  & enable postmorten history  \\
\ifgenc
  \OERef{GENKRC}      & generate K\&R C               \\
  \OERef{GENPROCLASS} & generate a specification of a procedure class \\
  \OERef{GENPROFILE}  & generate additional code for profiling \\
\fi
\ifgencode
  \OERef{GENPTRINIT}  & generate a local pointer initialization  \\
\fi
\ifgenc
  \OERef{GENSIZE}    & evaluate sizes of types               \\
  \OERef{GENTYPEDEF} & generate typedef's for types         \\
\fi
\ifgencode
\ifdll
  \OERef{IMPLIB}     & use import libraries                   \\
\fi
\fi
\ifgenc
  \OERef{INDEX16}    & index is of 16-bits                   \\
\fi
  \OERef{LINENO}     & generate line numbers
                   \ifgenc in C code files \fi
                   \ifgencode in object files \fi
                                                        \\
\ifgenc
  \OERef{NOEXTERN}   & do not generate prototypes for external C functions \\
\fi
  \OERef{NOHEADER}   & disable generation of a header file    \\
\ifgenc
  \OERef{NOOPTIMIZE}  & disable a set of opimizations \\
\else
  \OERef{NOOPTIMIZE}  & disable machine-independent opimizations \\
\fi
\ifgencode
  \OERef{NOPTRALIAS} & ignore pointer aliasing              \\
\fi
\ifgencode
  \OERef{ONECODESEG} & generate one code segment \\
\fi
  \OERef{PROCINLINE} & enable in-line procedure expansion \\
\ifgencode
  \OERef{SPACE}      & favor code size over speed \\
\fi
\ifgenc
  \OERef{TARGET16}   & C {\tt int} type is of 16-bits          \\
\fi
\ifgencode
\ifdll
  \OERef{USEDLL}     & use DLL version of the run-time library   \\
\fi
\fi
  \OERef{VERSIONKEY} & append version key to the module initialization \\
\ifgencode
  \OERef{VOLATILE}   & declare variables as volatile \\
\fi
\hline
\end{tabular}
\end{center}
\caption{Code generator control options}\label{table:opt:code}
\index{options!code control}
\end{table}

\begin{table}[htbp]
\begin{center}
\begin{tabular}{|l|l|}
\hline
\bf Option     & \bf Meaning \\
\hline
\OERef{BSCLOSURE}  & browse control option \\
\OERef{BSREDEFINE} & browse control option \\
\OERef{CHANGESYM}  & permission to change a symbol file \\
\OERef{FATFS}      & limit file names to 8.3 \\
\OERef{GCAUTO}     & enables implicit call of the garbage collector \\
\OERef{LONGNAME}   & use long names in batch files  \\
\OERef{M2}         & force the Modula-2 compiler  \\
\OERef{MAIN}       & mark an Oberon-2 main module     \\
\OERef{MAKEDEF}    & generate definition      \\
\OERef{MAKEFILE}   & generate makefile        \\
\OERef{O2}         & forces the Oberon-2 compiler  \\
\OERef{OVERWRITE}  & create a file, always overwrites the old one \\
\OERef{VERBOSE}    & produce verbose messages     \\
\OERef{WERR}       & treat warnings as errors \\
\OERef{WOFF}       & suppress warning messages    \\
\OERef{XCOMMENTS}  & preserve exported comments \\
\hline
\end{tabular}
\end{center}
\caption{Miscellaneous options}\label{table:opt:misc}
\index{options!miscellaneous options}
\end{table}

\pagebreak % To not break the following section with tables
\section{Options reference}\label{opt:bool:list}

This section lists all options in alphabetical order.  Those  options
that may  be arbitrarily  placed in the source code are marked as {\em
inline} options (See also \ref{m2:pragmas}).  There are also options
which can be placed in a source file, but only in a module header
(i.e. before any of the keywords \verb'"DEFINITION"', \verb'"IMPLEMENTATION"', and
\verb'"MODULE"') These options are marked as {\em header}.
If an option is not marked either as header or inline,
then the result of setting it in the source text is undefined.

Operation  modes  in  which  an  option has effect are listed
in square brackets ([]) after the option name; the character
'*' stands for all operation  modes. For example, [browse] means
that  the  option is used by the compiler in the BROWSE operation mode
only.

{\bf Note:} in the \See{MAKE}{}{xc:modes:make} and
\See{PROJECT}{}{xc:modes:project} modes the compiler switches to
the \See{COMPILE}{}{xc:modes:compile} mode to compile each module.

Run-time check options are ON by default. If not explicitly
specified, other options are OFF (FALSE) by default.

\newcommand{\inline}{({\em inline})}
\newcommand{\header}{({\em header})}

\ifonline \else
\begin{description}
\fi

\OptHead{\_\_GEN\_X86\_\_}{code generation for 386/486/Pentium/PentiumPro}
        \MLBegin{}\ModeC{}\MLEnd{}

        The compiler sets this option ON, if the code generation for
        386/486/Pentium/PentiumPro is in operation.

        The option can be used for compiling different text fragments
        for different targets. See also \ref{m2:pragmas:cc}.

\OptHead{\_\_GEN\_C\_\_}{ANSI C code generation}
        \MLBegin{}\ModeC{}\MLEnd{}

        The compiler sets this option ON, if the C code generation is
        in operation.

        The option can be used for compiling different text fragments
        for different targets. See also \ref{m2:pragmas:cc}.

\OptHead{ASSERT}{enable ASSERT generation}
        \MLBegin{}\ModeC{}\MLEnd{} \inline

        If the option is OFF, the compiler ignores all calls
        of the standard procedure \verb'ASSERT'.

        {\bf Warning:} Ensure that all \verb'ASSERT' calls in
        your program do not have side effects (i.e. do not contain
        calls of other function procedures) before setting this option OFF.

        The option is ON by default.


\OptHead{BSCLOSURE}{browse control option}
        \MLBegin{}\ModeB{}\MLEnd{}

        Include all visible methods.

        If the option is set ON, the browser includes all
        defined and inherited type-bound procedure
        declarations with all record declarations when creating
        a pseudo-definition   module.  See also
        \ref{o2:env:makedef}.

\OptHead{BSREDEFINE}{browse control option}
        \MLBegin{}\ModeB{}\MLEnd{}

\nopagebreak
        Include all redefined methods.

        If the option is set ON, the browser includes original
        definitions of any overwritten type-bound  procedures  with
        record declarations.
        See also \ref{o2:env:makedef}.

\OptHead{CHANGESYM}{permission to change a symbol file}
        \MLBegin{}\ModeC{}\MLEnd{} \header

        Permission to change a module interface (a symbol file).

        The \ot{} compiler creates a temporary symbol file
        every time an \ot{} module is compiled,
        compares this symbol file with the existing one and
        overwrites it with the new one if necessary. When
        the  option is OFF (by default), the compiler
        reports an error if interface of a module
        (and, hence, its symbol file) has been changed and does
        not replace the old symbol file.

        {\bf  Note:}  if  the \OERef{M2CMPSYM} option is set, the
        same  is valid for compilation  of a \mt{}
        definition  module,  i.e.,  the {\bf CHANGESYM} option
        should be set ON for the compilation to succeed
        if a module interface has been changed.

\OptHead{CHECKDINDEX}{check of dynamic array bounds}
        \MLBegin{}\ModeC{}\MLEnd{} \inline

        A check of dynamic array bounds.

        If the option is set ON, the compiler generates index checks
        for dynamic arrays (\verb'POINTER TO ARRAY OF T').

        The option is ON by default.


\OptHead{CHECKDIV}{check for a positive divisor (DIV and MOD)}
        \MLBegin{}\ModeC{}\MLEnd{} \inline

        If the option is set ON, the compiler generates a check if
        a divisor is positive in \verb'DIV' and \verb'MOD' operators.

        The option is ON by default.


\OptHead{CHECKINDEX}{check of static array bounds}
        \MLBegin{}\ModeC{}\MLEnd{} \inline

        A check of static array bounds.

        If the option is set ON, the compiler generates index checks
        for all arrays except dynamic (See the \OERef{CHECKDINDEX} option).

        The option is ON by default.

\OptHead{CHECKNIL}{NIL pointer dereference check}
        \MLBegin{}\ModeC{}\MLEnd{} \inline

        If the option is set ON, the compiler generates NIL checks
        on all pointer dereferences.

        The option is ON by default.

\OptHead{CHECKPROC}{check of a formal procedure call}
        \MLBegin{}\ModeC{}\MLEnd{} \inline

        If the option is set ON, the compiler generates a NIL check
        when calling a procedure variable.

        The option is ON by default.

\OptHead{CHECKRANGE}{range checks (range types and enumerations)}
        \MLBegin{}\ModeC{}\MLEnd{} \inline

        If the option is set ON, the compiler generates range checks
        for range types and enumerations.

        The option is ON by default.

\OptHead{CHECKSET}{range check of set operations}
        \MLBegin{}\ModeC{}\MLEnd{} \inline

        If the option is set ON, the compiler generates range checks
        for set operations (\verb'INCL', \verb'EXCL', set aggregates).

        The option is ON by default.


\OptHead{CHECKTYPE}{dynamic type guards (\ot{} only)}
        \MLBegin{}\ModeC{}, \ot{} only\MLEnd{} \inline

        If the option is set ON, the compiler generates dynamic type guards.

        The option is ON by default.

\ifgenc
\OptHead{COMMENT}{copy comments into a generated C code}
        \MLBegin{}\ModeC{}\MLEnd{} \header

        If the option is set ON,  the  compiler copies comments
        to appropriate places in the generated C code.
        Comments  from  an \ot{} module are only inserted  into
        the  C code file and not into the header file.
\fi

\ifgenc
\OptHead{CONVHDRNAME}{use file name in the '\#include' directive}
        \MLBegin{}\ModeC{}\MLEnd{}

        If the option is set ON, the compiler uses a file name
        in the \verb|#include| directive. Otherwise the compiler
        generates a module name postfixed by the header file extension.

\fi

\ifgenc\else
\OptHead{COVERFLOW}{cardinal overflow check}
        \MLBegin{}\ModeC{}\MLEnd{} \inline

        If the option is set ON, the compiler generates overflow checks
        for all cardinal (unsigned) arithmetic operators.

        The option is ON by default.
\fi

\ifgenc
\OptHead{CSTDLIB}{definition of the C standard library}
        \MLBegin{}\ModeC{},{\em foreign definition only}\MLEnd{} \header

        The option should be set when compiling a foreign definition
        module, otherwise it will be ignored.
        If the option is set ON, the compiler uses angle brackets \verb+<>+
        in the \verb|#include| directive, when importing the foreign
        definition. Otherwise the compiler uses quotes.
\begin{verbatim}
    #include <stdio.h>
    #include "MyLib.h"
\end{verbatim}
\fi

\ifgencode
\OptHead{DBGNESTEDPROC}{generate information about procedure nesting}
        \MLBegin{}\ModeC{}\MLEnd{}

        If this option is set ON, the compiler includes procedure nesting data
        into debug information (CodeView and HLL formats only, see the
        \OERef{DBGFMT} equation).

        This is a non-standard feature, so a third party debugger would not
        work correctly with an executable compiled with {\bf DBGNESTEDPROC} ON.
        For instance, MSVC does not display local variables of nested procedures.

        This option is OFF by default.

\OptHead{DBGQUALIDS}{generate qualified identifiers in debug info}
        \MLBegin{}\ModeC{}\MLEnd{}

        If the option is set ON, the compiler prefixes names of
        \mt{}/\ot{} global variable with the name of the respective module
        and underscore in debug information. This feature may help you
        distingushing identically named exported variables from different
        modules in third-party debuggers that do not support \mt{}/\ot{}.

        This option is OFF by default.

\OptHead{DEFLIBS}{put the default library names into object files}
        \MLBegin{}\ModeC{}\MLEnd{}

        If the option is set ON, the compiler writes the default
        library names to the generated object files.

        The option is ON by default.
\fi

\ifgenc
\OptHead{DIFADR16}{SYSTEM.DIFADR returns 16-bits value}
        \MLBegin{}\ModeC{}\MLEnd{}

        If the option is set ON, the compiler assumes that the
        difference between addresses  on the target  platform is a
        16-bit integer value, otherwise a 32-bit integer value.

        The correct setting of the option is required for
        the \verb'ADDADR', \verb'SUBADR', and \verb'DIFADR'
        system functions to work correctly.

        See \ref{maptoc:opt:config} for further details.
\fi

\ifgencode
\ifdll
\OptHead{DLLEXPORT}{show exported objects to DLL clients}
        \MLBegin{}\ModeC{}\MLEnd{} \inline

        Setting this option ON during compilation of a \mt{} definition module
        or an \ot{} module causes exported procedures and variables
        to be visible outside a dynamic link library.

        See \ref{dll:create:export} for more information on
        controlling export from DLLs.
\fi

\OptHead{DOREORDER}{perform instruction scheduling}
        \MLBegin{}\ModeC{}\MLEnd{} \header

        Setting this option ON enables the {\em instruction scheduling}
        mechanism of the x86 code generator. It reorders CPU instructions
        so that independent instructions can be executed simultaneously
        whenever possible.

        {\bf Note}: this optimization significantly slows down the compiler,
        but results in a code perfomance gain of 5-15\%.
\fi

\OptHead{FATFS}{limit file names to 8.3}
        \MLBegin{}*\MLEnd{}

        Forces the compiler to limit file names to FAT "8.3" convention.

\OptHead{GCAUTO}{enables implicit call of the garbage collector}
        \MLBegin{}\ModeC{},{\em top-level module only}\MLEnd{} \header

        Enables implicit  calls  of the garbage collector in the
        generated  program.  The option is ignored for all modules
        except the top-level module of the program.  We recommend   to
        set the option in the  project  or configuration file.

        See also \ref{rts:mm}.

\ifgencode
\OptHead{GENASM}{generate assembly text}
        \MLBegin{}\ModeC{}\MLEnd{} \header

        If this option is set ON, the compiler generates text in the 
        assembly language instead of object files. The only assembler
        supported in the current version is GNU Assembler.
\fi
\ifgenc
\OptHead{GENCDIV}{generate C division operators}
        \MLBegin{}\ModeC{}\MLEnd{} \header

        If the option is set ON, the compiler translates \verb'DIV' and
        \verb'MOD' operators to C division operators \verb'/' and \verb'%'.
        By default, \xds{} generates ???.

\OptHead{GENCPP}{generate C++}
        \MLBegin{}\ModeC{}\MLEnd{} \header

        If the option is set ON, the compiler generates C++ code.
        By default, ANSI C code is generated.
        (See also \OERef{GENKRC}).

\OptHead{GENCONSTENUM}{generate enumeration as constants}
        \MLBegin{}\ModeC{}\MLEnd{} \header

        If this option is set ON, a \mt{} enumeration type is translated
        to a set of integer constants. Otherwise, it is translated to
        a C enumeration (\verb'enum').
\fi

\ifgencode
\OptHead{GENCPREF}{generate underscore prefixes}
        \MLBegin{}\ModeC{}\MLEnd{} \header

        If the option is set ON, the compiler uses underscore as
        a prefix for all public names in object files.
% !!! now it can be specified inline !!!

\fi

\ifgenc
\OptHead{GENCTYPES}{generate C types}
        \MLBegin{}\ModeC{}\MLEnd{} \header

        If the option is set ON, the compiler generates standard C type
        names whenever possible, otherwise names defined in
        run-time support are used.

        See \ref{maptoc:opt:gen} for further details.
\fi

\ifgenc
\OptHead{GENDATE}{insert a date in a C file}
        \MLBegin{}\ModeC{}\MLEnd{}

        If the option is set ON (by default), the compiler inserts
        the current date into a generated C file.
\fi

\OptHead{GENDEBUG}{generate code in the debug mode}
        \MLBegin{}\ModeC{}\MLEnd{} \header

\ifgencode
        If the option is set ON, the compiler puts
        debug information into an object file.

        In some cases, switching the option ON may reduce
        code quality.

        See also the \OERef{DBGFMT} equation.
\fi
\ifgenc
        If the option is set ON, the compiler generates code in
        debug mode. If your program is compiled in this mode, the
        run-time system prints a stack of procedure calls (a file
        name and a number of a line) on abnormal termination of your
        program. If the option \OERef{LINENO} is also ON, the stack
        is printed in terms of original (\ot{}/\mt{}) source files,
        otherwise a file name and a number of a line of generated C files
        is printed.

        {\bf Note:} Setting the option ON will significantly enlarge
        your program and slow it down.

        See \ref{maptoc:opt:gen} for further details.
\fi

\ifgencode
\ifdll
\OptHead{GENDLL}{generate code for dynamic linking}
        \MLBegin{}\ModeC{}\MLEnd{}

        Setting this option ON forces generation of object and symbol files
        suitable for building DLLs. In most cases, the \OERef{USEDLL} option
        should be also turned on.

        See Chapter \ref{dll} for details on building DLLs.
\fi

\OptHead{GENFRAME}{always generate a procedure frame}
        \MLBegin{}\ModeC{}\MLEnd{} \header

        If the option is set ON, the compiler always generates a stack frame.
        It may be necessary to simplify debugging.
\fi

\ifgenc
\OptHead{GENFULLFNAME}{generate full name in '\#lineno' directive}
        \MLBegin{}\ModeC{}\MLEnd{}

        If the option is set ON, the compiler generates full file names,
        including file paths from redirection directives,
        in \verb'#lineno' directive.
\fi

\OptHead{GENHISTORY}{enable postmortem history}
        \MLBegin{}\ModeC{}\MLEnd{} \header

% !!! \gencode only !!!
% !!! errinfo.$$$

        If the option is set ON, the run-time system prints a stack of
        procedure calls (a file name and a line number) on abnormal
        termination of your program. It should be set when compiling a
        main module of the program. In this case the required part of
        the run-time system will be added to the program.  The option
        \OERef{LINENO} should be set for all modules in the program.

        See \ref{start:debug} for an example.

\ifgencode
        {\bf Note:} In some cases the printed list may contain incorrect
        lines, i.e. procedures that were not called in the given context
        (See \ref{rts:history}).
\fi

\ifgenc
\OptHead{GENKRC}{generate K\&R C}
        \MLBegin{}\ModeC{}\MLEnd{} \header

        If the option is set ON, the compiler generates K\&R C
        (no types in function prototypes, etc).
        It can be useful when porting software to platforms
        for which ANSI C compilers are not available.
        By default, the compiler generates ANSI C code.

        See \ref{maptoc:proc} and \ref{maptoc:opt:gen} for
        further details.

        {\bf Note:} Seeting this option ON removes the line length
        limit in the generated C text (see \OERef{GENWIDTH}).
\fi

\ifgenc
\OptHead{GENPROCLASS}{generate a specification of a procedure class}
        \MLBegin{}\ModeC{}\MLEnd{} \header

        If the option is set ON, the compiler inserts a special macro
        \verb|X2C_PROCLASS| call in function prototypes for \mt{} and
        \ot{} procedures. The option is set ON by default.

        See \ref{maptoc:proc} and \ref{maptoc:opt:gen} for
        further details.

\fi

\ifgenc
\OptHead{GENPROFILE}{generate additional code for profiling}
        \MLBegin{}\ModeC{}\MLEnd{} \header

        If the option is set ON, the compiler generates additional code
        to provide a profile of the program. See \ref{maptoc:opt:gen}
        for further details.

\fi

\ifgencode
\OptHead{GENPTRINIT}{generate a local pointer initialization}
        \MLBegin{}\ModeC{}, \ot{} only\MLEnd{} \header

        If the option is set ON, the compiler generates code
        for initialization of all local pointers, including
        variables, record fields and
        array elements. Values of all non-pointer
        record fields and array elements are undefined.

        The option is ON by default.
\fi

\ifgenc
\OptHead{GENSIZE}{evaluate sizes of types}
        \MLBegin{}\ModeC{}\MLEnd{} \header

        As  specified  in the language  reports the call of standard
        function  SIZE  can be used in constant expression.  Due to the
        specific of compilation to an intermediate language, the compiler
        does not know the sizes of most types during compilation.

        When the option is OFF, the compiler will report an error
        if an attempt is made to use SIZE of almost all types in
        a constant expression:

\verb'    CONST size = SIZE(MyRecord);'

        {\bf Note:} sizes of whole types, \verb'BOOLEAN', \verb'CHAR', and set types
        are known to the compiler.

        See \ref{maptoc:opt:sizeof} for further details.
\fi

\ifgenc
\OptHead{GENTYPEDEF}{generate typedef's for types}
        \MLBegin{}\ModeC{}\MLEnd{} \inline

        There  are two possible ways to generate a declaration of
        a record ({\tt struct}) type in C -- by using {\tt typedef} of
        not. In most cases the way used by the compiler
        is of no concern. However, when writing a foreign
        language interface module (See Chapter \ref{multilang}) it
        is desirable to control type declarations.

        When the option is ON, the compiler uses the {\tt typedef} form
        for all the types declared.
        The option can be used in the source text, e.g:
\begin{verbatim}
    <* GENTYPEDEF + *>
    TYPE FILE = RECORD END;
    <* GENTYPEDEF - *>
\end{verbatim}

        See \ref{maptoc:opt:gen} for further details.

\fi

\ifgencode
\ifdll
\OptHead{IMPLIB}{use import libraries}
        \MLBegin{}\ModeC{}\MLEnd{}

        When this option is set ON, the compiler assumes that import
        libraries will be used during linking to resolve references
        to external data and code contained in DLLs. Otherwise, it
        uses information contained in symbol files to form import
        definition records in output object files, allowing the
        executable to be linked without import libraries.

        See \ref{dll:using:load-time} for more information.
\fi
\fi

\ifgenc
\OptHead{INDEX16}{index is of 16 bits}
        \MLBegin{}\ModeC{}\MLEnd{}

        If the option is set ON, the compiler assumes that an index is
        of 16 bits on the target platform. By default, an index is of 32
        bits.

        See \ref{maptoc:opt:config} for further details.
\fi

\ifgenc\else
\OptHead{IOVERFLOW}{integer overflow check}
        \MLBegin{}\ModeC{}\MLEnd{} \inline

        If the option is set ON, the compiler generates overflow checks
        of all integer (signed) arithmetic operators.

        The option is ON by default.

\fi

\OptHead{LINENO}{generate line numbers}
        \MLBegin{}\ModeC{}\MLEnd{} \header

\ifgenc
        If the option is set ON, the compiler inserts
        a line of the form

\verb'    #line lineno [module]'

        into the generated  C  code  or header file
        for every generated statement, forcing the C compiler and
        other tools (such as a debugger) to refer to the original 
        \mt{} or \ot{} source text instead of the generated C code.

        {\bf Note:} Seeting this option ON removes the line length
        limit in the generated C text (see \OERef{GENWIDTH}).
\fi
\ifgencode
        If the option is set ON, the compiler inserts
        line number information into object files.
        This option should be set ON to get the postmortem
        history (See the \OERef{GENHISTORY} option) and
        for debugging.
\fi

\OptHead{LONGNAME}{use long names in batch files}
        \MLBegin{}\ModeM{},\ModeP{}\MLEnd{}

        Use long names.

        If the option is set ON, the compiler uses full
        path as a prefix for all module names
        in the generated batch files. See also
        \ref{xc:modes:batch}.

\OptHead{M2}{force the Modula-2 compiler}
        \MLBegin{}\ModeC{}\MLEnd{}

        Force the \mt{} compiler.

        If the option is set ON, the \mt{}  compiler is invoked
        regardless of file extension.  The
        option is ignored in MAKE and PROJECT modes.


\OptHead{M2ADDTYPES}{add SHORT and LONG types}
        \MLBegin{}\ModeC{},\mt{} only\MLEnd{} \header

        Add short and long modifications of whole types.

        If the option is set ON, the compiler recognizes the
        types {\tt SHORTINT}, {\tt LONGINT}, {\tt SHORTCARD}
        and {\tt LONGCARD} as pervasive identifiers .

        {\bf Warning:}  Usage of additional types may cause
        problems with the software portability to other
        compilers.

\OptHead{M2BASE16}{use 16-bits basic types in Modula-2}
        \MLBegin{}\ModeC{},\mt{} only\MLEnd{} \header

        If the option is set ON, the basic types {\tt INTEGER},
        {\tt CARDINAL}, and {\tt BITSET}
        are 16 bits wide in \mt{}. By default, they are 32 bits wide.

\OptHead{M2CMPSYM}{compare symbol files in Modula-2}
        \MLBegin{}\ModeC{},\mt{} only\MLEnd{}

        If the option is set ON, the \mt{} compiler compares the symbol file
        generated for a definition module with the old
        version exactly as the \ot{} compiler does.
        If the symbol files are equal, the old one
        is preserved, otherwise the compiler overwrites symbol
        file, but only if the \OERef{CHANGESYM} option is set ON.

\OptHead{M2EXTENSIONS}{enable Modula-2 extensions}
        \MLBegin{}\ModeC{},\mt{} only\MLEnd{} \header

        If the option is set ON, the compiler allows
        \See{\XDS{} \mt{} language extensions}{}{m2:ext}, such as line comment
        ("\verb|--|"), read-only parameters, etc.,
        to be used in the source code.

        {\bf Warning:}  Extensions usage may cause problems with
        porting to third-party compilers.

\OptHead{MAIN}{mark an Oberon-2 main module}
        \MLBegin{}\ModeC{}, \ot{} only\MLEnd{} \header

        Mark the \ot{} main module.

        If the option is set ON, the compiler generates a
        program entry point (`main' function) for the \ot{}
        module (See \ref{o2:env:main}).
        Recommended to be used in a module header.

\OptHead{MAKEDEF}{generate definition}
        \MLBegin{}\ModeC{},\ot{} only\MLEnd{}

        Forces the \ot{}
        compiler to generate a (pseudo-) definition module after
        successful compilation of an \ot{} module. The compiler
        preserves the so-called {\em exported} comments
        (i.e. comments started with `\verb|(**|')
        if the \OERef{XCOMMENTS} option is set ON.

        See \ref{o2:env:makedef}.

\OptHead{MAKEFILE}{generate makefile}
        \MLBegin{}\ModeP{}\MLEnd{}

        Forces the compiler to generate a makefile after
        successful compilation of a project.
        See also \ref{xc:modes:gen} and \ref{xc:template}.

\ifgenc
\OptHead{NOEXTERN}{do not generate prototypes for external C functions}
        \MLBegin{}\ModeC{}\MLEnd{} \inline

        If the option is set ON, the compiler does not generate C declarations for
        procedures defined as external.

        See \ref{multilang:extproc} for further details.
\fi

\OptHead{NOHEADER}{disable generation of a header file}
        \MLBegin{}\ModeC{},\ModeM{},\ModeP{}\MLEnd{} \header

\ifgenc
        If the option is set ON, the compiler does not create a C
        header file.

        See also \ref{maptoc:opt:foreign}.
\else
	This option is used by translators to C. Native code compilers
        recognize but ignore it.
\fi

\ifgenc
\OptHead{NOOPTIMIZE}{disable a set of opimizations}
\else
\OptHead{NOOPTIMIZE}{disable machine-independent opimizations}
\fi
        \MLBegin{}\ModeC{}\MLEnd{}
\ifgenc
        If the option is set OFF (by default), the compiler performs a
        set of optimizations, including constant expression evaluation,
        constant propagation, etc. If the option is ON, the compiler
        produces less efficient, but more readable text.

        We recommend to switch the option ON only if you are using XDS as a
        translator, i.e. if you will read or maintain the generated code.
\else
        If this option is set OFF (default), the machine-independent
        optimizer is invoked before code generation. Setting it ON
        causes less optimized, but still not straightforward code to
        be produced.
\fi

\ifgencode
\OptHead{NOPTRALIAS}{ignore pointer aliasing}
        \MLBegin{}\ModeC{}\MLEnd{} \header

        If the option is set ON, the compiler assumes that there is no
        pointer aliasing, i.e. there are no pointers bounded to
        non-structure variables. The only way to get a pointer to a
        variable is to use the low-level facilities from the module
        SYSTEM.  We recommend to turn this option ON for all modules
        except low-level ones.  {\bf Note:} the code quality is better
        if the option is ON.
\fi

\OptHead{O2}{force the Oberon-2 compiler}
        \MLBegin{}\ModeC{}\MLEnd{}

        Force \ot{} compiler.

        If the option is set ON, the \ot{} compiler is invoked
        regardless of the file extension.
        The option is ignored in MAKE and PROJECT modes.

\OptHead{O2ADDKWD}{enable additonal keywords in \ot}
        \MLBegin{}\ModeC{},\ot{} only\MLEnd{} \header

        Allows \mt{} \See{exceptions}{}{m2:ISO:exc} and \See{finalization}{}{m2:ISO:final}
        to be used in \ot{} programs, adding keywords \verb'EXCEPT', \verb'RETRY', and
        \verb'FINALLY'.

        {\bf Warning:} Usage of this extension will prevent your
        program from porting to other \ot{} compilers.

\OptHead{O2EXTENSIONS}{enable Oberon-2 extensions}
        \MLBegin{}\ModeC{},\ot{} only\MLEnd{} \header

        If the option is set ON, the compiler allows \ot{}
        language extensions to be used (See \ref{o2:ext}).

        {\bf Warning:} Extensions usage will affect
        portability to third-party \ot{} compilers.

\OptHead{O2ISOPRAGMA}{enable ISO Modula-2 pragmas in Oberon}
        \MLBegin{}\ModeC{},\ot{} only\MLEnd{}

        If the option is set ON, the compiler allows the ISO \mt{}
        style pragmas \verb|<* *>| to be used in \ot{}.
        See \ref{o2:pragmas} and \ref{m2:pragmas}.

        {\bf Warning:} Usage of ISO \mt{} pragmas may cause problems
        when porting source code to third-party \ot{} compilers.

\OptHead{O2NUMEXT}{enable Oberon-2 scientific extensions}
        \MLBegin{}\ModeC{},\ot{} only\MLEnd{} \header

        If the option is set ON, the compiler allows the \ot{}
        scientific language extensions to be used (See \ref{o2:ext}),
        including {\tt COMPLEX} and {\tt LONGCOMPLEX} types and the
        in-line exponentiation operator.

        {\bf Warning:}  Usage of additional types may cause
        problems with portability to other compilers.

\ifgencode
\OptHead{ONECODESEG}{generate one code segment}
        \MLBegin{}\ModeC{}\MLEnd{}

        If the option is ON, the compiler produces only one code
        segment which contains all code of a module, otherwise it generates
        a separate code segment for each procedure.

        {\bf Warning}: Setting this option ON disables smart linking.
\fi

\OptHead{OVERWRITE}{create a file, always overwrites the old one}
        \MLBegin{}*\MLEnd{}

        The option changes the way the compiler selects a directory
        for output files. If the option is OFF,
        the compiler always creates a file in the directory
        which  appears  first  in  the  search  path list
        correspondent to a pattern matching the file name.
        Otherwise, the compiler overwrites the old file,
        if it does exist in any directory of that list.
        See also \ref{xc:red}.

\OptHead{PROCINLINE}{enable in-line procedure expansion}
        \MLBegin{}\ModeC{}\MLEnd{}

        If the option is ON, the compiler tries to expand procedures
        in-line. In-line expansion of a procedure eliminates the overhead
        produced by a procedure call, parameter passing, register saving,
        etc. In addition, more optimizations become possible because
        the optimizer may process the actual parameters used in a particular
        call.

        A procedure is not expanded in-line under the following
        circumstances:
        \begin{itemize}
        \item the procedure is deemed too complex or too large by the
              compiler.
        \item there are too many calls of the procedure.
        \item the procedure is recursive.
        \end{itemize}

\ifgencode
\OptHead{SPACE}{favor code size over speed}
        \MLBegin{}\ModeC{}\MLEnd{}

        If the option is set ON, the compiler performs optimizations
        to produce smaller code, otherwise (by default)
        to produce faster code.
\fi

\OptHead{STORAGE}{enable the default memory management in Modula-2}
        \MLBegin{}\ModeC{}, \mt{} only\MLEnd{} \header

        If the option is set ON, the compiler uses the default memory
        allocation and deallocation procedures for the
        standard procedures \verb'NEW' and \verb'DISPOSE'.

        {\bf Warning:} Usage of this option may cause problems
        with software portability to other compilers.

\ifgenc
\OptHead{TARGET16}{C 'int' type is of 16 bits}
        \MLBegin{}\ModeC{}\MLEnd{}

        If the option is set ON, the compiler assumes the C {\tt
        int} type to be 16 bits wide on the target platform.

        See \ref{maptoc:opt:config} for further details.
\fi

\iftopspeed
\OptHead{TOPSPEED}{enable Topspeed \mt{}-compatible extensions}
        \MLBegin{}\ModeC{}\MLEnd{}

        Setting this option ON enables a set of language extensions that
        makes the compiler more compatible with TopSpeed.
        The extensions are described in Appendix \ref{tscp}.
\fi

\OptHead{VERBOSE}{produce verbose messages}
        \MLBegin{}\ModeM{},\ModeP{}\MLEnd{}

        If the option is set ON, the compiler will report
        a reason for each module (re)compilation (See \ref{xc:smart}).

\ifgencode
\ifdll
\OptHead{USEDLL}{use DLL version of the run-time library}
        \MLBegin{}\ModeC{}\MLEnd{}

        This options determines whether the run-time library has to be
        linked statically (OFF) or dynamically (ON). It has to be set ON if a DLL
        is being built (the \OERef{GENDLL} option is set ON), unless
        your application consists of a single DLL (see \ref{dll:foreign}).

        See Chapter \ref{dll} for more information on building
        and using DLLs.
\fi
\fi

\OptHead{VERSIONKEY}{append version key to the module initialization}
        \MLBegin{}\ModeC{}\MLEnd{}

        This option may be used to perform version checks
        at link time. If the option is set ON, the compiler generates
        a name of a module body as composition of
        \begin{itemize}
        \item a module name
        \item a string "\verb+_BEGIN_+"
        \item a time stamp
\ifgenc
        \item values of options \OERef{TARGET16}, \OERef{INDEX16} and
        \OERef{DIFADR16} in the packed form
\fi
        \end{itemize}
        If a \mt{} definition module or an \ot{} module imported by
        different compilation units has the same version, the same
        name is generated for each call of the module body.
        In all other cases unresolved references will be reported
        at link time.

        If the option is OFF, the compiler generates module body names
        in a form: \verb|<module_name>_BEGIN|.

        {\bf Note:} the option should be set when compiling
        a \mt{} definition module or an \ot{} module.

\ifgenc
        See \ref{maptoc:idents} for further details.
\fi

\ifgencode
\OptHead{VOLATILE}{declare variables as volatile}
        \MLBegin{}\ModeC{}\MLEnd{} \inline

        If this option appears to be switched ON during compilation
        of a variable definition, the compiler will assumes that references
        to that variable may have side effects or that the value of
        the variable may change in a way that can not be determined
        at compile time. As a result, the optimizer will not eliminate any
        operation involving that variable, and changes to the value of
        the variable will be stored immediately.

\ifthreads
        See \ref{threads:volatile} for more information on volatile
        variables usage in multithread programs.
\fi
\fi

\OptHead{WERR}{treat warningns as errors}
        \MLBegin{}*\MLEnd{}  \inline

        When the option \verb'WERRnnn' (e.g. \verb'WERR301') is set ON, the
        compiler treats the warning \verb'nnn' (301 in the above example) as error.
        See the \XC{}.msg file for warning texts and numbers.

        \verb'-WERR+' forces the compiler to treat all warnings as errors.

\OptHead{WOFF}{suppress warning messages}
        \MLBegin{}*\MLEnd{}  \inline

        When the option \verb'WOFFnnn' (e.g. \verb'WOFF301') is set ON, the
        compiler does not report the warning \verb'nnn' (301 in the above example).
        See the \XC{}.msg file for warning texts and numbers.

        \verb'-WOFF+' disables all warnings.

\OptHead{XCOMMENTS}{preserve exported comments}
        \MLBegin{}\ModeC{},\ot{} only\MLEnd{}

        If the option is set ON, the browser includes so-called
        {\em exported} comments
        (i.e. comments which start with "\verb'(**'")
        into a generated pseudo definition module.

        See also \ref{o2:env:makedef}.

\ifonline \else
\end{description}
\fi

\section{Equations}\label{opt:equ}
\index{equations}

An {\em equation} is a pair ({\tt name},{\tt value}), where {\tt value} is
in general case an arbitrary string. Some equations have a limited set of
valid values, some may not have the empty string as a value.

A compiler setup directive (See \ref{config:options}) is used
to set an equation value or to declare a new equation.

Equations may be set in a \See{configuration file}{}{config:cfg},
\See{on the command line}{}{xc:modes} and in a \See{project file}{}{xc:project}).
Some equations may be set in the source text, at an arbitrary position
(marked as {\em inline} in the reference), or only in the module header
(marked as {\em header}). At any point of operation, the most recent
value of an equation is in effect.

Alphabetical list of all equations may be found in the section \ref{opt:equ:list}.
\ifonline\else
See also tables
\ref{table:equ:ext} (page \pageref{table:equ:ext}),
\ref{table:equ:code} (page \pageref{table:equ:code}),
\ref{table:equ:misc} (page \pageref{table:equ:misc})
\fi

\begin{table}[htbp]
\begin{center}
\begin{tabular}{|l|c|l|}
\hline
\bf Name          & \bf Default & \bf File type \\
\hline
\OERef{BATEXT}   & \tt .bat    & recompilation batch file                     \\
\OERef{BSDEF}    & \tt .odf    & pseudo-definition file created by browser    \\
\ifgenc
\OERef{CODE}     & \tt \Code   & generated C code file                        \\
\fi
\ifgencode
\OERef{CODE}     & \tt \Code   & object file                                    \\
\fi
\OERef{DEF}      & \tt .def    & \mt{} definition module                      \\
\ifgenc
\OERef{HEADER}   & \tt \Header & generated C header file                      \\
\fi
\OERef{MKFEXT}   & \tt .mkf    & makefile                                     \\
\OERef{MOD}      & \tt .mod    & \mt{} implementation or main module          \\
\OERef{OBERON}   & \tt .ob2    & \ot{} module                                 \\
\OERef{OBJEXT}   & \tt \dotObj & object file                                  \\
\OERef{PRJEXT}   & \tt .prj    & project file                                 \\
\OERef{SYM}      & \tt .sym    & symbol file \\
\hline
\end{tabular}
\end{center}
\caption{File extensions}\label{table:equ:ext}
\index{options!file extensions}
\end{table}

\begin{table}[htbp]
\begin{center}
\begin{tabular}{|l|c|p{6.0cm}|}
\hline
\bf Name      & \bf Default & \bf Meaning \\
\hline
\iflinux
\OERef{ALIGNMENT}  & \tt 4 & data alignment ({\em please read details below})\\
\else
\OERef{ALIGNMENT}  & \tt 1 & data alignment \\
\fi
\ifgencode
\iflinux
\OERef{CC}         & \tt GCC & C compiler compatibility             \\
\else
\OERef{CC}         & \tt WATCOM & C compiler compatibility             \\
\fi
\OERef{CODENAME}   & \tt \_TEXT & Code segment name \\
\OERef{CPU}        & \tt GENERIC & CPU to optimize for \\
\fi
\ifgenc
\OERef{COPYRIGHT}   &       & copyright note                            \\
\fi
\ifgencode
\OERef{DATANAME}   & \tt \_DATA & data segment name \\
\OERef{DBGFMT}     & see desc. & debug information format \\
\ifdll
\OERef{DLLNAME}    &        & DLL name \\
\fi
\fi
\OERef{ENUMSIZE}    & \tt 4  & default size of enumeration types \\
\OERef{GCTHRESHOLD} &        & garbage collector threshold (obsolete) \\
\ifgenc
\OERef{GENIDLEN}   & \tt   30  & length of an identifier in the generated C text \\
\fi
\ifgenc
\OERef{GENINDENT}  & \tt   3   & indentation \\
\fi
\ifgenc
\OERef{GENWIDTH}   & \tt   78  & line width in the generated C text        \\
\fi
\OERef{HEAPLIMIT}  & \tt    0  & generated program heap limit    \\
%\ifgenc
%\OERef{INT64SUFFIX}& \tt   LL  & 64-bit integer-suffix        \\
%\fi
\ifgencode
\OERef{MINCPU}     & \tt  386  & CPU required for execution      \\
\OERef{OBJFMT}     & \tt \iflinux ELF \else OMF \fi & object file format \\
\fi
\OERef{SETSIZE}    & \tt 4     & default size of small set types \\
\OERef{STACKLIMIT} & \tt 0  & generated program stack limit   \\
\hline
\end{tabular}
\end{center}
\caption{Code generator equations}\label{table:equ:code}
\index{options!code control equations}
\end{table}

\begin{table}[htbp]
\begin{center}
\begin{tabular}{|l|c|p{6.5cm}|}
\hline
\bf Name      & \bf Default & \bf Meaning \\
\hline
\OERef{ATTENTION} & \tt !     & attention character in template files       \\
\OERef{BATNAME}   & \tt out   & batch file name                           \\
\OERef{BATWIDTH}  & \tt  128  & maximum line width in a batch file     \\
\OERef{BSTYLE}    & \tt  DEF  & browse style (See \ref{o2:env:makedef})      \\
\ifcomment
\OERef{COMPILE}   &       & C compiler command line (See \ref{seamless:comp}) \\
\fi
\OERef{COMPILERHEAP}  &    & heap limit of the compiler \\
\OERef{COMPILERTHRES} &    & compiler's garbage collector threshold (obsolete) \\
\OERef{DECOR}     & \tt hrtp   & control of compiler messages  \\
\ifgenc
\OERef{ENV\_HOST}  &        & host platform   \\
\OERef{ENV\_TARGET} &       & target platform \\
\fi
\OERef{ERRFMT}    & See \ref{opt:errfmt} & error message format       \\
\OERef{ERRLIM}    & \tt   16  & maximum number of errors                  \\
\OERef{FILE}      &       & name of the file being compiled           \\
\OERef{LINK}      &       & linker command line                       \\
\OERef{LOOKUP}    &       & lookup directive                          \\
\OERef{MKFNAME}   &       & makefile name                             \\
\OERef{MODULE}    &       & name of the module being compiled         \\
\OERef{PRJ}       &       & project file name                         \\
\OERef{PROJECT}   &       & project name                              \\
\OERef{TABSTOP}   & \tt 8 & tabulation alignment                      \\
\OERef{TEMPLATE}  &       & template name (for makefile)              \\
\hline
\end{tabular}
\end{center}
\caption{Miscellaneous equations}\label{table:equ:misc}
\index{options!miscellaneous equations}
\end{table}
\pagebreak % To not break the following section with tables

\section{Equations reference}\label{opt:equ:list}

Operation  modes in  which  an  equation has effect are
enclosed in square brackets  ([])  after  the equation name; the
character '*' stands  for  all  operation  modes.  For  example
[browse]  means  that  the  equation is used by the compiler in the
BROWSE  operation mode only. {\bf Note:} the compiler switches from
the MAKE and PROJECT mode to the COMPILE mode to compile a module.

\ifonline \else
\begin{description}
\fi

\EquHead{ALIGNMENT}{data alignment}
        \MLBegin{}\ModeC{}\MLEnd{} \inline

        This equation sets the {\em data alignment}. Valid values are:
        1,2,4, or 8. See
        \ifgenc \ref{maptoc:opt:sizeof} \fi
        \ifgencode \ref{lowlevel:recrep} \fi
        for further details.

        \ifgencode
          \iflinux
            {\bf Warning:} Since \XDS{} libraries are built through GCC which
            uses 4 byte alignment, you should always keep ALIGNMENT set to 4,
            unless you exactly know what you are doing.
            See \ref{multilang:ccomp} for more details.
          \fi
        \fi

\EquHead{ATTENTION}{attention character in template files}
        \MLBegin{}\ModeP{},\ModeG{}\MLEnd{}

        The equation defines an attention character which is used
        in template files ("!" by default).
        See \ref{xc:template}.

\EquHead{BATEXT}{recompilation batch file extension}
        \MLBegin{}\ModeM{},\ModeP{},{\em batch submode}\MLEnd{}

        Sets the file extension for recompilation
        batch files (by default {\bf .bat}).
        See \ref{xc:modes:batch}.

\EquHead{BATNAME}{batch file name}
        \MLBegin{}\ModeM{},\ModeP{},{\em batch submode}\MLEnd{}

        Sets the batch file name.

        The name of the project file will be used if no batch file  name
        is explicitly specified.
        See \ref{xc:modes:batch}.

\EquHead{BATWIDTH}{maximum line width in a batch file}
        \MLBegin{}\ModeM{},\ModeP{},{\em batch submode}\MLEnd{}

        Sets the maximum width of a line in a
        generated batch file (by default 128).
        See \ref{xc:modes:batch}.

\EquHead{BSDEF}{pseudo-definition file created by browser extension}
        \MLBegin{}\ModeB{}\MLEnd{}

        Sets the file extension for pseudo-definition
        modules created by the browser (by default {\bf .odf}).
        See \ref{xc:modes:browse}.

\EquHead{BSTYLE}{browse style}
        \MLBegin{}\ModeB{}\MLEnd{}

        Sets the {\em style} of generated pseudo-definition
        modules. See \ref{o2:env:makedef}.

\ifgencode
\EquHead{CC}{C compiler compatibility}
        \MLBegin{}\ModeC{}\MLEnd{}

        Sets the C compiler compatibility mode.
        The correspondent calling and naming conventions  will be used 
        for procedures and variables declared as \verb'["C"]'.

        \iflinux
          Currently the only valid value on Linux is "GCC".
        \else
          \ifwinnt
            Valid values on Windows are "WATCOM" and "MSVC".
          \else
            \JNO
          \fi
        \fi
% ???  , "BORLAND" and "SYMANTEC".

        If the value of the equation is undefined,
        \iflinux "GCC" \else "WATCOM" \fi
        is assumed.

        See \ref{multilang:ccomp} for more details.

\fi

\ifgenc
\EquHead{CODE}{generated C code file extension}
\else % gencode
\EquHead{CODE}{object file extension}
\fi
        \MLBegin{}*\MLEnd{}

        Sets the file extension for code
        files generated by the compiler
        (by default {\bf \Code}).

\ifgencode
\EquHead{CODENAME}{Code segment name}
        \MLBegin{}\ModeC{}\MLEnd{} \header

        Sets name for a code segment.
\fi

\ifcomment
\EquHead{COMPILE}{C compiler command line}
        \MLBegin{}\ModeC{}\MLEnd{}

        Defines a command line which will be executed
        after successful compilation of each module.
        As a rule, the equation is used for calling
        the C compiler. See Chapter \ref{seamless}.
\fi

\EquHead{COMPILERTHRES}{compiler's garbage collector threshold (obsolete)}
        \MLBegin{}*\MLEnd{}

        This equation is left for compatibility; it is ignored by
        the compiler. In versions prior to 2.50, it was used to fine tune
        the compiler's garbage collector. 

        See also \ref{config:memory}.

\EquHead{COMPILERHEAP}{heap limit of the compiler}
        \MLBegin{}*\MLEnd{}

        Sets the maximum amount of heap memory (in bytes),
        that can be used by the compiler.
        For systems with virtual memory, we recommend to use
        a value which is less than the amount of physical memory.

        Setting this equation to zero forces adaptive
        compiler heap size adjustment according to system load.


\ifgenc
\EquHead{COPYRIGHT}{copyright note}
        \MLBegin{}\ModeC{}\MLEnd{}

        This copyright  note  line  will be inserted as
        a comment into all generated C code/header
        files. See \ref{maptoc:layout}, \ref{maptoc:opt:rep}.
\fi

\ifgencode
\EquHead{CPU}{CPU to optimize for}
        \MLBegin{}\ModeC{}\MLEnd{}

        Specifies on which Intel x86 family representative
        the resulting program will be executed optimally.

        Valid values: "386", "486", "PENTIUM", and "PENTIUMPRO".
        The value must be "greater of equal" than the value
        of the \OERef{MINCPU} equation.

        There is also the special value "GENERIC", which means
        that the optimizer should not perform code transformations
        that may {\it significantly} reduce performance on a
        particular CPU.
\fi

\ifgencode
\EquHead{DATANAME}{Data segment name}
        \MLBegin{}\ModeC{}\MLEnd{} \header

        Sets name for a data segment.
\fi

\ifgencode
\EquHead{DBGFMT}{debug information format}
        \MLBegin{}\ModeC{}\MLEnd{}

        Sets debug information format for output object files.
        Valid values are "CodeView" and "HLL".
\fi

\EquHead{DECOR}{control of compiler messages}
        \MLBegin{}*\MLEnd{}

        The equation controls output of the \XC{} utility.
        The value of equation is a string
        that contains any combination of letters "h", "t",
        "r", "p" (capital letters are also allowed).
        Each character turns on output of
        \begin{description}
        \item[h]
                header line, which contains the name and version of
                the compiler's front-end and back-end
        \item[p]
                progress messages
        \item[r]
                compiler report: number of errors, lines, etc.
        \item[t]
                the summary of compilation of multiple files
        \end{description}

        By default, the equation value is "hrt".

\EquHead{DEF}{\mt{} definition module extension}
        \MLBegin{}*\MLEnd{}

        Sets the file extension for Modula-2 definition modules (by
        default {\bf .def}).

\ifgencode
\ifdll
\EquHead{DLLNAME}{DLL name}
        \MLBegin{}\ModeC{}\MLEnd{}

        This equation has to be set to the name of the DLL into which the
        currently compiled module will be linked. It has no effect if code
        for static linking is being generated (the \OERef{GENDLL} option is OFF).
        In the \See{PROJECT mode}{}{xc:modes:project}, if this equiation is
        not explicitly set, project file name without path and extension is assumed.
        See \ref{dll:create:env} for more information.
\fi
\fi

\EquHead{ENUMSIZE}{default size of enumeration types}
        \MLBegin{}\ModeC{}\MLEnd \inline

        Sets the default size for enumeration types in bytes (1,2, or 4).
        If an enumeration type does not fit in the current default size,
        the smallest suitable size will be taken.

\EquHead{ENV\_HOST}{host platform}
        \MLBegin{}*\MLEnd{}

        A symbolic name of the host platform. \ifgenc See also \ref{config:seamless}.\fi % !!!

\EquHead{ENV\_TARGET}{target platform}
        \MLBegin{}*\MLEnd{}

\ifgenc
        Sets a symbolic name of a target platform. A platform is a combination of
        operating system, file system, C compiler, its options, etc.
        See a list of available platforms in {\tt \cfg}. See also
        \ref{config:seamless}.
\fi
\ifgencode
        This equation is always set to a symbolic name of the target % ??? always ?
        platform (CPU/operating system).
\fi

\EquHead{ERRFMT}{error message format}
        \MLBegin{}*\MLEnd{}

        Sets the error message format. See \ref{opt:errfmt} for
        details.

\EquHead{ERRLIM}{maximum number of errors}
        \MLBegin{}*\MLEnd{}

        Sets the maximum number of errors allowed for
        one compilation unit (by default 16).

\EquHead{FILE}{name of the file being compiled}
        \MLBegin{}\ModeC\MLEnd{}

        The compiler sets this equation to the name of the currently
        compiled file. See also the \OERef{MODULE} equation.

\EquHead{GCTHRESHOLD}{garbage collector threshold (obsolete)}
        \MLBegin{}\ModeC{},{\em top-level module only}\MLEnd{}

        This equation is left for compatibility; it is ignored by
        the compiler. In versions prior to 2.50, it was used to fine tune
        the garbage collector. 

        See also \ref{rts:mm}.

\ifgenc
\EquHead{GENIDLEN}{length of an identifier in the generated C text}
        \MLBegin{}\ModeC{}\MLEnd{}

        The  maximum  length  of  an identifier in the
        generated C code (by default 30). {\bf Note:} the
        identifier length limit cannot be less than 6
        characters.  Small  values result in a more
        compact but less readable text.
        See also \ref{maptoc:idents}.
\fi

\ifgenc
\EquHead{GENINDENT}{indentation}
        \MLBegin{}\ModeC{}\MLEnd{}

        Sets indentation in the generated code
        (by default 3 characters).
\fi

\ifgenc
\EquHead{GENWIDTH}{line width in the generated C text}
        \MLBegin{}\ModeC{}\MLEnd{}

        The maximum width of a line in
        generated C code/header files (by default 78).

        {\bf Note:} This equation is ignored and the maximun length of 
        the line is not limited if at least one of the options 
        \OERef{LINENO} and \OERef{GENKRC} is set ON.
\fi

\ifgenc
\EquHead{HEADER}{generated C header file extension}
        \MLBegin{}*\MLEnd{}

        Sets the file extension for ANSI C header
        files generated by compiler (by default {\bf .h}).
\fi

\EquHead{HEAPLIMIT}{generated program heap limit}
        \MLBegin{}\ModeC{},{\em top-level module only}\MLEnd{}

        Sets the maximum amount of heap memory, that can be allocated
        by the generated program. The value is set in bytes.

        Setting this equation to zero enables the run-time system
        to dynamically adjust heap size according to application's
        memory demands and system load.

        The equation should be set when the top-level module of the
        program is compiled. We recommend to set it in a
        project file or the configuration file.

        See also \ref{rts:mm}.

\ifcomment
\EquHead{IMPEXT}{}
        \MLBegin{}\ModeM{},\ModeP{},\ModeG{}\MLEnd{}

        Sets the file extension for pseudo-implementation
        modules (by default {\bf .imp}).
        See \ref{xc:make} for details.
\fi

%\ifgenc
%\EquHead{INT64SUFFIX}{ 64-bit integer-suffix}
%        \MLBegin{}\ModeC{}\MLEnd{} \inline
%
%        Sets the suffix used to specify a constant of 64-bit integral 
%        type in the generated C code (by default {\bf LL}).
%
%        For example, Microsoft Visual C++ supports the following 64-bit
%        integer-suffix: {\bf i64}, {\bf LL}, {\bf ll}.  
%\fi


\EquHead{LINK}{linker command line}
        \MLBegin{}\ModeP{}\MLEnd{}

        Defines a command line, which will be executed after a
        successful completion of a project. As a rule, the equation is
        used for calling a linker or make utility.

        See \ref{start:build} for examples.

\EquHead{LOOKUP}{lookup directive}
        \MLBegin{}*\MLEnd{}

        Syntax:

\verb'    -LOOKUP = pattern = directory {";" directory }'

        The equation can be used for defining additional search paths
        that would complement those set in the redirection file.
        A configuration or project file may contain several
        {\bf LOOKUP} equations; they are also permitted on the
        command line.

        See also \ref{xc:red} and \ref{xc:project}.

\ifgencode
\EquHead{MINCPU}{minimal CPU required for execution}
        \MLBegin{}\ModeC{}\MLEnd{}

        Specifies an Intel x86 family representative
        which (or higher) is requried for the resulting program
        to be executed.

        Valid values: "GENERIC", "386", "486", "PENTIUM", and
        "PENTIUMPRO". For this equation, "GENERIC" is equivalent
        to "386". The value of the \OERef{CPU} equation must be
        "greater of equal" than the value of this equation.
\fi

\EquHead{MKFEXT}{makefile extension}
        \MLBegin{}\ModeG{}\MLEnd{}

        Sets the file extension for generated makefiles
        (by default {\bf .mkf}). See \ref{xc:modes:gen}.

\EquHead{MKFNAME}{makefile name}
        \MLBegin{}\ModeG{}\MLEnd{}

        Sets the name for a generated makefile. See \ref{xc:modes:gen}.

\EquHead{MOD}{\mt{} implementation or main module extension}
        \MLBegin{}*\MLEnd{}

        Sets the file extension for \mt{} implementation
        and program modules (by default {\bf .mod}).

\EquHead{MODULE}{name of the module being compiled}
        \MLBegin{}\ModeC\MLEnd{}

        The compiler sets this equation to the name of the currently
        compiled module. See also the \OERef{FILE} equation.

\EquHead{OBERON}{\ot{} module extension}
        \MLBegin{}*\MLEnd{}

        Sets the file extension for \ot{}
        modules (by default {\bf .ob2}).

\EquHead{OBJEXT}{object file extension}
        \MLBegin{}*\MLEnd{}

        Sets the file extension for object files
        (by default {\bf \dotObj}).

\ifgencode
\EquHead{OBJFMT}{object file format}
        \MLBegin{}\ModeC{}\MLEnd{}

        Sets format for output object files. Valid values are "OMF",
        "COFF" and "ELF".
\fi

\EquHead{PRJ}{project file name}
        \MLBegin{}\ModeC{},\ModeM{},\ModeP{}\MLEnd{}

        In the COMPILE and MAKE operation modes, the equation
        defines a project file to read settings from.
        In the PROJECT mode, the compiler sets this equation to
        a project file name from the command line.
        See \ref{xc:modes:project}.

\EquHead{PRJEXT}{project file extension}
        \MLBegin{}\ModeC{},\ModeM{},\ModeP{}\MLEnd{}

        Sets the file extension for project files
        (by default {\bf .prj}). See \ref{xc:modes:project}.

\EquHead{PROJECT}{project name}
        \MLBegin{}\ModeC{},\ModeM{},\ModeP{}\MLEnd{}

        If a project file name is defined, the compiler sets
        the equation to a project name without a file path and
        extension. For example, if the project file name is
        \verb'prj/Work.prj', the value of the equation is set
        to \verb'Work'. The equation may be used, for instance,
        in a template file to set the name of the executable file.

\EquHead{SETSIZE}{default size of small set types}   % \gencode only ???
        \MLBegin{}\ModeC{}\MLEnd \inline

        Sets the default size for small (16 elements or less) set
        types in bytes (1,2, or 4).
        If a set type does not fit in the current default size,
        the smallest suitable size will be taken.

\EquHead{STACKLIMIT}{generated program stack limit}
        \MLBegin{}\ModeC{},{\em top-level module only}\MLEnd{}

        Sets the maximum size of the stack in a generated program.
        The value is set in bytes.

        The equation should be set when a top-level module of a
        program is compiled. We recommend to set the option in the
        project or configuration file.

\ifgencode
        {\bf Note:} for some linkers the stack size
        should be also set as a linker option.
\fi

\EquHead{SYM}{symbol file extension}
        \MLBegin{}*\MLEnd{}

        Sets the file extension for symbol files
        (by default {\bf \Sym}). See \ref{usage:genfiles}.

\EquHead{TABSTOP}{tabulation alignment}
        \MLBegin{}\ModeG{}\MLEnd{}

        When reading text files, the compiler replaces the ASCII TAB character
        with the number of spaces required to align text (by default {\bf
        TABSTOP} is equal to 8). A wrong value may cause misplaced
        comments in a generated pseudo-definition module, incorrect
        error location in an error message, etc.  We recommend
        to set this equation to the number used in your text editor.

\EquHead{TEMPLATE}{template name (for makefile)}
        \MLBegin{}\ModeG{}\MLEnd{}

        Sets a name of a template file. See \ref{xc:template}.

\ifonline \else
\end{description}
\fi

\section{Error message format specification}\label{opt:errfmt}
\index{error message format}

The format in which \xds{} reports errors is user configurable through
the \OERef{ERRFMT} equation. Its syntax is as follows:

\verb'    { string "," [ argument ] ";" }'

Any format specification allowed in the C procedure \verb'printf' can be
used in {\tt string}.

\begin{tabular}{lcl}
\bf Argument &\bf Type &\bf Meaning \\
\hline
line         & integer  &  position in a source text   \\
column       & integer  &  position in a source text   \\
file         & string   &  name of a source file       \\
module       & string   &  module name                 \\
errmsg       & string   &  message text                \\
errno        & integer  &  error code                  \\
language     & string   &  Oberon-2 or Modula-2        \\
mode         & string   &  ERROR or WARNING or FAULT   \\
utility      & string   &  name of an utility          \\
\end{tabular}

Argument names are not case sensitive.
By default, the error message format includes the following clauses:

\begin{tabular}{lcl}
\verb+"(%s",file;+       &---& a file name                   \\
\verb+"%d",line;+        &---& a line number                 \\
\verb+",%d",column;+     &---& a column number               \\
\verb+") [%.1s] ",mode;+ &---& the first letter of an error mode  \\
\verb+"%s\n",errmsg;+    &---& an error message               \\
\end{tabular}

If a warning is reported for the file {\tt test.mod} at line 5,
column 6, the generated error message will look like this:

\verb'    (test.mod 5,6) [W] variable declared but never used'

\section{The system module COMPILER}
\label{opt:COMPILER}
\lindex{COMPILER}
\index{system modules!COMPILER}

The system module {\tt COMPILER} provides two procedures which
allow you to use compile-time values of options and equations in
your \mt{} or \ot{} program:
\begin{verbatim}
    PROCEDURE OPTION(<constant string>): BOOLEAN;
    PROCEDURE EQUATION(<constant string>): <constant string>;
\end{verbatim}

Both this procedures are evaluated at compile-time and may be used
in constant expressions.

{\bf Note:} The {\tt COMPILER} module is non-standard.

\Examples

\begin{verbatim}
Printf.printf("This program is optimized for the %s CPU\n",
              COMPILER.EQUATION("CPU"));

IF COMPILER.OPTION("__GEN_C__") THEN
  ...
END;
\end{verbatim}



