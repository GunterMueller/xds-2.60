\chapter{Configuring H2D}
\label{config}

\section{Setting up system search path}
\label{config:path}

If you installed H2D as part of an XDS package, no additional setup
is required. Otherwise you must tell your operating system
where to find the executable before using H2D.
Refer to the {\tt h2d.txt} file from the on-line documentation.

\section{Working configuration}
\label{config:files}

The H2D working configuration includes an executable file and a set of
system files:

\begin{flushleft}
\begin{tabular}{ll}
\tt h2d.red & \See{Search path redirection file}{}{config:red} \\
\tt h2d.cfg & \See{Configuration file}{}{config:cfg} \\
\tt h2d.msg & \See{Message file}{}{config:msg}
\end{tabular}
\end{flushleft}

Upon invocation, H2D tries to locate these files in the current
directory and then in the directory where H2D executable
resides. If a {\em redirection file}, {\tt h2d.red} is found,
all other files are searched for/created using paths defined in it,
otherwise the current directory is used for all input and output,
except files specified with directories.

The {\em configuration file} contains various H2D settings.
If the configuration file is not found, default settings are used.

The {\em message file} contains texts of error messages.

\section{Redirection file}
\label{config:red}
\index{redirection file}
\index{regular expressions}

Upon activation, H2D looks for a file called {\tt h2d.red} --- the
{\em redirection file}. This file defines directories in which all other
files are searched for or created. A redirection file has to be placed
in the current directory, otherwise the {\em master redirection file}
from the directory where H2D executable resides is used.

A redirection file consists of several {\em redirections}:

\verb'    Redirection = Pattern "=" directory {";" directory}'

{\tt Pattern} is a regular expression which all file names used by
H2D will be compared with. A regular expression is a string containing
certain special characters:
\begin{center}
\begin{tabular}{cl}
\bf Sequence & \bf Denotes \\
\hline
\verb+*+     & an arbitrary sequence of any characters, possibly empty \\
             & (equivalent to \verb|{\000-\377}| expression) \\
\verb+?+     & any single character \\
             & (equivalent  to \verb|[\000-\377]| expression) \\
\verb+[...]+ & one of the listed characters \\
\verb+{...}+ & an arbitrary sequence of the listed characters, possibly empty \\
\verb+\nnn+  & the ASCII character with octal code \verb|nnn|, where n is \verb|[0-7]| \\
\verb+&+     & the logical operation AND \\
\verb+|+     & the logical operation OR  \\
\verb+^+     & the logical operation NOT \\
\verb+(...)+ & the priority of operations
\end{tabular}
\end{center}

A  sequence of the  form \verb|a-b|  used  within  either
\verb|[]| or \verb|{}| brackets denotes all characters from
\verb|a| to \verb|b|.

When H2D looks for or intends to create a file, its name is sequentially
compared with all patterns from the top of the redirection file. A file
is created in the first directory of the list corresponding to the matched pattern.
A file is searched for in all directories in the list (from first to last)
until it is found or the directory list is exhausted. If a match is not found,
the file is created or searched for in the current directory. {\bf Note:}
If a match is found, the current direcory is {\em not} searched unless it is
explicitly specified in the directory list.

It is possible to put comment lines into the redirection file. A
comment line should be started with the \verb+%+ character.

\subsection*{Example}

\begin{verbatim}
*.h       = h; .; c:\bc\include
mac_*.def = macro;
*.def     = def;
mac_*.mod = macro;
*.h2d     = h2d;
\end{verbatim}

\section{Configuration file}
\label{config:cfg}

The configuration file is used to set options which control various
aspects of H2D behaviour: names of generated files, source/target language
extensions, mapping of C base types to Modula-2 types etc. It should reside
in the current directory or in the directory with H2D executable (the
{\em master} configuration file). However, it is recommended to use
a \ProjectFile{} instead of a local configuration file to specify
options for a particular set of header files.

An option is a pair ({\em name}, {\em value}). Every line in
the configuration file may contain only one option setup directive.
Arbitrary spaces are permitted. The \verb'%' character starts a
one-line comment. Option setup directives have the following syntax:

\verb'    Option = "-" name ("-" | "+" | "=" (string | integer))'

The same syntax is used for command line options and in a \ProjectFile{}.
Command-line options have the highest priority. Options specified
in a project file override the configuration file settings.

Options, their meanings and valid values are described in Chapter
\ref{options}.

\ifonline\else
Figure \ref{config:cfg:example} contains a configuration file example.
\fi

\begin{figure}[bht]
{\ifonline\else\small\fi
\begin{verbatim}
-DEFEXT  = def    % file extensions
-HEADEXT = h
-MODEXT  = d
-PRJEXT  = prj
-TREEEXT = inc
-DIREXT  = dir

-DEFPFX = h2d_    % prefix for output definition modules
-MACPFX = m_      % prefix for macro prototype modules

-BACKEND = COMMON % M2 compiler compatibility mode: C, NATIVE, COMMON
-GENMACRO-        % do not generate macro prototype modules
-GENWIDTH = 70    % maximum string length in output files
-COMMENTPOS = 0   % comment position
-CHANGEDEF+       % allow to overwrite existing definition modules
-PROGRESS+        % enable progress indicator
-CSTDLIB-         % do not set C standard library option
-CPPCOMMENTS+     % recognize C++ comments
-MERGEALL+        % merge all #included headers
-GENSEP-          % separate merged headers with comments
-GENLONGNAMES-    % prepen module name with directory names
-GENENUM = CONST  % enum transtaltion mode: CONST, ENUM, AUTO
-GENTREE+         % create file with include/merge tree
-GENDIRS+         % extract non-standrard directives
-GENROVARS+       % translate constants to read-only variables

% C BASE TYPES SYNONYMS:

-ctype = signed char        =  1, CHAR
-ctype = signed int         =  4, SYSTEM.int
  .  .  .
-ctype = long float         =  8, UNDEF
-ctype = long double        =  8, UNDEF

% MODULA-2 TYPES:

-m2type = INTEGER         = 4, SIGNED
-m2type = SHORTINT        = 1, SIGNED
  .  .  .
-m2type = SYSTEM.SET32    = 4, SET
-m2type = SYSTEM.int      = 4, SIGNED
-m2type = SYSTEM.unsigned = 4, UNSIGNED
\end{verbatim}
} % end of \small
\caption{Configuration file example}
\label{config:cfg:example}
\end{figure}

\section{Customizing H2D messages}
\label{config:msg}

The file {\tt h2d.msg} contains error messages in the form

\verb'    number text'

The following is an excerpt from {\tt h2d.msg}:

\begin{verbatim}
    001 Can't open file %s
       .  .  .
    010 Invalid use of modifier
       .  .  .
\end{verbatim}

Some messages contain format specifiers for additional arguments. In
the example above, the message number 001 contains a {\tt \%s} specifier
which is substituted with a file name when the message is printed.

In order to use a language other than English for messages it is
necessary to translate message texts, preserving error
numbers and the number and {\em order} of format specifiers.

