\ifgencode
\chapter{Optimizing a program}\label{optimize}
\index{optimizing a program}

% !!! IDE style sheets
% !!! Linker options

It sometimes happens with almost all compilers that the unoptimized
version of a program works properly, but the optimized one does not or vice versa.
If the compiler has a dozen of optimization control options it may be extremely
difficult to test the compiler itself. The compiler manufacturer has to
check all possible combinations of options. Fortunately, this
is not the case with \XDS{}.

Unlike many other compilers, \XDS{} performs optimizations by default.
Most of them may be turned off by setting the \OERef{NOOPTIMIZE} option ON.
However, the code generator always performs some low-level
optimizations\footnote{As a result, disabling optimizations
optimizer significantly {\em slows down} the complier}.
Instruction scheduling can be turned on or off using the \OERef{DOREORDER}
option. The last option that implicitly disables some of optimizations
is the \OERef{GENDEBUG} option.

There are still several ways to control the generated code.
First of all, you have to choose what is more important for you: performance
or compactness. By default, the option \OERef{SPACE} is set OFF,
forcing the compiler to favor the code effeciency.

To get the maximum performance, do the following:
\begin{itemize}
\item turn \OERef{GENFRAME} off
\item turn \OERef{SPACE} off            % !!! References for printed version
\item turn \OERef{GENDEBUG} off
\item turn \OERef{NOOPTIMIZE} off
\iflinux \else % ALIGNMENT should ALWAYS be set to 4 for Linux   % !!!
\item set \OERef{ALIGNMENT} to 4 (to 8 if your program operates with LONGREALs)
\fi
\item turn \OERef{DOREORDER} on
\item set \OERef{CPU} and \OERef{MINCPU} equations according to your
      target
\item turn run-time checks and overflow checks off
\end{itemize}
It is possible not to turn run-time checks off in the product versions
of your programs, because the code generator usually removes redundant
checks. A typical program runs only 10-15\% faster with all run-time
checks turned off (but the code size is usually significantly smaller).

Two options should be used with care:
\begin{itemize}
\item the \OERef{PROCINLINE} option allows the compiler to expand
      procedures in-line. As a rule, switching the option ON leads
      to faster but bigger code.
      However, the effect of this option depends on your
      programming style (size of procedures, etc).
\item the \OERef{NOPTRALIAS} option allows
      the compiler to assume that there is no
      pointer aliasing, i.e. there are no pointers bounded to
      non-structure variables.
      The code quality is better if the option is ON.
\end{itemize}

\paragraph{Example of project file for maximum performance}
\begin{verbatim}
    -alignment=4          % is unnecessary under Linux
    -noptralias+
    -procinline+
    -space
    -doreorder+
    -cpu=486
    -genframe

    -checkindex
    -checkrange
    -checknil
    -ioverflow
    -coverflow

    -gendebug
    -genhistory
    -lineno
    !module Foo.mod
\end{verbatim}
In some cases, it may be better to set different options for
different modules in your program. See {\tt dry.mod} from \XDS{}
samples.

\fi % end ifgencode ----------------------
