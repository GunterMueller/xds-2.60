\chapter{Low-level programming}\label{lowlevel}
\index{Low-level programming}

\section{Data representation}\label{lowlevel:datarep}

The internal representation of values of \mt{}
and \ot{} basic types is described in the tables
\ref{table:m2:rep} and \ref{table:o2:rep}. In the
table \ref{table:system:rep} the representation of system
types is described.

\begin{table}
\begin{tabular}{|l|c|l|} \hline
\bf Modula-2 type  & \bf Bits &\bf Representation \\ \hline
\tt SHORTINT       & 8     & signed \\
\tt INTEGER        & 16/32 & signed (See \ref{lowlevel:datarep:whole})\\
\tt LONGINT        & 32    & signed \\
\tt SHORTCARD      & 8     & unsigned \\
\tt CARDINAL       & 16/32 & unsigned (See \ref{lowlevel:datarep:whole})\\
\tt LONGCARD       & 32    & unsigned \\
\tt CHAR           & 8     & unsigned \\
\tt BOOLEAN        & 8/32  & unsigned (See \ref{lowlevel:datarep:bool})\\
                   &       & 0 for FALSE, 1 for TRUE \\
\tt subranges      &       & according to the base type \\
\tt REAL           & 32    & 80x87 single-precision data format \\
\tt LONGREAL       & 64    & 80x87 double-precision data format \\
\tt LONGLONGREAL   & 80    & 80x87 extended-precision data format \\
\hline
\end{tabular}
\caption{Representation of Modula-2 basic types}\label{table:m2:rep}
\end{table}

\begin{table}
\begin{tabular}{|l|c|l|} \hline
\bf Oberon-2 type  & \bf Bits &\bf Representation \\ \hline
\tt SHORTINT       & 8        & signed \\
\tt INTEGER        & 16       & signed \\
\tt LONGINT        & 32       & signed \\
\tt CHAR           & 8        & unsigned \\
\tt BOOLEAN        & 8        & unsigned byte \\
                   &          & 0 for FALSE, 1 for TRUE \\
\tt REAL           & 32       & 80x87 single-precision data format \\
\tt LONGREAL       & 64       & 80x87 double-precision data format \\
\tt LONGLONGREAL   & 80       & 80x87 extended-precision data format \\
\tt SET            & 32       & packed set \\
\hline
\end{tabular}
\caption{Representation of Oberon-2 basic types}\label{table:o2:rep}
\end{table}

\begin{table}
\begin{tabular}{|l|c|l|} \hline
\bf System type  & \bf Bits  &\bf Representation \\ \hline
\tt ADDRESS      & 32    & unsigned   \\
\tt BOOL8        & 8     & unsigned \\
\tt BOOL16       & 16     & unsigned \\
\tt BOOL32       & 32    & unsigned \\
\tt BYTE         & 8     & unsigned  \\
\tt CARD8        & 8     & unsigned \\
\tt CARD16       & 16    & unsigned \\
\tt CARD32       & 32    & unsigned \\
\tt INT8         & 8     & signed \\
\tt INT16        & 16    & signed \\
\tt INT32        & 32    & signed \\
\tt LOC          & 8     & unsigned  \\
\tt WORD         & 32    & ARRAY [0..3] OF LOC  \\
\hline
\end{tabular}
\caption{Representation of SYSTEM types}\label{table:system:rep}
\end{table}

\subsection{Modula-2 INTEGER and CARDINAL types}\label{lowlevel:datarep:whole}

If the option \OERef{M2BASE16} is OFF, objects of types \verb'INTEGER'
and \verb'CARDINAL' are 4 bytes (32 bits) long, otherwise they are
2 bytes (16 bits) long.

\subsection{Modula-2 BOOLEAN type}\label{lowlevel:datarep:bool}

A value of the type \verb'BOOLEAN' occupies 1 byte of memory.

\subsection{Modula-2 enumeration types}

Representation of enumeration type values depends on the current
\OERef{ENUMSIZE} equation setting. Values of an enumeration type which
fits the specified size (1, 2, or 4 bytes) occupy exactly that
number of bytes; otherwise the smallest suitable size from
that list is taken.

\subsection{Modula-2 set types}

Sete are represented as bit arrays. The \OERef{SETSIZE} equation
specifies the default size for small sets (1, 2, or 4 bytes).

If the option \OERef{M2BASE16} is OFF, the type {\tt BITSET}
is represented by 32 bits, otherwise by 16 bits.

\subsection{Pointer, address, and opaque types}

The \XDS{} compiler allocates 4 bytes of storage for a value
of a pointer, address, or opaque type.
Address arithmetic is implemented as 32-bit unsigned arithmetic
without overflow checks.

\subsection{Procedure types}

Procedure types are represented by 4 bytes which hold
an address of a procedure entry point in the task code segment.

\subsection{Record types}\label{lowlevel:recrep}

Records are represented by a continuous memory segment containing all
record components (fields) in a representation corresponding to their
types. The compiler aligns each field according to its size and the
current alignment (1,2,4, or 8), which may be set with the \OERef{ALIGNMENT}
equation. Fields, which sizes, being rounded to the nearest power of 2,
are less or equal to the current alignment, are placed at offsets which are
multiple of their (rounded) sizes. Offsets of all other fields
are multiples of the current alignment. Variant parts are aligned at the
largest alignment of variant fields. Size of a record is rounded so that
size of an array of such records is a multiple of the record size and
the number of elements in the array, and each record in the array is
correctly aligned.

\begin{verbatim}
    TYPE
      R1 = RECORD                (* ALIGNMENT   1   2   4  *)
             f1: CHAR;           (* f1 offset   0   0   0  *)
             f2: SYSTEM.CARD16;  (* f2 offset   1   2   2  *)
             f3: SYSTEM.CARD16;  (* f3 offset   3   4   4  *)
             f4: CARDINAL;       (* f4 offset   5   6   8  *)
             f5: CHAR;           (* f1 offset   9   10  12 *)
           END;                  (* SIZE(R1)    10  12  16 *)
\end{verbatim}


\subsection{Array types}\label{lowlevel:opendesc}

An array is represented by a continuous memory segment containing all
array elements in a representation corresponding to their type.

Note that elements within an array can be aligned, so in general  % !!!
for

\verb'    TYPE A = ARRAY [0..N-1] OF T;'

{\tt SIZE(A)} may be not equal to {\tt SIZE(T) * N}.

Open arrays, as well as procedure formal parameters of type
{\tt ARRAY OF ... ARRAY OF T}, are represented by an open array descriptor.
For an $N$-dimensional open array, the descriptor is an array of $2N$
32-bit elements, which are:
\begin{itemize}
\item   the first element is the address of the array data
\item   the second is the highest dimension
\item   for each of $N-1$ higher dimensions, the descriptor contains
        the size in bytes of a next dimension array and a number of elements
        in the dimension
\end{itemize}

Let {\tt A} be a dynamic $3$-dimensional array
of {\tt INTEGER} ({\tt SIZE(INTEGER)=2} in \ot{}) created as

\verb'    NEW(A,4,3,6)'

then its descriptor is a 6-element array containing:

\begin{verbatim}
     #0:  Address of array itself
     #1:   6
     #2:  12   (6*2)
     #3:   3   
     #4:  36   (12*3)
     #5:   4
\end{verbatim}

\section{Sequence parameters}\label{lowlevel:seqrep}

The array of bytes which is passed to a procedure in place of a formal
SEQ-parameter is formed as follows:
\begin{itemize}
\item
  values of all actual parameters forming the sequence are represented
  as described below and concatenated in the array in their textual order
\item integer values are converted to \verb'LONGINT'
\item \verb'BOOLEAN', \verb'CHAR', cardinal, and enumeration values are converted
      to \verb'LONGCARD'
\item range type values are converted according to their base type
\item real values are converted to \verb'LONGREAL'
\item pointer, address, opaque, and procedure type values
are converted to \verb'ADDRESS'
\item
    a structured value (record or array) is interpreted
    as a one-dimensional array of bytes and
    is represented by a 3-element descriptor:
    \begin{itemize}
    \item the address of the structure
    \item a zero 32-bit word (reserved for future extensions)
    \item size of the structure (in LOCs) minus one
    \end{itemize}
\end{itemize}

\Example
\begin{verbatim}
    PROCEDURE write(SEQ args: SYSTEM.BYTE);
    BEGIN
    END write;

    VAR i: INTEGER;
        c: SYSTEM.CARD8;
        r: LONGREAL;
        S: RECORD a: LONGINT; c: CHAR END;
        p: POINTER TO ARRAY OF CHAR;
     .  .  .

    write(i,c,S,r,p^);
\end{verbatim}

For this call the actual byte array passed to {\tt write} will contain:
\begin{itemize}
\item  4 bytes of the sign-extended value of {\tt i}
\item  4 bytes of the zero-extended value of {\tt c}
\item  12 bytes of the array descriptor
\begin{itemize}
    \item 4 bytes containing the address of {\tt S}
    \item 4 bytes containing 0
    \item 4 bytes containing 4 (\verb'SIZE(S)-1')
\end{itemize}
\item  8 bytes value of {\tt r} in the double-precision 80387 format
\item  12 bytes of the array descriptor
\begin{itemize}
    \item 4 bytes containing the address of the \verb'P' data
    \item 4 bytes containing the value 0
    \item 4 bytes containing \verb'SIZE(p^)-1'
\end{itemize}
\end{itemize}

\section{Calling and naming conventions}
\label{lowlevel:conv}

The calling and naming conventions for \mt{}, \ot{}, and
foreign procedures are described in this section.

\subsection{General considerations}

All parameters are always passed on the stack. The number of bytes
occupied by a parameter is a multiple of 4. High-order bytes of
parameters which are of shorter types (e.g. \verb'CHAR',
\verb'SYSTEM.CARD16') are undefined.

Value parameters of scalar types (boolean, character, enumeration, whole,
range, real, pointer, opaque, and procedure) and sets of size not
greater than 32 bit are placed onto the stack. A complex type value
parameter is passed as a pair of real.

Value parameters of all other types (even an array of a single \verb'CHAR')
are passed by reference. A procedure is responsible for copying its non-scalar
value parameter onto the stack, unless it is marked as read-only.

{\bf Warning:} In C, a {\em caller} should copy value parameters of
structure type onto the stack. You should provide a wrapper C function
which receives these parameters by reference. Fortunately, this is
a very rare case.

{\bf Note:} The number of 4-byte words pushed onto the stack is
passed in the AL register to a \verb'"SysCall"' foreign procedure.

\subsection{Open arrays}

For an $N$-dimensional open array parameter $N+1$ parameters are
actually passed --- the address of the array and its sizes in
all dimensions from left to right. This is true for \mt{} and
\ot{} procedures only. In case of a foreign procedure,
only the address is passed.

\subsection{Oberon-2 records}

To a formal VAR-parameter which type is an \ot{} record type,
the address of the actual parameter and the address of its dynamic type
descriptor are passed.

\subsection{Result parameter}
\label{lowlevel:conv:result}

If a function procedure result type is not scalar, it receives one extra
parameter --- the address of a temporary variable in which the
procedure should store the reslut. {\bf Note:} This may be incompatible with C.

A complex result is returned as a record with two real fields.

\subsection{Nested procedures}
\label{lowlevel:conv:nested}

A nested \mt{} or \ot{} procedure, which access scopes of outer
procedures, receives their {\em bases} as extra parameters.
More precisely, the procedure $P$ receives bases of all outer procedures
which scopes are accessed by $P$ or any procedure nested in $P$.

The base of a procedure is the address at which the procedure's
return address resides on the stack.

\subsection{Oberon-2 receivers}
\label{lowlevel:conv:o2receivers}

An extra parameter --- receiver --- is passed to an \ot{} type
bound procedure. A reference to its dynamic type descriptor
is also passed if the receiver is declared as a VAR-parameter.

\subsection{Sequence parameters}
\label{lowlevel:conv:seq}

Sequence parameters for a \mt{}/\ot{} procedure are collected
into a temporary variable, which is then passed as an \verb'ARRAY OF BYTE'
(i.e. its address and size are passed). For foreign procedures,
a C-compatible approach is used --- parameters are pushed onto the
stack. In either case, all ordinal type parameters are extended to
4 bytes, \verb'REAL's to \verb'LONGREAL's, non-scalar type parameters are
passed by reference.

\subsection{Order of parameters}

The abstract order of parameters (all categories are optional):

\begin{itemize}
\item \See{address of a temporary variable to store the result}{}{lowlevel:conv:result}
\item \See{bases of outer procedures}{}{lowlevel:conv:nested}
\item \See{receiver}{}{lowlevel:conv:o2receivers} (Oberon-2 procedures only)
\item regular parameters
\item \See{sequence parameter}{}{lowlevel:conv:seq}
\end{itemize}

Actual order, in all cases except \verb'"Pascal"' foreign procedures,
is from-right-to-left, i.e. the last sequence parameter is pushed
onto the stack first, the result parameter is pushed last.

\subsection{Stack cleanup}

The stack space allocated for parameters has to be freed upon return
from a procedure. Depending on the language of the procedure,
it is performed by the caller (\verb'"C"' and \verb'"SysCall"') or
the procedure itself (\mt{}, \ot{}, \verb'"StdCall"', \verb'"Pascal"').

\subsection{Register usage}

A procedure must preserve reisters \verb'ESI', \verb'EDI', \verb'EBP',
and \verb'EBX' registers, keep \verb'ES=DS', and clear the \verb'D' flag.

The FPU stack must be empty before a call to a procedure and upon
return from it. Exceptions are procedures which return \verb'REAL' or
\verb'LONGREAL'. In this case, the result is placed in \verb'ST(0)'.

{\bf Note}: If the \OERef{CC} equation is set to either \verb'"WATCOM"' or
\verb'"SYMANTEC"', foreign procedures declared as \verb'"C"' are considered
to return \verb'REAL' results in \verb'EAX', and \verb'LONGREAL' results
in \verb'EAX' (low order bytes) and \verb'EDX' (high order bytes).

\subsection{Naming conventions}

External names of exported procedures in object modules are
built accroding to the following rules:

\begin{center}
\begin{tabular}{lll}
\bf Convention & \bf Name is & \bf As in \\
\hline
\tt "Modula"       & prepended with the module name and "\_" & \verb'Module_Proc' \\
\tt "Oberon"       & ditto                                 & \verb'Module_Proc' \\
\tt "C"            & prepended with "\_" (see note)        & \verb'_Proc'       \\
\tt "Pascal"       & capitalized                           & \verb'PROC'        \\
\tt "StdCall"      & unchanged                             & \verb'Proc'        \\
\tt "SysCall"      & unchanged                             & \verb'Proc'        \\
\end{tabular}
\end{center}

{\bf Note:} If the \OERef{CC} equation is set to \verb'"WATCOM"',
external names of \verb'"C"' foreign procedures are {\em} not prepended
with an underscore character.

%input{codeproc.tex} % -------------------------------
