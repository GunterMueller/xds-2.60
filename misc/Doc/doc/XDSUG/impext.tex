If the definition module is found, and implementation one is not, \xds{}
tries to find a {\em pseudo implementation} module
which contains only the import list of the module:
\begin{verbatim}
IMPLEMENTATION MODULE module_name;
IMPORT module_list;
END module_name.
\end{verbatim}

The equation {\bf IMPEXT} is used to specify a file name extension
for pseudo implementation modules (default {\bf .imp}).
\index{.imp (See IMPEXT)}
Pseudo implementation modules are useful
when a project contains a module written in a foreign language,
and this module imports a \mt{} or \ot{} module.
Let a project contain modules {\bf A}, {\bf B} and {\bf C},
where
        \begin{center}
        \begin{tabular}{|c|c|l|} \hline
        \bf Module name & Import  & Language \\ \hline
        \bf A           & B       & \mt{}    \\ \hline
        \bf B           & C       & C        \\ \hline
        \bf C           &         & \mt{}    \\ \hline
        \end{tabular}
        \end{center}
To find all module in project \xds{}
needs a method to determine which modules are imported by
the module {\bf B}. It can be done with the help of pseudo
implementation module (file {\tt B.imp}):
\begin{verbatim}
IMPLEMENTATION MODULE B;
IMPORT C;
END B.
\end{verbatim}
In this example we assume that the definition module {\tt B.def}
is already included in the project.
