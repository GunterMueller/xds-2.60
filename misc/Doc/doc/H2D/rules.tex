\chapter{Translation Rules}
\label{rules}

\section{Comments}
\label{rules:comments}

All comments from the original C text are copied to generated definition
modules. Their placement, however, is not preserved in some cases.
The \OERef{COMMENTPOS} option may be used to align comments
which are placed next to declarations.

C++ compilers are usually able to recognize C++-style comments (beginning with
'{\tt //}') even while operating in C mode. The \OERef{CPPCOMMENTS}
option controls whether H2D recognizes such comments as well.

\section{Identifiers}
\label{rules:ids}

In most cases, H2D preserves original C identifiers. Exceptions are
structure, union, and enumeration tags, which constitute a separate name
space in C. If there is a constant, type, variable, or function identifier
which coincides with a tag, H2D appends "\verb'_struct'", "\verb'_union'"
or "\verb'_enum'" to that tag.

In some situations, H2D itself generates additional identifiers, e.g. for
\See{unnamed function arguments}{}{rules:functions},
\See{derived types}{}{rules:types:derived}, and
\See{formal types}{}{rules:functions}.

H2D may append digits to generated identifiers to avoid conflicts with
existent ones.

Identifiers matching Modula-2 {\em keywords} are not allowed in source files.
However, Modula-2 {\em pervasive identifiers} (e.g. \verb'INTEGER' or
\verb'HALT') are permitted.

\subsection*{Example}

{\samepage
The following C declarations were taken from the \verb'sys/stat.h' file:

{\ifonline\else\small\fi
\begin{verbatim}
struct stat { ... };
int stat( const char *, struct stat * );

TYPE stat_struct = RECORD ... END;
     PtrChar = POINTER TO CHAR;
PROCEDURE stat(arg0: PtrChar; arg1: stat_struct): SYSTEM.int;
\end{verbatim}
} % \small
} % \samepage

\section{Types}
\label{rules:types}

C types are translated to Modula-2 types according to the following table:

\begin{center}
\begin{tabular}{ll}
\bf C type          & \bf Modula-2 type \\
\hline
base ({\tt int}, {\tt char}, etc.)  & (see \ref{using:modrules:mapping}) \\
\tt void*           & \tt SYSTEM.ADDRESS \\
pointer             & pointer        \\
array               & array          \\
enumeration         & (see \ref{rules:types:enum}) \\
structure           & record         \\
union               & variant record \\
pointer to function & procedure type
\end{tabular}
\end{center}

{\bf Notes:}

\begin{itemize}
\item Structure, union, and enumeration {\em tags} are not preserved in cases of
      collision. It may cause problems with Modula-2 to C convertors (e.g. XDS-C).
      See \ref{rules:ids}.
\end{itemize}

\subsection*{Examples}
{\samepage
{\ifonline\else\small\fi

%!!! More examples needed
\begin{verbatim}
struct STRUCTURE{       TYPE STRUCTURE = RECORD
  int    field1;            field1: SYSTEM.int;
  char   field2;            field2: CHAR;
  double field3;            field3: LONGREAL;
};                        END;

union  UNION {          TYPE UNION = RECORD
  int    field1;            CASE : INTEGER OF
  char   field2;             0: field1: SYSTEM.int;
  double field3;            |1: field2: CHAR;
};                          |2: field3: LONGREAL;
                            END;
                          END;

\end{verbatim}
} % \small
} % \samepage

\subsection{Derived types}
\label{rules:types:derived}

For objects declared as having derived types (pointer or array)
either a new \mt{} type is introduced or a previously declared type synonym
is used to improve readability. For pointers to base types,
a new type is always declared.

In particular, a C structure may contain fields which type
is defined as pointer to that structure. In this case H2D
also automatically inserts a necessary forward pointer type
declaration.

This may cause type compatibility problems.
Fortunately, in \See{XDS}{Appendix }{XDS}, compatibility
rules for foreign objects are relaxed, e.g. two "C" pointer types
are compatible if their {\em base types} are the same.
Additional setup or postprocessing may be required when H2D
is used with third-party Modula-2 compilers.

{\ifonline\else\small\fi
{\samepage
\begin{verbatim}
char *str1;             TYPE
char *str1;               H2D_PtrSChar = POINTER TO CHAR;

                        VAR
                          str1: H2D_PtrSChar;
                          str2: H2D_PtrSChar;
\end{verbatim}
} % \samepage
\pagebreak[1]
{\samepage
\begin{verbatim}
struct s {              TYPE
  int i;                  s = RECORD
};                          i: SYSTEM.int;
                          END;
struct s *p;              H2D_Ptrs = POINTER TO s;

                        VAR
                          p: H2D_Ptrs;
\end{verbatim}
} % \samepage
\pagebreak[1]
{\samepage
\begin{verbatim}
struct s {              TYPE
  int i;                  s = RECORD
};                          i: SYSTEM.int;
                          END;

typedef struct s *ps;     ps = POINTER TO s;

struct s *p;            VAR
                          p: ps;
\end{verbatim}
} % \samepage
\pagebreak[1]
{\samepage
\begin{verbatim}
struct Node {           TYPE
   int i;                 PtrNode = POINTER TO Node;
   struct Node *next;     Node = RECORD
};                          i: SYSTEM.int;
                            next: PtrNode;
                          END;
\end{verbatim}
} % \samepage
} % \small
\pagebreak[1]

\subsection{Enumeration}
\label{rules:types:enum}

An enumeration (\verb'enum') is not actually a distinct type in C ---
it is just a convenient way to declare integer constants (but in
C++ enumeration {\em is} a distinct type). Moreover, since it is
possible in C to explicitly specify enumeration constant value,
translation to Modula-2 enumeration type may be incorrect.
H2D may translate C enumerations into either Modula-2
enumeration types or Modula-2 constant declarations, depending
on the \OERef{GENENUM} option setting.
For instance, if \OERef{GENENUM} is set to \verb'"Const"' or
\verb'"Mixed"', the following C type synonym declaration:

\verb'    typedef enum{ one=1, two } Number;'

will be translated to

{\samepage
\begin{verbatim}
    (* H2D: enumerated type: Number *)
    CONST
      one = 1;
      two = 2;
    TYPE
      Number = SYSTEM.int;
    (* H2D: End of enumerated type: Number *)
\end{verbatim}
} % \samepage

If \OERef{GENENUM} is set to \verb'"Enum"', the same
declaration will be translated unsafely (a warning comment will be added):

\begin{verbatim}
    TYPE
      Number = (
        one, (* H2D: integer value was 1 *)
        two
      );
\end{verbatim}

\section{Type synonyms}
\label{rules:typedef}

C declarations of type synonyms ({\tt typedef}) are translated to Modula-2 type
declarations. If there are multiple synonyms declared for a type, their
equivalence is preserved:

\begin{verbatim}
typedef char String[256];  TYPE String = ARRAY [0..255] OF CHAR;
typedef String *PString;        PString = POINTER TO String;
typedef String *Buffer;         Buffer = PString;
\end{verbatim}

{\bf Note:} In C, function type synonyms may be used in function
declarations. These synonyms are processed in a way they are
processed by a C compiler and do not appear in output files
(see \ref{rules:functions}).

\section{Variables}
\label{rules:variables}

Variables are translated to variables. Variables declared with the
{\tt const} qualifier are translated to read-only variables (XDS extension).
The {\tt volatile} qualifier is currently ignored.  % !!! <* +VOLATILE *>

\begin{verbatim}
extern int i;              VAR i  : SYSTEM.int;
extern const int j;        VAR j- : SYSTEM.int;
\end{verbatim}

% !!! Variables with initialization go to C-part only.

\section{Function prototypes}
\label{rules:functions}

C function prototypes declared as {\tt void} are translated to proper procedure
declarations; other are translated to function procedure declarations.
If there is no name specified for a function parameter, {\tt arg}{\it x}
is substituted, where {\it x} is a number unique for each unnamed parameter.

In C, a derived type (more precisely, pointer or array) may be specified for a
function parameter. In Modula-2, the {\em formal type} of a procedure parameter
have to be either type name or open array type. H2D translates parameters of
array type to open array value parameters.

The translation procedure for pointers is more complicated. By default,
H2D searches for a type synonym, previously declared via \verb'typedef'.
The synonym, if found, is used as formal type; otherwise H2D automatically
declares one. If automatic declaration is undesirable, required synonyms
may be declared in the \ProjectFile{}. Other variants of
translation may be explicitly specified by means of the \PPDir{variant}.

See also \ref{rules:qualifiers}.

\subsection*{Examples}

{\ifonline\else\small\fi
\begin{verbatim}
void p(int,int);              PROCEDURE p ( arg0 : SYSTEM.int;
                                            arg1 : SYSTEM.int );

int f(char c);                PROCEDURE f ( c : CHAR ): SYSTEM.int;

void P(T *t);                 TYPE PtrT = POINTER TO T;
                              PROCEDURE P ( t : PtrT );

void Q(T t[])                 PROCEDURE Q ( t : ARRAY OF T );

int strlen(char *);           PROCEDURE strlen ( arg0 : ARRAY OF CHAR )
#variant strlen(0) : ARRAY                     : SYSTEM.int;

\end{verbatim}
} % end \small

\section{Non-standard qualifiers}
\label{rules:qualifiers}

In practice, header files are not "pure" ANSI C. The most common
extension is a set of additional keywords (qualifiers) which may be used to
specify calling/naming conventions used in a particular library or API.

Since XDS provides the similar mechanism called {\em direct language
specification (DLS)}, H2D recognizes a number of such keywords, which are
translated to the following DLS strings:

\begin{center}
\begin{tabular}{ll}
\bf C keyword & \bf DLS String \\
\hline
\tt cdecl     & none \\
\tt fortran   & none \\      % !!! warning
\tt interrupt & none \\      % !!! warning
\tt pascal    & {\tt "Pascal"} for types and variables \\
              & {\tt "StdCall"} for functions \\ % ???
\tt syscall   & \tt "Syscall" \\
\end{tabular}
\end{center}

{\tt near}, {\tt far}, and {\tt huge} qualifiers are recognized but
ignored.

\section{Preprocessor directives}
\label{rules:pp}

\subsection{Macro definitions}
\label{rules:pp:define}

A {\tt \#define} C preprocessor directive may contain an {\em object-like}
definition or a {\em function-like} definition which are translated
differently.

\verb'#define identifier Text'

If {\tt Text} is a constant expression, the directive is translated to
a constant declaration or a read-only variable declararion
(see \ref{using:modrules:constants}).
If {\tt Text} is a type identifier, the directive is translated to
a type declaration. If {\tt Text} is an identifier of a function, a macro
definition, or a constant, the directive is translated to a constant declaration.
In all other cases, it is interpreted in a C preprocessor manner.

\verb'#define identifier"(" idenifier, { identifier } ")" Text'

Translated to a proper procedure declaration with all parameters having
type {\tt ARRAY OF SYSTEM.BYTE}. These declarations are marked with a special
comment and may be corrected after translation to reflect the actual semantics
of a macro by changing parameter types and/or adding return types.
See also \ref{using:fit:native} for information about macro prototype modules.

\verb'#undef identifier'

Undefines {\tt identifier} as it is done by a C preprocessor.

\subsubsection*{Example}

{\ifonline\else\small\fi
\begin{verbatim}
#define str_constant "Hello World!\n"

#define constant 0x10
#define constant_synonym constant

#define macro_with_params(p1,p2,p3) p1+p2+p3
#define macro_with_params_synonym macro_with_params

int function(int);
#define function_synonym function

typedef int INT;
#define INTEGER INT
\end{verbatim}

\Sep %------------------------------------------------------

\begin{verbatim}
CONST
  str_constant = 'Hello World!' + 12C;
  constant = 10H;
  constant_synonym = constant;

<* IF __GEN_C__ THEN *>

(* H2D: this procedure was generated from Macro. *)
PROCEDURE macro_with_params ( p1, p2, p3: ARRAY OF SYSTEM.BYTE );

<* ELSE *>

PROCEDURE  / macro_with_params ( p1, p2, p3: ARRAY OF SYSTEM.BYTE );

<* END *>

<* IF __GEN_C__ THEN *>

CONST
  macro_with_params_synonym = macro_with_params;

<* END *>

PROCEDURE function ( arg0: SYSTEM.int ): SYSTEM.int;

CONST
  function_synonym = function;

TYPE
  INT = SYSTEM.int;

  INTEGER = SYSTEM.int;
\end{verbatim}
} % end \small

\subsection{File inclusion}
\label{rules:pp:include}

\begin{verbatim}
#include <file_name>
#include "file_name"
\end{verbatim}

If the file specified by {\tt file\_name} has to be merged with the current file
(see \ref{using:merge}), H2D treats this directive exactly as a C preprocessor,
i.e. replaces it with contents of a specified file. Otherwise,
{\tt file\_name} is added to the import list and the file specified by it is
translated into a separate definition module. If \verb'file_name' contains
directories, the output files are placed to the same subdirectory.

The \OERef{GENLONGNAMES} option controls conversion of included header
file names which contain path:

\verb'    #include <sys/stat.h>'

is translated to

\verb'    IMPORT stat;    '

if \OERef{GENLONGNAMES} is OFF and to

\verb'    IMPORT sys_stat;'

if \OERef{GENLONGNAMES} is ON.

See also \ref{rules:modulenames}.

\subsection{Conditional compilation}
\label{rules:pp:conditional}

H2D handles conditional compilation directives {\tt \#if}, {\tt \#ifdef},
{\tt \#ifndef}, {\tt \#else}, and {\tt \#endif} the same way as a C preprocessor
does. A \ProjectFile{} may be used to define constants which are used
in arguments of these directives.

\subsection{Other directives}
\label{rules:pp:other}

H2D recognizes and ignores {\tt \#line}, {\tt \#error}, and {\tt \#pragma}
C preprocessor directives.

\section{Non-standard preprocessor directives}
\label{rules:nonstandard}

H2D recognizes two non-standard preprocessor
directives: {\tt \#merge} and {\tt \#variant}.
These directives are related to definiton module
generation only and do not affect the C text, so they may be placed
arbitrarily in a header file. Typically they are collected
in project files inside a corresponding \ProjDir{header}.

The advanced technique is to put these directives right into working
copies of header files, next to the corresponding declarations.
Then, after successful translation of all headers, these
directives may be extracted with the help of \OERef{GENDIRS}
option and moved to the project file. Now original headers
may be used for translation.

\subsection{\#merge}
\label{rules:nonstandard:merge}

\begin{verbatim}
#merge ( <file_name> | "file_name" )
\end{verbatim}

This directive lists included header files which should be merged even
if the \OERef{MERGEALL} option is OFF. This feature may be
useful in some cases (see \ref{using:merge}).

When placed in a header file, this directive has effect only in this file.
When placed in a \ProjectFile{}, it has effect in all headers matching
the surrounding \ProjDir{header}.

\subsection{\#variant}
\label{rules:nonstandard:variant}

\begin{verbatim}
#variant Designator ":" Type
Designator = identifier { "^" | "[ ]" | "." identifier } |
             Parameter
Parameter  = identifier "(" number ")"
Type       = qualident
\end{verbatim}

This form of the \verb'#variant' directive allows to explicitly
specify a \mt{} \verb'Type' for an object denoted by \verb'Designator'.
See \ref{using:modrules:mapping} for more information.

{\tt Designator} specifies a named object or its element
which is subject to the \verb'#variant' directive:

\begin{flushleft}
\begin{tabular}{ll}
\tt "\^{}"         & pointer dereference \\
\tt "[ ]"          & array indexing \\
\tt "." identifier & structure or union field selection
\end{tabular}
\end{flushleft}

{\tt Parameter} specifies a function \verb'identifier' and its
parameter \verb'number' (zero-based).


\begin{verbatim}
#variant Parameter  ":" ( "VAR" | "ARRAY" | "VAR ARRAY" )
\end{verbatim}

This form is used to control translation of a
function \verb'Parameter', which has a pointer type.

By default, pointer type function parameters are translated to
pointer type procedure parameters (see \ref{rules:functions}).
The \verb'#variant' directive allows to specify one of the
following alternative rules for a particular parameter:

\begin{center}
\begin{tabular}{ll}
\tt Modifier  & {\tt T *p} is translated to \\
\hline
\tt VAR       & \tt VAR p: T           \\
\tt ARRAY     & \tt p: ARRAY OF T      \\
\tt VAR ARRAY & \tt VAR p: ARRAY OF T
\end{tabular}
\end{center}

See also \ref{using:modrules:parameters}.

\begin{verbatim}
#variant f(0) : VAR ARRAY
void f(char*);
\end{verbatim}

\begin{verbatim}
PROCEDURE f ( VAR arg0: ARRAY OF CHAR );
\end{verbatim}

A {\tt \#variant} directive has effect only in the file where it
is located, or, if specified in a project file, in all files matching
the surrounding {\tt !header} directive. Therefore, {\tt Designator}
should specify an object declared in this file or in one of the files
which it includes. If an object specified by {\tt Designator} is not present,
an error message is displayed.

\section{Module names}
\label{rules:modulenames}
By default, H2D uses a header file name without "\verb'.h'" extension as
a definition module name. If a file name contains characters which are
not allowed in identifiers, the \ProjDir{name} must be used
in a \ProjectFile{} to specify a proper identifier.

% !!! Have to write about all XDS extensions used.
