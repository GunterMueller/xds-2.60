\chapter{Creating multithread programs}
\label{threads}

Multithreading is an operating system feature that is supported by Native XDS-x86
for Windows. This chapter describes
the creation of multithread programs using Native XDS-x86 and discusses
restrictions that multithreading imposes on these programs.

\section{Overview}

A {\em thread} is the smallest unit of execution within a process.
Each thread has its own stack and registers; other resources
of a process, such as global data, files, pipes, etc., are shared by
all its threads. Any process has at least one primary thread, also
called the main thread or thread 1.
It is started by the operating system when the process is created
and may itself create other threads when necessary. When the main thread
terminates, the whole process terminates.

All threads of a process are executed concurrently in the sense that the
operating system allocates small amounts of CPU time ({\em time slices})
to threads according to their {\em priorities}. This process is called
{\em scheduling}. Of course, on a multiprocessor system true concurrency
is possible, so the application may execute faster.

The application has no way to precisely control scheduling of its threads.
It may, however, suspend threads, block them on semaphores, and % \ref !!!
modify their priorities.

\section{The multithread library}
\label{threads:mtlib}

Native XDS-x86 includes two implementations of the runtime library: singlethread
and multithread:

\begin{tabular}{ll}
\tt LIBXDS.LIB   & singlethread for static linking \\
\tt LIBXDSMT.LIB & multithread for static linking \\
\tt XDS230I.LIB  & singlethread for dynamic linking \\ % \ref !!!
\tt XDS230MI.LIB & mulitthread for dynamic linking
\end{tabular}

A multithread environment imposes certain restrictions on the runtime library.
Many library procedures share data and other resources. The access to these
resources must be {\em serialized}, i.e. not more than one thread at a time
may use the resource, provided the resource is modified during its usage.

\begin{quotation}
{\footnotesize
Consider a process that has two threads, $A$ and $B$ and the thread $A$
dynamically allocates a variable using the procedure \verb'NEW'. Now, the
operating system schedules the thread $B$ for execution before the completion
of the procedure \verb'NEW' in the thread $A$, and the thread $B$ also
issues a call to \verb'NEW'. Because the thread $A$ might have left the
heap in an intermediate state, the second call of \verb'NEW' may fail or
corrupt the heap. Heap corruption may as well happen when the thread $A$
resumes execution.

In the multithread library, processing of the second call will be suspended
until completion of the first call, eliminating the possibility of heap
corruption.
} % \footnotesize
\end{quotation}

The multithread implementation of the \XDS{} runtime library uses
{\em semaphores} to protect resources accessed by a thread from
being corrupted by other threads. So operations involving input/output
channels, memory allocation and deallocation, etc. are serialized
by the runtime library.

The library procedures that do not use global data and do not access
resources are {\em reentrant}, i.e. may be safely called by more than one
thread at a time while not blocking any of the calling threads.
Note, however, that procedures with VAR parameters and procedures that
accept parameters that contain pointers are not truly reentrant.
If you access the same piece of memory from multiple threads simultaneously,
the results are unpredictable, so you should provide your own serialization.

\begin{quotation}
{\footnotesize
Let the thread $A$ call {\verb'Strings.Append("Word",s)'} to append \verb'"Word"'
to the string kept in the variable \verb's'. If the thread $A$ had not yet
put the trailing \verb'0C' when the thread $B$ was scheduled for execution,
the call to {\verb'Strings.Length(s)'} from the thread $B$ may return
incorrect result if the unused rest of \verb's' is not filled with zeroes.
A semaphore must have been used to protect \verb's' from being accessed
concurrently when the state of \verb's' is not known.
} % \footnotesize
\end{quotation}

The configuration file and the template file included in your \XDS{}
package allow you to select the version of the runtime library to
be linked with your program. Turn the {\bf MULTITHREAD} option ON
in the project file or in the command line to link with the multithread
library. By default, this option is set OFF and the singlethread
library is linked in.

\section{Creating and destroying threads}
\label{threads:create}

In the \See{multithread library}{}{threads:mtlib}, the ISO \mt{}
standard module \verb'Processes' is implemented over threads.
The procedure \verb'Processes.Create' (\verb'Processes.Start') not
just creates a thread but also performs actions required to ensure correct
processing of runtime library calls and exception handling in that
thread. If you use the multithread library, you {\em must}
start all threads that use the runtime
library and exception handling via one of these two procedures.

The procedure \verb'Processes.Create' (\verb'Processes.Start')
accepts the new thread's stack size in bytes through the \verb'extraSpace'
parameter. The \verb'procUrg' parameter is the initial priority of
that thread and is interpreted differently depending on the target
operating system:

{\bf Windows NT/95}

\begin{tabular}{rl}
\tt procUrg & \bf Priority value                   \\
\hline
         -3 & \tt THREAD\_PRIORITY\_IDLE           \\
         -2 & \tt THREAD\_PRIORITY\_LOWEST         \\
         -1 & \tt THREAD\_PRIORITY\_BELOW\_NORMAL  \\
          0 & \tt THREAD\_PRIORITY\_NORMAL         \\
          1 & \tt THREAD\_PRIORITY\_ABOVE\_NORMAL  \\
          2 & \tt THREAD\_PRIORITY\_HIGHEST        \\
          3 & \tt THREAD\_PRIORITY\_TIME\_CRITICAL
\end{tabular}

As can be seen, the \verb'procUrg' value of zero means regular priority
(the priority that the main thread has after it is started)
on all operating systems. If you do not need to alter
thread priority, you may safely pass zero to \verb'procUrg'
regardless of the target operating system.

The \verb'Processes.StopMe' procedure effectively terminates
the calling thread. % What about thread 1
Another way for a thread to terminate normally is to \verb'RETURN'
from the thread's body (the procedure passed as the \verb'procBody'
parameter to \verb'Processes.Create' or \verb'Processes.Start').

Threads that do not use the runtime library and exception handling
may be started using the target operating system API call:
\verb'OS2.DosCreateThread' or \verb'Windows.CreateThread'. In this
case, the procedure comprising the body of the thread must be
declared with the \verb'["SysCall"]' or the \verb'["StdCall"]'
\See{direct language specifier}{}{multilang:direct} respectively.
The \verb'Processes.StopMe' procedure should {\em not} be used
to terminate such a thread.

\ifcomment !!!
 The process termination functions abort, exit, and _exit end all threads within the process, not just the
 thread that calls the termination function.  In general, you should allow only thread 1 to terminate a process,
 and only after all other threads have ended.

 Note:

    1. If your program exits from a signal or exception handler it may be necessary to terminate the process from
       a thread other than thread 1.

    2. A routine that resides in a DLL must not terminate the process, except in the case of a critical error.  If
       the DLL and the executable for the process have different runtime libraries, terminating the process from
       the DLL would bypass any onexit or atexit functions that the executable may have registered.
\fi

\section{Operating system specifics}

The module \verb'Processes' provides a portable interface to the multithreading
facilities of the target operating system. Unfortunately, this modules, as
defined in ISO10514 is too general and restricted in comparison with the Win32
API. The \XDS{} library module \verb'Win32Processes' implement Windows-specific 
extensions to 
\verb'Processes', overviewed in the following subsections. Refer
to the respective definition modules for more information.

\subsection{Thread handles}

The procedure \verb'GetThreadHandle' returns the handle of the thread
associated with the given process id. This handle may be then used in all
thread-related API calls; the only exception is that a thread created using
\verb'Processes.Create' or \verb'Processes.Start' {\em must} be stopped
using \verb'Processes.StopMe'.

\subsection{Event sources}

The ISO \mt{} standard does not specify what is an event source, declaring
this feature implementation dependent. The procedure \verb'MakeSource'
can be used to create event sources from the undelaying operating system
resources, such as semaphores. These sources may be then used in calls to
\verb'Processes.Attach', \verb'Processes.Detach', \verb'Processes.IsAttached',
and \verb'Processes.Handler'.

See the correspondent definition module for more information.

\section{Volatile variables}
\label{threads:volatile}

The optimizations performed by the Native XDS-x86 compiler may
cause some variables to be temporarily placed on registers, and
to receive values different from those specified in the source
code and at different point of execution. These optimizations
should be disabled for variables that are referenced by more
than one thread.

If the \OERef{VOLATILE} option was switched ON during compilation
of a variable definition, the compiler assumes that references
to that variable may have side effects or that the value of
the variable may change in a way that can not be determined
at compile time. The optimizer does not eliminate any
operation involving that variable, and changes to the value of
the variable are stored immediately.

\begin{verbatim}
    <* PUSH *>
    <* VOLATILE+ *>
    VAR
      common: CARDINAL;
    <* POP *>
\end{verbatim}

\section{Exception handling}

\subsection{\mt{} exceptions}

All threads share a single instance of each exception source. As a result,
a thread will be blocked when an exception is raised in it if another thread
is handling an exception with the same source, until it leaves the exceptional
execution state.

\subsection{System exceptions}

A thread started via an API call is responsible for intercepting and handling
system exceptions, such as arithmetic overflow or access violation.

