\chapter{Introduction}
\pagenumbering{arabic}
\label{intro}
%\epsfbox{sbear.eps}

Sooner or later, every Modula-2 programmer encounters four problems.
These are: absence, incompleteness, unportability, and low quality of
libraries. At the same time, C/C++ programmers usually have problems
{\em choosing} from a huge set of free, public domain, shareware, and
commercial libraries of various purpose, size, and quality which are
in many cases portable or are available for a number of platforms.
Moreover, the {\em Application Programming Interfaces (APIs)} of the
most widely used software products (operating systems, database engines,
etc.), are defined in terms of the C programming language.

In order to use this resources galore from Modula-2, a programmer needs, first,
a Modula-2 compiler which supports C calling/naming conventions and a set of
types corresponding to C types, and, second, definition modules
corresponding to the C headers of the library/API. Finding a suitable compiler
is not a very big deal, but manual conversion of C headers turns to a real
nightmare when it comes to, say, the X Window API. That is why we created H2D.

H2D does the job automatically, i.e. translates C header files into Modula-2
definition modules. H2D is intended to be used with
\See{XDS}{Appendix }{XDS}
version 2.10 or later and is included in the XDS distribution package.
However, the generated definition modules may be used with any ISO-compliant
Modula-2 compiler. The required modifications are minor and may be done
using text editor macros or a simple {\bf REXX}, {\bf sed}, etc script.

The source language is a subset of ANSI C, which includes declarations and
preprocessor directives, with some extensions (See \ref{rules:qualifiers} and
Chapter \ref{project}).
Destination language is {\bf ISO Modula-2} with some XDS language extensions.
XDS allows to use the resulting definition modules with {\em both}
Modula-2 and Oberon-2.

H2D generates definition modules suitable for either XDS-C, Native XDS, or
both. In case of Native XDS, module template for function-like C macros
may be generated (See \ref{using:fit:native}). In case of XDS-C, an extra
header file containing C declarations of types introduced by H2D is generated
(See \ref{using:fit:c}).

\section{New in version 1.30}

Major improvements in v1.30:

\begin{itemize}
\item Generalized \PPDir{variant}
\item \See{Custom mapping of C base types to Modula-2 types}{}{using:modrules:mapping}
\item \See{Non-standard directives extraction}{}{rules:nonstandard}
\item \See{Options renamed to follow XDS compilers style}{Chapter }{options}
\item Control file syntax now closely matches used by XDS compilers
      (see Chapters \ref{config} and \ref{project})
\end{itemize}

\section{Typographic conventions}
\label{intro:conv}

\subsection{Language descriptions}
\label{intro:conv:EBNF}

Where formal descriptions for language syntax constructions appear, an
{\em extended Backus-Naur Formalism (EBNF)} is used.

These descriptions are set in a monospaced font.

\begin{verbatim}
Text = Text [{Text}] | Text.
\end{verbatim}

In EBNF, brackets \verb+[+ and \verb+]+ denote optionality of the
enclosed expression, braces \verb+{+ and \verb+}+ denote repetition
(possibly 0 times), and the line \verb+|+ denotes other possible valid
descriptions.

Non-terminal symbols start with an upper case letter (e.g. \verb+Statement+).
Terminal symbols either start with a lower case letter (e.g. \verb+ident+), or are
written in all upper case letters (e.g. \verb+BEGIN+), or are enclosed within
quotation marks (e.g. \verb+":="+).

\subsection{Source code fragments}
\label{intro:conv:cource}

When fragments of a source code are used for examples or appear within
a text they are set in a monospaced font.

\begin{verbatim}
/* example.h */

typedef unsigned long int UINT;
\end{verbatim}

