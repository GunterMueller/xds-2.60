\chapter{Checking your program for errors}
\label{osa}

The amount of resources required to debug and thoroughly test a program
grows exponentially as the size of the program grows.
To help you reduce it, XDS includes a error checker, which is based 
on the XDS compiler construction framework and uses powerful data flow 
analysis algorithms to detect potential run-time errors {\em statically}, 
i.e. without program execution.

{\bf Note:} In the current version of this chapter it is assumed that the host 
platform is Windows.


\section{Overview}
\label{osa:overview}

The static error checker OSA checks \ot{} and \mt{} programs for run-time 
errors by analysing the source code. The checker is able to detect
all kinds of violations of \mt{} and \ot{} dynamic semantics,
which are typically found (or not found) during debugging and testing stages
of development.

OSA also checks the analysed program for violations of well organized 
program criteria. Of course, violations of these criteria are not errors,
but they often indicate other program weaknesses that may be errors. 

Some (but not all) of the erroneous situation types in a program that OSA
is able to detect are listed below:
\begin{itemize}
\item usage of an uninitialized variable 
\item dereference of a pointer whose value is \verb'NIL' in
      "\verb'<designator>^'" constructs
\item assignment of a variable whose value is never used
\item type guard violation
\item usage of a dead variable (a variable that was already deallocated)
\item unreachable branches in statements \verb'IF', \verb'CASE', and \ot{} 
      \verb'WITH'
\item always true (or always false) operands in constructs like
      "\verb'<Expr1> OR <Expr2>'" and "\verb'<Expr1> AND <Expr2>'"
\item procedure without exit
\item \verb'ASSERT' condition is always false
\item overflow (underflow) in arithmetic operators and type
      transfers
\item array index is out of bounds
\item division by zero
\item \verb'FOR' loop repeats zero times
\item \verb'ALLOCATE' or \verb'NEW' is used to create an array whose length is 
      less than or equal to zero
\end{itemize}

Actually, this is a typical list of erroneous situations that an ordinary
error checker, e.g. for the C language, can detect. But OSA has two
important advantages:
\begin{enumerate}
\item OSA distinguishes between a definite error, a warning, and a
      doubtful situation. In the latter case, OSA produces a message called 
      "conditional error" that in practice appear to indicate an error
      in approximately one case from five.

\item OSA performs context-sensitive data flow analysis, so it is able to
      recognize an error that shows itself for some calls of the 
      procedure containing it, but does not for other calls. 
      Thus, OSA handles situations of high complexity.
\end{enumerate}

Data-flow analysis performed by OSA requires a considerable amount of
time and memory. If your program causes OSA to exhaust the resources 
of your computer, you may wish to read the section \ref{osa:speedup}, which
describes how to speedup analysis in exchange for its quality.

OSA was originally developed at the Institute of Informatics Systems,
Novosibirsk, Russia, by Vladimir I. Shelekhov (\verb'vshel@iis.nsk.su') and
Sergey V. Kuksenko (\verb'sergic@iis.nsk.su').

\section{The OSA environment}
\label{osa:env}

OSA is based on the XDS compiler construction framework,
and inherits the user interface of XDS compilers.
XDS compiler configuration, control files, operation modes,
options and equations, etc., are fully applicable to OSA.
Refer to chapters \ref{config}, \ref{usage}, and \ref{options}
for more information.

Similarly to an XDS compiler environment, the OSA environment consists of
the executable file, {\tt osa\dotExe{}}, and a set of system files: 
\verb'osa.red', \verb'osa.cfg', and \verb'osa.msg'. The \verb'showmsg.exe' 
utility may be used to transmit OSA messages into XDS environment 
(Windows only).

You may wish to create subdirectories \verb'msg', \verb'sym', \verb'irf', 
and \verb'lst' for analyser output files in your working directory. 
You may continue to translate programs that reside in that directory 
with an XDS compiler.

A \See{redirection file}{}{xc:red}, \verb'osa.red', defines directories
in which OSA seeks and creates files of variuos types. The most important
are your program source files and XDS library files. \verb'osa.red' must 
reside in the directory that is the current directory when you start OSA.
If there is no redirection file in the current directory,
the {\it master redirection file} is read from
the directory in which the OSA executable resides.

An OSA redirection file example is shown below:

\begin{verbatim}
*.msg  = c:\xds\bin
*.cfg  = c:\xds\bin
*.prj  = prj 
*.sym  = sym; c:\xds\sym\x86
*.def  = def; c:\xds\def\iso; c:\xds\def\ts
*.mod  = mod
*.ob2  = oberon; appl
*.mes  = msg
*.lst  = lst
*.grf  = lst
*.irf  = irf
\end{verbatim}

According to this file, OSA will search for \ot{} modules 
in the subdirectories \verb'oberon' and \verb'appl' of the current directory, 
for \mt{} definition modules --- in the \verb'def' subdirectory 
and in XDS system directories, \verb'C:\XDS\DEF\ISO' and \verb'C:\XDS\DEF\TS',
and so on. The OSA message listing for a program will be written to the
file \verb'msg\<main module name>.mes'.

The analyser may be set up so that it will save the partially analysed
intermediate representation of the program to file and then will
terminate (see \ref{osa:speedup:save}). The default extension for these 
files is \verb'irf', and this redirection file forces OSA to place
them into the subdirectory \verb'irf'.

The \See{procedure call graph}{}{osa:graph} of a program is written to 
the \verb'lst' subdirectory if the option \OERef{GRAPH} is ON.

\subsection{Symbol files}
\label{osa:env:sym}

Symbol files for program modules are created and/or read when a program is
translated by the XDS \mt{}/\ot{} front-end, which is part of OSA.
As a result, the analyzer may read symbol files that were created by an
XDS compiler\footnote{Provided that the compiler is built in the same mode
(normal, trial, or demo) and uses the same version of the front-end.}.

The equation \OERef{SYM} (sybmol file extension)
may be used to help you better distinguish between OSA symbol 
files and your XDS compiler symbol files. For instance, to
have the extension \verb'osm' to be used for OSA symbol files,
add the line

\verb'    -sym=osm'

to your \verb'osa.cfg' file. Now, to have these files placed
into the \verb'osa_sym' subdirectory, add  the line

\verb'    *.osm = osa_sym;' 

to your \verb'osa.red' file.


\section{Starting OSA}
\label{osa:start}

Generally, OSA is invoked like an XDS compiler. All XDS
\See{operation modes}{}{xc:modes}, except GEN, BROWSE, and BATCH,
may be used. The \Ref{ALL}{xc:modes:all} submode is always assumed.
The error checker recognizes all XDS options and equations that are
not specific to a particular code generator.

{\bf Note:} The option \OERef{MAIN} has to be set ON (inline or in the 
command line) for the main \ot{} module.

If your program consists of a single module, OSA can be invoked 
in the \Ref{COMPILE}{xc:modes:compile} mode:
\index{osa\dotExe{}@{osa\dotExe{}}}

\verb'    osa [=c] <source file name> [<options and equations>]'

If your program includes several modules but has no
project file, use the \Ref{MAKE}{xc:modes:make} mode:

\verb'    osa =m <top-level module file> [<options and equations>]'

Finally, if your program includes several modules and has a
project file, use the \Ref{PROJECT}{xc:modes:project} mode:

\verb'    osa =p <project file name> [<options and equations>]'

A project file example is presented below:

\begin{verbatim}
-graph+         % OSA option
-o2extensions+  % Modula-2/Oberon-2 front-end options
-m2extensions+
-m2addtypes+
!module Start.mod
\end{verbatim}

Given the main module of a program, OSA searches for other
modules according to XDS make strategy, using paths specified
in the redirection file and \OERef{LOOKUP} equations. 
If the source text of a module cannot be found using search
paths, then at least the correspondent symbol file 
must be present. To produce a symbol file for a module, 
use the command line like

\verb'    osa <module name> +msym'


\section{Analysed configuration}
\label{osa:anconf}

Upon startup, the analyser builds the analysed configuration of the 
specified program, using the redirection file and \OERef{LOOKUP} equations.
An analysed configuration may be the whole program or some part of it
and includes procedures (and module bodies) called from the main module 
or from top procedures (see section \ref{osa:speedup:part}). 
If the source code of a module is not available to OSA, then procedures
of this module are called {\em external procedures} and variables
exported from this module are called {\em external variables}.

{\bf Note:} The term "external procedure" shall not be mixed with
the XDS language extension that allows to declare procedures as external
(see \ref{multilang:extproc}).

An analysed configuration should have the least possible number of external
entities, because data flow analysis of external entities has to be
very rough to be true.

{\bf Note:} In \ot{} programs, if the analysed configuration includes a 
call of a method for the object \verb'Obj', then all procedures, 
bodies of which contain calls include calls \verb'NEW(Obj)' or 
assignment statements that assign values to \verb'Obj', must be 
also included in the analysed configuration. 
Otherwise data flow analysis results will be incorrect.


\section{Analysis phases and stages}
\label{osa:phases}

The analyser processes your program in several passes: 

\begin{enumerate}
\item Creation of the XDS intermediate representation
      of your program by the XDS \mt{}/\ot{} front-end.
\item Conversion of the XDS intermediate representation to the OSA 
      intermediate representation.
\item Data flow analysis
\item Error analysis
\end{enumerate}

In turn, data flow analysis in OSA consists of the following phases:

\begin{tabular}{ll}
\bf CI  &  context-insensitive analysis \\
\bf CIF &  CI phase finalization        \\
\bf CS  &  context-sensitive analysis   \\
\bf CSF &  CS phase finalization        \\
\bf VC  &  value calculation
\end{tabular}

Each phase consists of a sequence of {\em stages}. 
A stage is a minimal independent step of analysis.
Stages are sequentially numbered through all phases.

You may explicitly terminate an analysis phase at a specific stage
by substituting the number of that stage into the 
\OERef{CUT} equation, or interactively by
pressing the {\bf Esc} key (see section \ref{osa:speedup}). 
This will cut off parts of the program that were not analysed yet
for the remaining phases.

Finalization phases were introduced for correct cleanup after 
an explicit cutoff of an analysis phase.
If the analysis phase was not cut off, the respective finalization
phase is not invoked. So it is recommended not to cut off
a finalization phase.  

Sometimes, if the CI phase was failed due to lack of resources,
the option \OERef{CI} can be of help. If this option is set OFF,
data flow analysis is started with the CS phase.

If the VC option is set OFF, the VC phase will be omitted.
This feature may be used, for instance, if OSA crashes during this
phase.

The Error analysis part of OSA consists of the following phases:

\begin{tabular}{ll}
\bf EP  &  preliminary analysis         \\
\bf EU  &  analysis of variable usages  \\
\bf EM  &  message printing             \\
\end{tabular}

There are the following modes of the EU phase:
\begin{itemize}
\item FULL  -  full analysis (by default)
\item ROUGH -  simplified analysis that is less precise but needs less 
               resources 
\item NONE  -  EU phase is omitted 
\end{itemize}

Desirable mode of the EU phase may be chosen by the equation \OERef{EU}. 


\section{Procedure call graph output}
\label{osa:graph}

If the option \OERef{GRAPH} is set ON, the analyser outputs the procedure
call graph when data flow analysis finishes or is terminated using
one of the equations \OERef{CUT}, \OERef{SAVE}, or \OERef{SAVE\_MEM},
or intercatively.

If the program being passed through OSA is large or complicated, 
data flow analysis may slow down drastically at some stage 
due to lack of resources. The procedure call graph may help you 
understand which procedures cause the difficulties.

There are two sections in a procedure call graph file.
The first part shows, for each procedure, which procedures are 
called in its body. The second part shows, for each procedure, 
from what procedures it may be called.

Each procedure belongs to a certain {\em layer}. 
Procedures in the layer 0 call no other procedures.
A procedure $P$ that belong to the layer $N$, may call 
a procedure belonging to a layer whose number is less than $N$ 
(or is equal to $N$ if the procedure $P$ is recursive).

A procedure layer in the first section of a procedure call graph
has the following structure: 

\begin{verbatim}
 ----------------------- LAYER N ----------------------
 <list of procedure names>
 <calls in the body of the first procedure>
 ............................................ and so on
 <calls in the body of the last procedure>
\end{verbatim}

In the \verb'<list of procedure names>' the name of a procedure that
is already processed by OSA is followed by two numbers: the number of
contexts and the number of variables in the collected data flow information
for this procedure. The second number strongly correlates
with the amount of memory required for analysis of the procedure.
If the number of contexts is in the range 200-400 or above (this may happen 
for a recursive procedure) then, as a rule, the analysis of such 
procedure is slow. The equation \OERef{ONE} may be used to speedup
analysis of such procedures (see \ref{osa:speedup}).

In the begining of the first section, the external procedures 
(see \ref{osa:anconf}) are showed. The OSA analysis will be drastically
slow and inprecise if the second number in brackets is high for some
external procedure.

In the second section of a procedure call graph each line 
has the following structure:
 
\verb'    <procedure name> <-- <list of calling procedures>'

Each call in the \verb'<list of calling procedures>' is 

\verb'    <procedure name> ( <line number>, <column number> )'

A fragment of a procedure call graph is shown below.

\begin{verbatim}
 ----------------------- LAYER 7 ----------------------
 slP.openOut(15,121) slP.boolExpr(4,455)! 
     slP.openOut-->slP.warning,ManagerDesc:xiFi.Open;  
     slP.boolExpr-->slP.Value!,slK.push,slK.fatal!;  
\end{verbatim}

It shows which procedures are called from procedures \verb'slP.openOut' 
and \verb'slP.boolExpr', which belong to layer 7 of the call graph. 
The number of contexts for \verb'slP.openOut' is 15; the number of 
variable definitions (assignments) is 121. The respective values for 
\verb'slP.boolExpr' are 4 and 455.

%!!! Continue explanations

The token \verb'"!"' indicates procedures whose analysis has not been 
finished yet. In the above fragment, procedures \verb'slP.Value', 
\verb'slK.fatal', and, as a result, \verb'slP.boolExpr' are still
processed by data flow analysis. This may give you some idea on
the progress of data flow analysis.

In the second part of a procedure call graph file, calls to 
each procedure are shown. For instance, if you want to find out
from where the procedure \verb'Value' is called, search for the string
"\verb'Value<--'" in the call graph file. 
You might find a string like:

\verb'    slP.Value<--slP.boolExpr(351,14), slP.Cbra(341,4), slH.Binar() ' 

which indicates that procedures \verb'slP.boolExpr', \verb'slP.Cbra',
and \verb'slH.Binar' have calls to \verb'Value' in their bodies,
but only procedures \verb'slP.boolExpr' and \verb'slP.Cbra' were processed 
by data flow analysis, because coordinates of calls are shown in brackets
only for these two procedures. The procedure \verb'slH.Binar' did not  
undergo data flow analysis yet.


\section{How to speedup OSA analysis}
\label{osa:speedup}

If data flow analysis exhausts the resources of your system,
or takes too much time, you may use one or more of
the following techniques:

\begin{itemize}
\item Use the equation \OERef{SAVE} to have OSA save the partially
      analysed program to file after the given analysis stage
      is finished. Then, invoke OSA again specifying the equation
      \OERef{READ} to continue analysis (see section \ref{osa:speedup:save}).
\item Any analysis phase may be explicitly terminated after the
      given stage is finished using the equation \OERef{CUT}.
      The remaining phases will work only on the program part
      that has been processed by the terminated phase (see section \ref{osa:speedup:cut}). 
\item A well closed part of a program (based on one or a few procedures) 
      may be independently passed through OSA (see section \ref{osa:speedup:part}).
\item Use the equation \OERef{ONE} to analyse specific
      procedures context-insensitive mode only (see section \ref{osa:speedup:nocsa}).
\end{itemize}

To have the names of procedures processed in data flow analysis
displayed on the screen, use the option \OERef{SHOW}.

These features may be activated interactively by pressing the {\bf Esc}
key during OSA execution (Windows only) when analysis phases
are working. If you press {\bf Esc} when data flow analysis is working, 
then after the end of the current analysis stage the following menu
will be written to the screen:

\begin{verbatim}
ANALYSIS INTERRUPTED!
(S) - save intermediate file and terminate
(C) - cut off non-analysed part of the program and skip to next phase
(G) - output procedure call graph to file
(H) - toggle progress indicator
(N) - continue analysis
(F) - terminate immediately
\end{verbatim}

If you type "g", OSA outputs the current procedure call graph to
a file (see \ref{osa:graph}), and then redisplays the menu. 
You may now look at the call call graph in another window and 
then make your choice.

If you type "h", OSA triggers the option \OERef{SHOW}, and then
redisplays the menu. To see the names of procedures processed by
data flow analysis, switch this option ON.

OSA is written in the \ot{} language, which integral memory management relies 
on garbage collection. However, the garbage collector is off by default,
so that it does not slow down the analyser. If you notice increased
swapping activity during analysis, terminate OSA and turn on garbage 
collection by specifying the equations \OERef{COLLECTMIN} and 
\OERef{COLLECTMAX} in the OSA configuration file, \verb'osa.cfg':

\verb'    -CollectMin = 110' \\
\verb'    -CollectMax = 127' 

In this example, garbage collection will be invoked after the current analysis 
stage is finished, if the amount of memory allocated by the analyser is
in the range 110-127 megabytes. This range is close to optimal for 
a system with 128M RAM running under OS/2 Warp version 3. 
We recommend to put this range close to the amount of available physical memory
before you start OSA, but you'll have to experiment until you find suitable values
for your system.


\subsection{Analysing part of a program}
\label{osa:speedup:part}

Part of a program, defined by one or more procedures, 
may be independently analysed by using the equation \OERef{TOP}.

Type 

\verb'    osa =p <project file name> -top=<procedure list>, '

where 
{\samepage
\begin{verbatim}
<procedure list> ::= 
          <procedure> |
          <procedure list>;<procedure>
<procedure> ::= 
          <ordinary procedure> |
          <type-bound procedure>
<ordinary procedure> ::= 
          <module name>.<procedure name>
<type-bound procedure> ::= 
          <module name>.<type name>::<procedure name>
\end{verbatim}
} % \samepage

Note that blanks and commas are not permitted inside a \verb'<procedure list>'.
For \ot{} type-bound procedures the name of the respective pointer type 
or the name of the record type may be used.

When the equation \OERef{TOP} is specified, the analysed configuration
is built as if the top-level module of the program contained only calls
to the given procedures, with unknown values of parameters, so
it consists of these procedures and all procedures that may be 
reached from them during execution.

{\bf Note:} It is a user's responsibility to ensure that the
resulting analysed configuration is {\em well closed}.
This means that variables declared in imported modules are 
not changed by modules that are not included in the analysed configuration. 


\subsection{Saving a partially analysed program to file}
\label{osa:speedup:save}

A partially analysed intermediate representation of a program may be
saved to file after any stage using the \OERef{SAVE} equation.
To continue analysis, the \OERef{READ} equation shall be used when
the analyser is invoked next time.

For instance, if the main program module is named \verb'Main', and 
the project name is \verb'MainProj', the command line

\verb'    osa =p MainProj -save=456 '

would save the intermediate representation to file \verb'Main_456.irf'
after completion of the analysis stage number 456. To continue analysis, 
type 

\verb'    osa -read=Main_456'

The amount of memory required to load a previously saved partially 
analysed program is typically much less than was allocated before 
it was saved. So this feature may be used to decrease OSA memory 
requirements.

There are three special values of the \OERef{SAVE} equation.
Specifying \verb'-save=last' causes the analysed program to be saved after
completion of data flow analysis. \verb'-save=first', caused it to be saved
right before data flow analysis. \verb'-save=check', caused it to be saved
after completion of the EU phase of the Error analysis 
(see section \ref{osa:phases}). 

The equation \OERef{SAVE\_MEM} may be used to save a partially analysed 
program to file if the amount of allocated memory exceeds the given
value (in megabytes) after completion of the current analysis stage.


\subsection{Excluding program parts from subsequent phases}
\label{osa:speedup:cut}

An analysis phase may be explicitly terminated using the equation 
\OERef{CUT} or interactively by pressing the {\bf Esc} key.
The current phase will be terminated after completion of the
specified stage. The part of the program that has not been
processed during this stage is cut off; the subsequent phases 
will process only the remaining part.

There is one special value of the \OERef{CUT} equation. If you type 

\verb'    osa -read=<file> -cut=now' 

then after reading the partially program from "\verb'file'", 
the current phase will be terminated, and the next analysis 
phase will be started.


\subsection{Simplified analysis of procedures}
\label{osa:speedup:nocsa}

It may happen that data flow analysis works a very long time
at some stage. A possible cause is the constantly growing number of contexts
for some (usually recursive) procedures. This may be confirmed
by studying the procedure call graph after the current analysis stage
(see section \ref{osa:graph}).

If the above appears to be the case, the equation \OERef{ONE} may be used 
in the next invocation of OSA to specify the list of procedures
that have to be analysed context-insensitively. 
The structure of a procedure list is described in section \ref{osa:speedup:part}.
Analysis will be faster but rough for listed procedures.

To have the names of procedures processed in data flow analysis
displayed on the screen, use the option \OERef{SHOW}.


\subsection{Turning off context-insensitive analysis}
\label{osa:speedup:nocia}

If the option \OERef{CI} option is set OFF, the CI analysis phase 
will be skipped and data flow analysis will be started from the CS phase
(see \ref{osa:phases}).
This may help if the CI phase fails due to lack of resources.


\subsection{Turning off analysis of variable usage}
\label{osa:speedup:noeua}

Setting the equation \OERef{EU} to \verb'ROUGH' causes the analysis of 
variable usage to be performed in simplified mode using less resourses.
Setting it to \verb'NONE' completely disables the EU phase of the 
Error analysis. See section \ref{osa:phases} for more information.

This may help if the Error analysis fails due to lack of memory.


\section{OSA messages}
\label{osa:msg}

{\samepage
Let's run the example program \verb'weak2' (below) through OSA by typing
the following string: 

\verb'    osa weak2.ob2'

This will produce a message file \verb'weak2.mes' in the approriate 
directory.

\begin{verbatim}
1  MODULE weak2;
2    IMPORT M;
3    TYPE p_rec= POINTER TO RECORD field: INTEGER END;
4    VAR X,Y:   p_rec;
5    VAR i,j,k: INTEGER;
6    PROCEDURE F(Z:p_rec);
7    BEGIN  Z^.field:=k+1  END F;
8  BEGIN
9    NEW(X); NEW(Y);
10   M.G(i,i+1);
11   IF i=0 THEN F(X); M.G(j,j+1) ELSIF i>0 THEN F(Y) END;
12 END weak2.
\end{verbatim}
} %\samepage


\subsection{Message modes}
\label{osa:msg:modes}

A message listing produced by OSA includes error messages,
conditional error messages and warnings. An
error message corresponds to an absolute error; 
the error site in source code is reached for
every execution of a program. An error site for a conditional
error is nested inside a control statement that may alter execution flow,
such as \verb'IF', \verb'CASE', or \verb'WHILE'.
A conditional error turns to the actual error if error site is
reached for any execution of program. This depends on the values of
conditions in the \verb'IF', \verb'CASE', etc. statements that lead to 
error site.

The resulting OSA message listing includes the following messages: 

{\footnotesize
\begin{verbatim}
################### Unassigned Variables (ABSOLUTE ERRORS) #################

[E] (t/weak2.ob2 10,9) variable i not assigned
      M.G(i,i$+1);

################# Unassigned Variables (CONDITIONAL ERRORS) ################

[C] (t/weak2.ob2 7,20) variable k not assigned
      BEGIN  Z^.field:=k$+1  END F;
[C] (t/weak2.ob2 11,27) variable j not assigned
      IF i=0 THEN F(X); M.G(j,j$+1) ELSIF i>0 THEN F(Y) END;
\end{verbatim}
} % \footnotesize

The first message reports an absolute error because the
expression \verb'i=0' is always calculated when the program is executed.
The third message indicates a conditional error, because the condition 
\verb'j>0' is inside an \verb'IF' statement so it is possible that it will not
be calculated during program execution (if the value of \verb'i' is zero).
The second message is also conditional. Even though the statement 
\verb'Z^.field:=k+1' in the body of the procedure \verb'F' is not 
nested inside an \verb'IF', \verb'CASE', or another control statement,
both calls to \verb'F' are nested inside an \verb'IF' statement. 
So this is an actual error only if \verb'(i=0) OR (j>0)' is true
during execution of \verb'weak2'. 


\subsection{Variables}

The message listing for the above example program \verb'weak2' also
includes the following message: 

{\footnotesize
\begin{verbatim}
######################## UNUSABLE VARIABLES (Strong) #######################

[W] (t/weak2.ob2 7,17) variables $new0.field,$new1.field assigned but not
   used
      BEGIN  Z^.field$:=k+1  END F;
  #DEREFERENCES Z^=$new0,$new1!
  #DECLARATIONS $new0:(9,2) $new1:(9,10) Z:(6,14)
\end{verbatim}
} % \footnotesize

This warning reports that the values of variables \verb'$new0.field' and
\verb'$new1.field' are not used in program \verb'weak2', and so
the assignment \verb'Z^.field :=  k+1' to these variables is superfluous.

Variables \verb'$new0' and \verb'$new1' denote heap variables generated after
execution of statements \verb'NEW(X)' and \verb'NEW(Y)', respectively. 
That is, \verb'X^=$new0' and \verb'Y^=$new1'. 

The section \verb'#DEREFERENCES' in the message shows the values of
dereferenced variables which constitute target variable of the
assignment statement. For the target variable \verb'Z^.field', variables
\verb'$new0' and \verb'$new1' are showed as the values of variable \verb'Z'. 

The symbol "\verb'!'", located at the end of the \verb'#DEREFERENCES' section, 
indicates that both variables \verb'$new0' and \verb'$new1' were recognized 
as obligatory (must-) values of \verb'Z'.
The symbol "\verb'?'" in that position would mean that 
variables after the symbol "\verb'?'" were recognized 
as possible (may-) values of  \verb'Z'.

In the \verb'#DECLARATIONS' section, coordinates (line and column numbers)
of the variable declaration in the source code are shown for each variable
from the \verb'#DEREFERENCES' section.


\subsection{Contexts}

A {\em context} of a procedure \verb'F' is a subset of the full set of
call chains to the procedure \verb'F' from the body of the main program 
module or from \See{top procedures}{}{osa:speedup:part}.
Let us illustrate the notion of context 
on the example program \verb'weak3' below. 

\begin{verbatim}
1  MODULE weak3;
2    TYPE p_rec= POINTER TO RECORD field: INTEGER END;
3    VAR X,Y,U:   p_rec;
4    VAR i,k:     INTEGER; 
5 
6    PROCEDURE F(Z:p_rec);
7    BEGIN  Z^.field:=k+1  END F;
8
9    PROCEDURE G(R:p_rec);
10   BEGIN  IF k#0 THEN F(R) END END G;
11 BEGIN
12   NEW(X); NEW(Y);
13   IF i=0 THEN F(X) ELSE G(Y) END;
14   G(U);
15 END weak3.
\end{verbatim}

The full set of call chains to the procedure \verb'F' (that is the
full context) is the following: 

\verb'     $body_weak3(13,14)' $\rightarrow$ \verb'F' \\
\verb'     $body_weak3(13,24)' $\rightarrow$ \verb'G(10,21)' $\rightarrow$ \verb'F' \\
\verb'     $body_weak3(14,2) ' $\rightarrow$ \verb'G(10,21)' $\rightarrow$ \verb'F'

Each OSA message holds only within the context given in the \verb'#CONTEXT' 
section of the message. If the \verb'#CONTEXT' section is
not present, the message holds for all possible call chains.

A context containing a single call chain 

\verb'    $body_weak3(14,2) -> G(10,21) -> F' 

is used in the following message: 

{\footnotesize
\begin{verbatim}
######## NIL pointer dereference & type guard (CONDITIONAL ERRORS) #########

[C] (t/weak3.ob2 7,10) NIL pointer dereference for variable Z
      BEGIN  Z$^.field:=k+1  END F;
  #CONTEXT F(7,10) <- G(10,21?) <- $body_weak3(14,2)
\end{verbatim}
} % \footnotesize

It means that this message is valid when the procedure \verb'F' is
called from the procedure \verb'G' at position (10,21), 
and \verb'G', in turn, is called from the body of the module \verb'weak3'
at position \verb'(14,2)'.

The symbol "\verb'?'" in \verb'G(10,21?)' means that the indicated call of 
\verb'F' in the body of \verb'G' is under condition. 

A context including two call chains is defined in the message below. 

{\footnotesize
\begin{verbatim}
[W] (t/weak3.ob2 7,17) variable $new0.field,$new1.field assigned but not used
      BEGIN  Z^.field$:=k+1  END F;
  #DEREFERENCES Z^=$new0,$new1!
  #DECLARATIONS $new0:(12,2) $new1:(12,10) Z:(6,14) 
  #CONTEXT F(7,17) <- G(10,21?) <- $body_weak3(13,24?)
      # F(7,17) <- $body_weak3(13,14?)
\end{verbatim}
} % \footnotesize

In general case, a context is described by a tree-like chain of calls.
Each new branch of the tree begins with the symbol '\verb'#'" followed by a
procedure call entered before. 


\subsection{Variable denotations}
\label{osa:msg:var}

The variable name modes used in OSA message list are described
below.

\begin{verbatim}
<variable name> ::= 
          <variable identifier> |
          <heap variable> |
          <variable name>.<field name> |
          <variable name>[] |
          <previous instance of variable> |
          <external variable denotation> |
          ~<variable name> |
          <temporary variable>
\end{verbatim}

If \verb'A' is a name of an array variable. \verb'A[]' denotes an arbitrary
element or the array \verb'A'. \verb'~A' denotes a local variable \verb'A'
after return from from the procedure containing it, when the address of
\verb'A' is the value of a pointer variable which is global
relative to that procedure. 

\begin{verbatim}
<heap variable> ::= 
          $new<number of NEW or ALLOCATE statement>
<number of NEW or ALLOCATE statement> ::=  <integer> 
<previous instance of variable> ::= 
          <variable identifier>' |
          <heap variable>' 
\end{verbatim}

Name \verb'$newK' denotes a variable generated by the \verb'K'-th occurrence of
a call to the \verb'ALLOCATE' procedure (when the option \OERef{ALLOCATE} 
is ON) or to the \verb'NEW' procedure in source code. 
Any previous instance of a variable generated before by the same 
\verb'NEW' or \verb'ALLOCATE' statement is denoted by \verb"$newK'". 

In the example program below, after execution of the statement
\verb'NEW(n)', the variable \verb'n' points to a heap variable \verb'$new0' 
\verb'(n^=$new0)'. If the variable \verb'list' is not equal to \verb'NIL'
on line 10 (that is when \verb'i' is greater than one), the variable 
\verb'list' points to \verb'$new0' generated in the previous call of
the procedure \verb'to_list', and hence \verb"list^=$new0'". 

\begin{verbatim}
1  MODULE previous;
2  TYPE pList = POINTER TO List;
3       List = RECORD elem:INTEGER; next:pList END;
4  VAR list: pList; i: INTEGER;
5
6  PROCEDURE to_list(elem:INTEGER);
7    VAR n: pList;
8  BEGIN
9    NEW(n); n.elem:=elem;
10   IF list#NIL THEN n.next:=list END;
11   list:=n;
12 END to_list;
13
14 BEGIN FOR i:=1 TO 8 DO to_list(i) END
15 END previous.
\end{verbatim}

If \verb'z' be a variable identifier declared in a recursive procedure
\verb'F', then \verb'z' denotes that variable for the current call
of \verb'F' and \verb"z'" denotes the same variable in any previous 
recursive call.

\begin{verbatim}
<external variable denotation> ::= #<type name>
\end{verbatim}

A variable declared beyond the analysed configuration of a program
is called an external variable. A pointer variable that receives its value as
a result of execution of an external procedure (a procedure whose body is 
not included in the analysed configuration), is assumed to point to an
an external variable. Such variables are denoted by \verb'#T'
where \verb'T' is the name of the pointer base type.

\begin{verbatim}
<temporary variable> ::= 
          $temp_<number of WITH statement> |
          $agr_<number of array or record constructor> 
\end{verbatim}

A temporary variable is created by data flow analyser for each
\mt{} \verb'WITH' statement, and for each occurrence of
an array or record constructor in the source code. 


\section{OSA options}
\label{osa:options}

All XDS compiler options not specific to a particular back-end
may be used with OSA. See Chapter \ref{options} for more information.

\begin{table}[htbp]
\begin{center}
\begin{tabular}{|l|c|l|}
\hline
\bf Option  & \bf Default & \bf Meaning \\
\hline
\OERef{GRAPH}        & OFF & enable procedure call graph output \\
\OERef{CLOSED}       & ON  & assume well closed configuration \\
\OERef{ALLOCATE}     & ON  & assume standard heap manager \\
\OERef{SHOW}         & OFF & print names of analysed procedures \\
\OERef{MSYM}         & OFF & only make symbol files \\
\OERef{FIELD\_ALIAS} & OFF & assume variant field aliasing \\
\OERef{CI}           & ON  & enable context-insensitive phase \\
\OERef{CV}           & ON  & enable value calculation phase \\
\hline
\end{tabular}
\end{center}
\caption{OSA options}
\label{table:opt:osa}
\index{options!OSA}
\end{table}

\ifonline \else
\begin{description}
\fi
\OptHead{GRAPH}{enable procedure call graph output}

        Setting this option ON enables procedure call graph output
        after end of the data flow analysis. See section \ref{osa:graph}
        for more information.

        This option is OFF by default.

\OptHead{CLOSED}{assume well closed configuration}
 
        If this option is set ON, OSA assumes that the part of the 
        program being analysed is well closed for used \mt{} modules.

        This option is ON by default.

        See section \ref{osa:speedup:part} for more information.

\OptHead{ALLOCATE}{assume standard heap manager}

        If this options is set ON, the analyser assumes that the
        procedure \verb'ALLOCATE' used in the program is declared in
        the standard module \verb'Storage'. This may significantly improve 
        analysis results.

        This option is ON by default.

\OptHead{SHOW}{print names of analysed procedures}

        Setting this option ON enables output of the names of procedures
        processed by data flow analysis to the screen.

        This option is OFF by default.

        See sections \ref{osa:speedup:cut} and \ref{osa:speedup}.

\OptHead{MSYM}{only make symbol files} 

        This option, when set ON, causes OSA to terminate immediately 
        after creation of the XDS internal representation of your program.
        Use it if you only need to create a symbol
        file for some module.

        This option is OFF by default.

        See section \ref{osa:start}. 

\OptHead{FIELD\_ALIAS}{assume variant field aliasing} \inline

        If this option is set OFF, the analyser treats fields of
        variant parts of \mt{} record types as if they were non-overlapping. 
        This does not affect analysis correctness, provided that in no event
        the program accesses a variant field different from what was assigned 
        last, and the quality of analysis is better.
        Set this option ON inline for types that do not fall into the above 
        category.

        This option is OFF by default.
   
\OptHead{CI}{enable context-insensitive phase}

        Setting this option OFF forces the analyser to omit the
        context-insensitive phase of analysis and proceed
        directly to the context-sensitive phase,
        like it was in OSA versions 0.33 and below.

        This option is ON by default.

        See section \ref{osa:phases}. 

\OptHead{CV}{enable value calculation phase}

        Setting this option OFF disables the value calculation phase
        of analysis.

        This option is ON by default.

        See section \ref{osa:phases}.

\ifonline \else
\end{description}
\fi


\section{OSA equations}
\label{osa:equations}

All XDS compiler equations not specific to a particular back-end
may be used with OSA. See chapter \ref{options} for more information.

\ifonline \else
\begin{description}
\fi
\EquHead{TOP}{fake top-level module}

        Use this equation to specify a list of procedures that
        are to be used as starting call graph nodes when selecting the part
        of program to analyse. By default, the main module body
        is the starting node of the call graph.

        See section \ref{osa:speedup:part} for more information.

\EquHead{SAVE\_MEM}{memory limit}

        The analysed program is saved after the stage that caused OSA
        to exceed the specified amount of memory. Then, the
        \OERef{READ} equation can be used then to continue analysis.

        See section \ref{osa:speedup:save} for more information.

\EquHead{CUT}{data flow analysis stop stage number}

        If specified, this equation causes data flow analysis to stop
        after stage with the given number. \havetowrite

        See section \ref{osa:speedup:save} for more information.

\EquHead{SAVE}{data flow analysis save stage number}

        When the analysis stage with the specified number is finished,
        the analysed program is saved into a file.

        See sections \ref{osa:env} and \ref{osa:speedup:save}. 

\EquHead{READ}{continue a prefiously saved analysis}

        If specified, this equation forces OSA to read a partially
        analysed program from the given file, previously saved by
        using the \OERef{SAVE} equation or the \OERef{SAVE\_MEM} equation.
        The data-flow analysis is then continued

        See also section \ref{osa:speedup:save}. 

\EquHead{ONE}{analyse given procedures context-insensitively}

        Only the context-insensitive data flow analysis is applied 
        to listed procedures, if any. Use this feature to speedup
        analysis in exchange for its accuracy.

        See also section \ref{osa:speedup}. 

\EquHead{EU}{analysis mode of EU phase}

        Set this equation to \verb'NONE' to disable the EU (variable usage)
        phase of the Error analysis. Set it to \verb'ROUGH' to simplify 
        analysis in that phase. The default setting is \verb'FULL'.

        See also sections \ref{osa:phases} and \ref{osa:speedup:noeua}


\EquHead{INIT}{default value for variables}

        By default the initial values of all variables are undefined. 
        Using this equation, you may significantly improve
        the value calculation analysis done by OSA.

\EquHead{COLLECTMIN}{minimal amount of memory for garbage collection}

        The minimal amount of allocated memory, in megabytes, that
        causes the garbage collector to be invoked after the current
        analysis stage is completed.

        See section \ref{osa:speedup}. 

\EquHead{COLLECTMAX}{maximum amount of memory for garbage collection}

        The maximum amount of allocated memory, in megabytes, that
        causes the garbage collector to be invoked after the current
        analysis stage is completed.

        See section \ref{osa:speedup}. 

\EquHead{MSGEXT}{message file extension}

        Specifies an extension for OSA message list files.
        By default, the extension \verb'mes' is used.

\EquHead{IRFEXT}{intermediate file extension}

        Specifies an extension for intermediate analysis files.
        By default, the extension \verb'irf' is used.

\EquHead{GRFEXT}{call graph file extension}

        Specifies an extension for output files that contain
        procedure call graphs.
        By default, the extension \verb'grf' is used.

\ifonline \else
\end{description}
\fi

