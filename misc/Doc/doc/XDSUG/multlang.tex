% !!! M2/O2 and GC (if a program contains O2 module)
% !!! Review Interfacing to C (it rlates to SysCall etc. as well)
%

\chapter{Multilanguage programming}\label{multilang}
\index{multilanguage programming}

\xds{} allows you to mix \mt{}, \ot{}, C, and Assembler
modules, libraries, and object files in one project.

\section{Modula-2 and Oberon-2}\label{multilang:m2o2}
\index{multilanguage programming!Modula-2/Oberon-2}

It is not necessary to notify the compiler of using \mt{}
objects in \ot{} module and vice versa. The compiler will detect
the language automatically when processing symbol files on
\verb'IMPORT' clause.

\subsection{Basic types}

In \ot{} the basic types have the same length on all platforms. In
\mt{} the size of types {\tt INTEGER}, {\tt CARDINAL} and {\tt
BITSET} may be different and depends on the value of the
\OERef{M2BASE16} option. The following table summarizes the correspondence
between the basic types.
\begin{center}
\begin{tabular}{lcccc}
\bf Type & \bf Size & \bf \ot{} & \multicolumn{2}{c|}{\bf Modula-2}     \\
         &      &              & \tt M2BASE16+ & \tt M2BASE16-   \\ \hline
integer  & 8    & \tt SHORTINT &     ---       &     ---      \\
integer  & 16   & \tt INTEGER  & \tt INTEGER   &     ---      \\
integer  & 32   & \tt LONGINT  &     ---       & \tt INTEGER  \\
cardinal & 8    &     ---      &     ---       &     ---      \\
cardinal & 16   &     ---      & \tt CARDINAL  &     ---      \\
cardinal & 32   &     ---      &     ---       & \tt CARDINAL \\
bitset   & 16   &     ---      & \tt BITSET    &     ---      \\
bitset   & 32   & \tt SET      &     ---       & \tt BITSET
\end{tabular}
\end{center}

The system types {\tt INT} and {\tt CARD} correspond to
\mt{} {\tt INTEGER} and {\tt CARDINAL} types respectively.
We recommend to use {\tt INT} and {\tt CARD} in \ot{} when importing \mt{}
modules. For example, if the procedure {\tt Foo} is defined
in the \mt{} definition module \verb'M' as
\begin{verbatim}
    DEFINITION MODULE M;

    PROCEDURE Foo(VAR x: INTEGER);

    END M.
\end{verbatim}
its portable usage in \ot{} is as follows:
\begin{verbatim}
    VAR x: SYSTEM.INT;
       .  .  .
      M.Foo(x);
\end{verbatim}

\subsection{Data structures}

\xds{} allows any \mt{} data structures to be used in \ot{}
modules, even those that can not be defined in \ot{} (e.g.
variant records, range types, set types, enumerations, etc).

However, usage of \mt{} types in \ot{} and vice versa is
restricted. Whenever possible \xds{} tries to produce the correct
code. If a correct translation is impossible, an error
is reported:
\begin{itemize}
\item
  a \mt{} record field type cannot be of an \ot{} pointer, record
  or array type;
\item
  a \mt{} pointer to an \ot{} record cannot  be  used  in  specific
  \ot{} constructs (type-bound procedures, type guards, etc);
\item
   an opaque type can not be defined as an Oberon pointer.
\end{itemize}

Standard procedures \verb'NEW' and \verb'DISPOSE' are always applied according to
the language of a parameter's type. For example, for the following
declarations in an \ot{} module:
\begin{verbatim}
    TYPE
      Rec = RECORD END;
      MP  = POINTER ["Modula"] TO Rec; (* Modula pointer *)
      OP  = POINTER TO Rec;     (* Oberon pointer *)
    VAR
      m: MP;
      o: OP;
\end{verbatim}
the call \verb|NEW(m)| will be treated as a call to the \mt{}
default {\tt ALLOCATE}, while \verb|NEW(o)| will be treated as
a call of the standard \ot{} run-time routine.
See also \ref{multilang:direct}.

Implicit memory deallocation (garbage collection) is applied to \ot{}
objects only. If a variable of a \mt{} pointer type is declared in
an \ot{} module, it shall be deallocated explicitly.

\paragraph{Example: Using the Modula data type in Oberon}
\begin{verbatim}
    (* Modula-2*) DEFINITION MODULE m2;
    TYPE
      Rec = RECORD  (* a record with variant parts *)
        CASE tag: BOOLEAN OF
          |TRUE:  i: INTEGER;
          |FALSE: r: REAL;
        END;
      END;
      Ptr = POINTER TO Rec;

    VAR
      r: Rec;
      p: Ptr;

    PROCEDURE Foo(VAR r: Rec);

    END m2.

    (* Oberon-2 *) MODULE o2;

    IMPORT m2; (* import of a Modula-2 module *)

    VAR
      r: m2.Rec;  (* using the Modula-2 record type *)
      p: m2.Ptr;  (* using the Modula-2 pointer type *)
      x: POINTER TO m2.Rec;

    BEGIN
      NEW(p);     (* Modula-2 default ALLOCATE *)
      NEW(x);     (* Oberon-2 NEW *)
      m2.Foo(r);
      m2.Foo(p^);
      m2.Foo(x^);
    END o2.
\end{verbatim}

\subsection{Garbage collection}

It is important to remember that \mt{} and \ot{} have different
approaches to memory utilization. When a program contains both
\mt{} and \ot{} modules, garbage collection is used.
See \ref{rts:mm} for more information.

%--------------------------------------------------------------

\section{Direct language specification}\label{multilang:direct}
\index{multilanguage programming!language specification}
\index{C interface!language specification}

The compiler must know the implementation language of a
module to take into account different semantics of different
languages and to produce correct code.

In some cases, it is necessary for a procedure or data type to be
implemented according to the rules of a language other than that
of the whole module. In \xds{}, it is possible to explicitly
specify the language of a type or object. {\em Direct language
specification \index{DLS}(DLS)} is allowed either if
language extensions are enabled or if the module \verb'SYSTEM' is imported.

In a record, pointer, or procedure type declaration, or in a procedure declaration,
the desired language (or, more precisely, the way in which that
declaration is treated by the compiler) can be specified as \verb'"[" language "]"'
immediately following the keyword \verb'RECORD', \verb'POINTER',
or \verb'PROCEDURE'. \verb'language' can be a string or integer constant
expression\footnote{We recommend to use strings, integer values are
preserved for backward compatibility.}:
\begin{center}
\begin{tabular}{lll}
\bf Convention & \bf String    & \bf Integer \\
\hline
Oberon-2       & \tt "Oberon"  & \tt 0       \\
Modula-2       & \tt "Modula"  & \tt 1       \\
C              & \tt "C"       & \tt 2       \\
Pascal         & \tt "Pascal"  & \tt 5       \\
Win32 API      & \tt "StdCall" & \tt 7       \\
OS/2 API       & \tt "SysCall" & \tt 8       \\
\end{tabular}
\end{center}

Examples:

\verb'    TYPE'\\
\verb'      UntracedPtr = POINTER ["Modula"] TO Rec;'

Here {\tt UntracedPtr} is defined as a \mt{} pointer, hence all variables
of that type will not be traced by garbage collector.

\verb'    PROCEDURE ["C"] sig_handler (id : SYSTEM.int);'\\
\verb'     .  .  .'\\
\verb'      signal.signal(signal.SYSSEGV, sig_handler);'\\

Here \verb'sig_handler' has C calling and naming conventions, so it
can be installed as a signal handler into C run-time support.

A direct language specification clause placed after a name of a field,
constant, type, or variable points out that the name of the object
will be treated according to the rules of the specified
language.
\begin{verbatim}
    TYPE
      Rec ["C"] = RECORD
        name ["C"]: INTEGER;
      END;

    CONST pi ["C"] = 3.14159;

    VAR buffer[]["C"]: POINTER TO INTEGER;
\end{verbatim}
{\bf Note:} In ISO \mt{}, an absolute address may be specified
for a variable after its name in square brackets, so
the empty brackets are required in the last line.

A procedure name is treated according to the language of
its declaration, so in the following declaration:

\verb'    PROCEDURE ["C"] Foo;'

both the procedure type and the procedure name are treated
according to the C language rules.
{\bf Note:} If you are using a C++ compiler, the {\tt Foo}
function should be declared with C name mangling style.
Consult your C++ manuals for further information.

%----------------------------------------------------------------

\section{Interfacing to C}\label{multilang:C}
\index{multilanguage programming!interface to C}
\index{C interface}

Special efforts were made in \XDS{} to provide convenient
interface to other languages, primarily to the C language.
The main goal is to allow direct usage of existing C libraries and
APIs in \mt{}/\ot{} programs.

\subsection{Foreign definition module}
\index{foreign definition module}

A \See{direct language specification}{}{multilang:direct}
clause may appear immediately after keywords \verb'DEFINITION'
\verb'MODULE'. The effect is that all objects defined in
that module are translated according to the specified
language rules, thus making unnecessary direct language
specifications for each object.

Several options are often used in foreign definition modules.
\ifgenc
See \ref{maptoc:opt:foreign} for the
description of options used to create a foreign definition module.
\fi

\Example
\begin{verbatim}
    <*+ M2EXTENSIONS *>
    <*+ CSTDLIB *>      (* C standard library *)
    <*+ NOHEADER *>     (* we already have header file *)
    DEFINITION MODULE ["C"] string;

    IMPORT SYSTEM;

    PROCEDURE strlen(s: ARRAY OF CHAR): SYSTEM.size_t;
    PROCEDURE strcmp(s1: ARRAY OF CHAR;
                     s2: ARRAY OF CHAR): SYSTEM.int;
    END string.
\end{verbatim}

Take the following considerations into account when designing
your own foreign definition module:
\begin{itemize}
\item
        If you are developing an interface to an existing header file,
        use the {\bf NOHEADER} option to disable generation of the header
        file.  This option is meaningful for translators only.
\item
        If the header file is a standard header file, use the
        {\bf CSTDLIB} option. This option is meaningful for the translators
        only.
\item
        Use the special \verb'SYSTEM' types {\tt int}, {\tt unsigned},
        \verb|size_t|, and {\tt void} for corresponding
        C types.
\item
        \xds{} compilers use relaxed type compatibility rules
        for foreign entities. See \ref{multilang:relax} for
        more information.

\end{itemize}
\ifgenc
Definition modules for ANSI C libraries ({\tt stdio.def},
{\tt string.def}, etc) can be used as tutorial examples.
\fi

\subsection{External procedures specification}\label{multilang:extproc}
\index{multilanguage programming!external procedures}
\index{C interface!external procedures}

In some cases, it may be desirable not to write a foreign
definition module but to use some C or API functions directly.
\xds{} compilers allow a function to be declared as external.

The declaration of an external procedure consists of a procedure
header only. The procedure name in the header is prefixed by the
symbol \verb|"/"|.

\verb'    PROCEDURE ["C"] / putchar(ch: SYSTEM.int): SYSTEM.int;'

%------------------------------------------------------------------

\section{Relaxation of compatibility rules}
\label{multilang:relax}

The compiler performs all semantic checks for an object or type
according to its language specification. Any object declared as that
of \mt{} or \ot{} is subject to \mt{} or \ot{} compatibility rules
respectively.  The compiler uses relaxed compatibility rules for
objects and types declared as \verb'"C"', \verb'"Pascal"',
\verb'"StdCall"', and \verb'"SysCall"'.

\subsection{Assignment compatibility}

Two pointer type objects are considered assignment compatible, if

\begin{itemize}
\item they are of the same \mt{} or \ot{} type.
\item at least one of their types is declared as
      \verb'"C"', \verb'"Pascal"', \verb'"StdCall"', or \verb'"SysCall"',
      and their {\it base types} are the same.
\end{itemize}

\begin{verbatim}
    VAR
      x: POINTER TO T;
      y: POINTER TO T;
      z: POINTER ["C"] TO T;
    BEGIN
      x := y;       -- error
      y := z;       -- ok
      z := y;       -- ok
\end{verbatim}

\subsection{Parameter compatibility}
\label{multilang:parmcomp}

For procedures declared as \verb'"C"', \verb'"Pascal"',
\verb'"StdCall"', or \verb'"SysCall"', the
type compatibility rules for parameters are significantly relaxed:

If a formal value parameter is of the type declared as \verb'POINTER TO T',
the actual parameter can be of any of the following types:

\begin{itemize}
\item the same type (the only case for regular \mt{}/\ot{}
      procedures);
\item another type declared as \verb'POINTER TO T'.
\item any array type which elements are of type {\tt T}.
      In this case the address of the first array element is passed,
      as it is done in C.
\item the type {\tt T} itself, if {\tt T} is a record type.
      In this case the address of the actual parameter is passed.
\end{itemize}

If a formal parameter is an open array of type \verb'T',
the actual parameter can be of any of the following types:

\begin{itemize}
\item an (open) array of type \verb'T' (the only case for regular \mt{}/\ot{}
      procedures);
\item type \verb'T' itself (if \OERef{M2EXTENSIONS}\iftopspeed or \OERef{TOPSPEED}\fi
      option is set ON);
\item any type declared as \verb'POINTER TO T'.
\end{itemize}

This relaxation, in conjunction with the
\See{{\tt SYSTEM.REF} function procedure}{}{m2:sysfunc:nonstandard},
simplifies \mt{}/\ot{} calls to C libraries and the target operating
system API, preserving the advantages of the type checking mechanism
provided by that languages.

\Example
\begin{verbatim}
    TYPE
      Str = POINTER TO CHAR;
      Rec = RECORD ... END;
      Ptr = POINTER TO Rec;

    PROCEDURE ["C"] Foo(s: Str); ... END Foo;
    PROCEDURE ["C"] Bar(p: Ptr);  ... END Bar;
    PROCEDURE ["C"] FooBar(a: ARRAY OF CHAR);  ... END FooBar;

    VAR
      s: Str;
      a: ARRAY [0..5] OF CHAR;
      p: POINTER TO ARRAY OF CHAR;
      R: Rec;
      A: ARRAY [0..20] OF REC;
      P: POINTER TO REC;

      Foo(s);    (* allowed - the same type *)
      Foo(a);    (* allowed for the "C" procedure *)
      Foo(p^);   (* allowed for the "C" procedure *)
      Bar(R);    (* the same as Bar(SYSTEM.REF(R)); *)
      Bar(A);    (* allowed for the "C" procedure *)
      Bar(P);    (* allowed for the "C" procedure *)
      FooBar(s); (* allowed for the "C" procedure *)
\end{verbatim}

\subsection{Ignoring function result}
\index{C interface!using C functions}

It is a standard practice in C programming to ignore the result of
a function call. Some standard library functions are designed
taking that practice into account. E.g. the string copy
function accepts the destination string as a variable parameter (in
terms of \mt{}) and returns a pointer to it:

\verb'    extern char *strcpy(char *, const char *);'

In many cases, the result of the {\tt strcpy} function call is ignored.

In \xds{}, it is possible to ignore results of functions
defined as \verb'"C"', \verb'"Pascal"', \verb'"StdCall"', or \verb'"SysCall"'.
Thus, the function {\tt strcpy} defined in the \verb'string.def'
foreign definition module as

\verb'    PROCEDURE ["C"] strcpy(VAR d: ARRAY OF CHAR;'\\
\verb'                               s: ARRAY OF CHAR): ADDRESS;'

can be used as a proper procedure or as function procedure:

\verb'    strcpy(d,s);'\\
\verb'    ptr:=strcpy(d,s);'

% This section has to be included if and only if gencode is true.
\ifgencode

% !!! DS_EQ_SS - throw away?

\section{Configuring XDS for a C Compiler}\label{multilang:ccomp}

% CC equation: WATCOM/SYMANTEC/BORLAND/MSVC/OS2SYSCALL
%       options: GENCPREF, DS_NEQ_SS, ONECODESEG

Different C compilers have different naming and calling conventions.
If you use C functions or libraries in your projects,
you have to specify your C compiler using the \OERef{CC} equation in
order to have all C functions to be called in a way compatible with
that compiler. The compiler also sets the default values of some
other options and equations according to the value of the \OERef{CC} equation.
\iflinux \else See \ref{multilang:ccomp:opt}. \fi % this section not included for Linux

\iflinux
  For Linux \XDS{} supports the GCC (ELF) compiler. Therefore, the \OERef{CC}
  equation has to be set to \verb'"GCC"', written in any case. If the equation value
  is not set, \verb'"GCC"' is assumed by default.
\fi

\ifwinnt
  For Windows NT and Windows 95 \XDS{} supports the MSVC++ and Watcom
  (stack convention) compilers. The corresponding values of the \OERef{CC}
  equation are \verb'"MSVC"' and \verb'"WATCOM"', written in any case.
  If the equation is not set, the compiler will assume \verb'"WATCOM"'
  by default. Add the line

  \verb'    -cc=Watcom'

  or

  \verb'    -cc=MSVC'

  to your configuration file.
\fi

Alignment of data structures is controlled by the \OERef{ALIGNMENT} equation.

\iflinux
  {\bf ATTENTION!} Libraries included in \XDS{} distribution are built via
  GCC. Since GCC usually produces aligned code, the \OERef{ALIGNMENT} equation
  has to be set to 4. Setting it to other values may cause unpredictable results.
  Don't change it unless you exactly know what you are doing!
\fi

Names in an object file produced by a C compiler may have leading
underscore. If you are going to use C modules and libraries, you have to
force \XDS{} to use the same naming rules. To do this, turn the
\OERef{GENCPREF} option ON in the foreign definition modules:

\verb'    <* +GENCPREF *>' \\
\verb'    DEFINTION MODULE ["C"] stdio;'

\iflinux
Since GCC (ELF) produces no underscore prefixes you should not turn this option
ON.
\fi

\ifcomment !!!
See {\tt samples.txt} from \XDS{} on-line documentation for more
information.
\fi

\subsection{Possible problems}

To use a C function or a data type from \mt{} or \ot{} you have
to express its type in one of these languages. Usually
it is done in a foreign definition module (See \ref{multilang:C}).
The current version of \XDS{} does not support
all calling conventions, so direct usage of some functions
is not possible, namely:
\begin{itemize}
\item functions with a parameter of a structured type, passed by value,
      e.g.:

\verb'    void foo(struct MyStruct s);'

\item functions that return structured types, e.g.:

\verb'    struct MyStruct foo(void)'

\item C functions with Pascal calling convention that return
      a real type.

\iflinux \else % not Linux
\item functions that are compiled with non-stack calling conventions.
      {\bf Note:} stack calling conventions shall be set for
      Watcom using "-3s", "-4s", or "-5s" option.
\fi
\end{itemize}

\iflinux \else % not Linux
  \XDS{} does not support usage of data structures with
  non-standard alignments. If the \OERef{ALIGNMENT} equation is set
  to {\em n}, use the option "-zp{\em n}" for Watcom C and
  "-Zp{\em n}" for MSVC.
\fi

Both \mt{} and C/C++ have exception handling and finalization
facilities.  Unpredictable results may occur
if you try to utilize that facilities from both languages in one program.

\iflinux \else % not Linux
\subsection{Using an unsupported compiler}

\XDS{} does not support all available C compilers.
You can use additional configuration options (See \ref{multilang:ccomp:opt})
to adapt \XDS{} to your C compiler.
The \OERef{DEFLIBS} option should be switched off in that case.

It may be necessary to make some changes in the run-time support
or to build a special version of the library for a particular C compiler.
It can be done under terms of a special support program.

\subsection{Additional configuration options}\label{multilang:ccomp:opt}

The following options can be used to adapt \XDS{} to an unsupported
C compiler. We recommend to use these options with care.

\begin{itemize}
\ifcomment
\item[DS\_NEQ\_SS] \index{DS\_NEQ\_SS@{\bf DS\_NEQ\_SS}} \mbox{}

        If the option is ON, the compiler assumes that DS register
        is not equal to SS register.
\fi

\item The \OERef{GENCPREF} option controls whether the compiler should precede
      all external names in object files with an underscore prefix (ON),
      or leave them unchanged (OFF).

\item The \OERef{ONECODESEG} option controls whether the compiler should
      produce a single code segment for all code of a module (ON),
      or a separate code segment for each procedure (OFF).
\end{itemize}

The table below shows the default values of these options for the supported
C compilers:
\begin{center}
\begin{tabular}{lcc} \
\OERef{CC} setting & \tt WATCOM & \tt  MSVC    \\
\hline
\ifcomment
        DS\_NEQ\_SS &   ?    &    ?      \\
\fi
\OERef{GENCPREF}    &   OFF  &    ON     \\
\OERef{ONECODESEG}  &   OFF  &    ON
\end{tabular}
\end{center}
\fi % not Linux

\fi % not gencode

\ifcomment % ------------------------------------------------------
%!!!!

\section{Interfacing to other languages}
\index{multilanguage programming!other languages}

Although \xds{} compiles to ANSI C, it is possible to use {Pascal} and
{FORTRAN} libraries from \mt{}/\ot{} sources.

Most C compilers can recognize the special keyword {\tt pascal}
(and/or {\tt fortran}), immediately preceding a C function name.

\begin{verbatim}
void pascal PrintScreen (void);
/* pascal function declaration */
\end{verbatim}

The naming and parameter passing conventions are different in
C and in {Pascal}/{FORTRAN}, and therefore a special keyword is necessary.
If your C compiler supports the foreign library interface, you can use
{Pascal}/{FORTRAN} procedures, variables, data types, etc. in the
same way as C objects.

To do it, you should first create the corresponding C header file if it
does not exist. Consult your C compiler documentation for details.

After this, a Modula-2 definition module should be created.
Special keywords such as {\tt pascal}, {\tt far}, {\tt volatile} may
be omitted in the definition module.
Compile this definition file with the {\bf NOHEADER}
option to produce a symbol file.

Having the Pascal declaration
\begin{verbatim}
  function Foo(p: real): real;
\end{verbatim}
the respective C header file line should be
\begin{verbatim}
  float pascal far Foo(float p);
\end{verbatim}

and the definition file
\begin{verbatim}
<*+ NOHEADER *>
DEFINITION MODULE [2] Pascals;

  PROCEDURE Foo(p: REAL): REAL;
END Pascals.
\end{verbatim}

\fi % comment -----------------------------------------------------
