\chapter{ISO Modula-2 standard libraries}\label{lib:ISO}
\index{ISO libraries}\index{libraries!ISO}

It is our aim to provide the full set of ISO Modula-2 libraries.
However, some modules are unimplemented on a particular
hardware/sortware platform. % !!! (See Chapter \ref{limits}).
System libraries are described in the {\em XDS Modula-2} Chapter of the
{\em XDS User's Guide}.

This chapter does not contain a complete reference. A brief
description and some samples are provided for each group of
modules. For more information, refer to the {\em ISO Modula-2
Library Reference}.

\section{Input/output library}\label{lib:ISO:IO}

The IO library allows one to read and write the data streams
over one or more channels. Channels are connected to the source
of input data, or to destination of output data, known as
devices. A set of devices can be extended.

A group of modules is provided to operate on the default input and
output channel (\ref{lib:ISO:stdio}). Another group of modules provide
facilities to operate on channels specified explicitly by a parameter
(\ref{lib:ISO:chanio}). The device modules provide facilities to get a
channel connected to a source (\ref{lib:ISO:device}). The primitive
device-independent operations are provided by the module {\bf IOChan};
the module {\bf IOLink} allows specialized device module
to be implemented (See \ref{lib:ISO:lowio}).

\subsection{Reading and writing via default channels}
\label{lib:ISO:stdio}

The following modules provide procedures that operate via default
input and output channels and do not take a parameter which identifies
a channel:
\begin{flushleft}
\begin{tabular}{ll}
\bf IOConsts  & Types and constants for IO modules                     \\
\bf SLongIO   & LONGREAL numbers IO operations                         \\
\bf SRawIO    & Raw IO operations (no conversion or interpretation)    \\
\bf SRealIO   & REAL numbers IO operations                             \\
\bf SResultIO & Read results for the default input channel             \\
\bf STextIO   & Character and string types IO operations               \\
\bf SWholeIO  & Whole numbers IO operations                            \\
\end{tabular}
\end{flushleft}
\lindex{IOConsts}
\lindex{SLongIO}
\lindex{SRawIO}
\lindex{SRealIO}
\lindex{SResultIO}
\lindex{STextIO}
\lindex{SWholeIO}

\noindent
The module {\bf STextIO} resembles the well-known {\bf InOut} library.
The {\em Hello, World} program is implemented in the following example:
\begin{verbatim}
MODULE Hello;

IMPORT  STextIO;

BEGIN
  STextIO.WriteString('Hello, World!');
  STextIO.WriteLn;
END Hello.
\end{verbatim}

The modules {\bf SWholeIO}, {\bf SRealIO}, {\bf SLongIO} provides
facilities for the input and output of whole and real numbers in
a decimal form using text operations on a channel.

\begin{verbatim}
PROCEDURE Print(stage: CARDINAL; val: REAL);
BEGIN
  STextIO.WriteString("On stage");
  SWholeIO.WriteCard(stage,0);
  STextIO.WriteString(" the value is equal to ");
  SRealIO.WriteReal(val,15);
  STextIO.WriteLn;
END Print;
\end{verbatim}

The module {\bf SIOResult} allows one to determine whether the last
input operation from a default input channel succeed. Text
operations produce or consume data streams as a sequence of
characters and line marks. The text input procedures (such as
{\tt ReadString} never remove a line mark from the input stream.
The procedure {\tt SkipLine} should be used to pass a line mark.

The {\tt Copy} procedure reads strings from the input channel and
copies them to the output channel.
\begin{verbatim}
PROCEDURE Copy;
  VAR s: ARRAY [0..63] OF CHAR;
BEGIN
  LOOP
    STextIO.ReadString(s);
    CASE SIOResult.ReadResult() OF
     |SIOResult.allRight:
        STextIO.WriteString(s);
     |SIOResult.endOfLine:
        STextIO.SkipLine;
        STextIO.WriteLn;
     |SIOResult.endOfInput:
        EXIT
    END;
  END;
END Copy;
\end{verbatim}

\noindent
No procedure is provided to get the result of a `write'
operation. Device errors are reported by raising an exception
(See module {\bf IOChan}).

\subsection{Reading and writing data}\label{lib:ISO:chanio}

For all modules in this group a channel is specified by an actual
parameter of the type {\bf IOChan.ChanId}.
\begin{flushleft}
\begin{tabular}{ll}
\bf IOResult &   Read results for specified channels                     \\
\bf LongIO &     LONGREAL numbers IO operations                         \\
\bf RawIO &      Raw IO operations (no conversion or interpretation)     \\
\bf RealIO &     REAL numbers IO operations                             \\
\bf TextIO &     Character and string types IO operations                \\
\bf WholeIO &    Whole numbers IO operations                            \\
\end{tabular}
\end{flushleft}
\lindex{IOResult}
\lindex{LongIO}
\lindex{RawIO}
\lindex{RealIO}
\lindex{TextIO}
\lindex{WholeIO}

\noindent
The following procedure copies an input channel to an output channel
byte by byte:
\begin{verbatim}
PROCEDURE CopyChars(in,out: IOChan.ChanId);
  VAR ch: CHAR;
BEGIN
  LOOP
    TextIO.ReadChar(in,ch);
    CASE IOResult.ReadResult(in) OF
     |IOResult.allRight:
        TextIO.WriteChar(out,ch);
     |IOResult.endOfLine:
        TextIO.SkipLine(in);
        TextIO.WriteLn(out);
     |IOResult.endOfInput:
        EXIT
    END;
  END;
END CopyChars;
\end{verbatim}

\subsection{Device modules}\label{lib:ISO:device}

The device modules allows to get a channel connected to a stream,
a file, program arguments and to default channels.

\begin{flushleft}
\begin{tabular}{ll}
\bf ChanConsts  & Common types and values for channel open \\
                & requests and results                                   \\
\bf ProgramArgs & Access to program arguments                            \\
\bf RndFile     & Random access files                                    \\
\bf SeqFile     & Rewindable sequential files                            \\
\bf StdChans    & Access to standard and default channels                \\
\bf StreamFile  & Independent sequential data streams                    \\
\end{tabular}
\end{flushleft}
\lindex{ChanConsts}
\lindex{ProgramArgs}
\lindex{RndFile}
\lindex{SeqFile}
\lindex{StdChans}
\lindex{StreamFile}

\noindent
In the following example a channel connected to a rewindable
file is opened:
\begin{verbatim}
MODULE Example;

IMPORT  SeqFile, STextIO, TextIO;

CONST flags = SeqFile.text + SeqFile.old;

VAR
  cid: SeqFile.ChanId;
  res: SeqFile.OpenResults;
  i  : CARDINAL;

BEGIN
  SeqFile.OpenWrite(cid,"example.dat",flags,res);
  IF res = SeqFile.opened THEN
    FOR i:=0 TO 9 DO
      TextIO.WriteString(cid,"Hello");
      TextIO.WriteLn(cid);
    END;
    SeqFile.Close(cid);
  ELSE
    STextIO.WriteString("Open error");
    STextIO.WriteLn;
  END;
END Example.
\end{verbatim}

The module {\bf StdChans} allows one to get channels already open to
sources and destinations of standard input, standard output and
standard error output. Default channels initially corresponds to
the standard channels, but their values may be changed to
redirect input or output.

\begin{verbatim}
PROCEDURE RedirectOutput(cid: StdChans.ChanId);
BEGIN
(* writing to the current default channel: *)
  STextIO.WriteString("Redirecting output...");
  STextIO.WriteLn;
(* redirecting output: *)
  StdChans.SetOutChan(cid);
END RedirectOutput;
\end{verbatim}

After the call of {\tt RedirectOutput(cid)} all subsequent output
via modules {\bf STextIO}, {\bf SWholeIO}, etc will be written to
the channel {\tt cid}. To restore output call
\begin{verbatim}
  StdChans.SetOutChan(StdChans.StdOutChan());
\end{verbatim}

The module {\bf ProgramArgs} provides a channel to access
program's arguments. The following program prints all its arguments.
\begin{verbatim}
MODULE Args;

IMPORT ProgramArgs, TextIO, STextIO;

VAR
  str: ARRAY [0..255] OF CHAR;
  cid: ProgramArgs.ChanId;

BEGIN
  cid:=ProgramArgs.ArgChan();
  WHILE ProgramArgs.IsArgPresent() DO
    TextIO.ReadToken(cid,str);
    (* Note: read result test is omitted *)
    STextIO.WriteString(str); STextIO.WriteLn;
  END;
END Args.
\end{verbatim}

\subsection{Low-level IO modules}\label{lib:ISO:lowio}

Two low-level modules are described in this section. The module
{\bf IOChan} defines the type {\tt ChanId} that is used to
identify channels and provides a set of procedures forming the
channel's interface in a device-independent manner.
\lindex{IOChan}

The module {\bf IOLink} provides facilities that allow one to define
new device modules. Let us implement an encryption channel, i.e.
a channel that encrypts all information that is written to it. To
make the encryption device-independent we need a channel for input/output
operations.
\lindex{IOLink}

In the following example a sketch of the encryption device module
is shown.
\begin{verbatim}
DEFINITION MODULE EncryptChan;

IMPORT IOChan, ChanConsts;

TYPE
  ChanId = IOChan.ChanId;
  OpenResults = ChanConsts.OpenResults;

PROCEDURE Connect(VAR cid: ChanId;
                       io: ChanId;
                  VAR res: OpenResults);
(* Attempts to open an encryption channel. All I/O
   operations will be made through "io" channel.
*)

PROCEDURE Close(VAR cid: ChanId);
(* Closes the channel. *)

END EncryptChan.
\end{verbatim}

Values of the type {\tt DeviceId} are used to identify device
modules. By calling the procedure {\tt DeviceTablePtrValue}, a
device module can obtain a pointer to a device table for the
channel. Each channel has it own copy of a device table. A
device table contains a field in which the device module can
store private data. In our example, the {\tt io} channel will be
stored in this field. The device table also serves as a method
table (or virtual function table) in object-oriented languages.
It contains the procedure variables for each device procedure. All
fields are initialized by the call of {\tt MakeChan} procedure. A
device module has to assign its own device procedures to the
fields of a device table. See the {\tt Connect} procedure below.

\begin{verbatim}
IMPLEMENTATION MODULE EncryptChan;

IMPORT IOChan, IOLink, ChanConsts, SYSTEM;

(* "did" is used to identify the channel's kind: *)
VAR did: IOLink.DeviceId;

PROCEDURE EncryptChar(from: SYSTEM.ADDRESS;
                         i: CARDINAL;
                    VAR ch: CHAR);
BEGIN
  ch:='a'; (* very simple encryption :-) *)
END EncryptChar;

PROCEDURE TextWrite(x: IOLink.DeviceTablePtr;
                 from: SYSTEM.ADDRESS;
                  len: CARDINAL);
  VAR i: CARDINAL;
     ch: CHAR;
    cid: IOChan.ChanId;
BEGIN
  (* get the channel id *)
  cid:=SYSTEM.CAST(IOChan.ChanId,x^.cd);
  FOR i:=0 TO len-1 DO
    (* encrypt i-th character *)
    EncryptChar(from,i,ch);
    (* write an encrypted character *)
    IOChan.TextWrite(cid,SYSTEM.ADR(ch),1);
  END;
END TextWrite;

PROCEDURE Connect(VAR cid: ChanId;
                       io: ChanId;
                  VAR res: OpenResults);
  VAR x: IOLink.DeviceTablePtr;
BEGIN
  IOLink.MakeChan(did,cid);
  IF cid = IOChan.InvalidChan() THEN
    res:=ChanConsts.outOfChans
  ELSE
    (* get a pointer to the device table *)
    x:=IOLink.DeviceTablePtrValue(cid,did,
                       IOChan.notAvailable,"");
    (* store the channel id *)
    x^.cd:=SYSTEM.CAST(SYSTEM.ADDRESS,io);
    x^.doTextWrite:=TextWrite;
    (* ... *)
  END;
END Connect;

PROCEDURE Close(VAR cid: ChanId);
BEGIN
  IOLink.UnMakeChan(did,cid);
END Close;

BEGIN
  IOLink.AllocateDeviceId(did);
END EncryptChan.
\end{verbatim}

The module {\bf EncryptChan} can be used as any standard
device module.

\section{String conversions}\label{lib:ISO:conv}

The string conversion library admits the conversion of the values
of numeric data types to and from the character string
representation. It contains the following modules:
\begin{flushleft}
\begin{tabular}{ll}
\bf ConvTypes &    Common types used in the string conversion modules\\
\bf LongConv &     Low-level LONGREAL/string conversions         \\
\bf LongStr &      LONGREAL/string conversions                   \\
\bf RealConv &     Low-level REAL/string conversions         \\
\bf RealStr &      REAL/string conversions                   \\
\bf WholeConv &    Low-level whole number/string conversions         \\
\bf WholeStr &     Whole number/string conversions                   \\
\end{tabular}
\end{flushleft}
\lindex{ConvTypes}
\lindex{LongConv}
\lindex{LongStr}
\lindex{RealConv}
\lindex{RealStr}
\lindex{WholeConv}
\lindex{WholeStr}

\noindent
The module {\bf ConvTypes} defines the enumeration type {\tt
ConvResults}. It also defines the types {\tt ScanClass} and {\tt
ScanState} to use in the low-level conversion modules.

The low-level conversion modules allow to control lexical
scanning of character sequences. For example, the {\bf WholeConv}
module implements procedures {\tt ScanInt} and {\tt ScanCard}
representing the start state for a finite state scanner for
signed and unsigned whole numbers. In the following example the procedure
{\tt ScanInt} is used to locate a
position of the first character in a string which is not
a part of an integer.
\begin{verbatim}
PROCEDURE SkipInt(str: ARRAY OF CHAR;
              VAR pos: CARDINAL);
  VAR len: CARDINAL;
    state: ConvTypes.ConvState;
    class: ConvTypes.ConvClass;
BEGIN
  pos:=0; len:=LENGTH(str);
  state:=WholeConv.ScanInt;
  WHILE pos < len DO
    state(str[pos],class,state);
    IF   (class = WholeConv.invalid)
      OR (class = WholeConv.terminator)
    THEN
      RETURN
    END;
    INC(pos);
  END;
END SkipInt;
\end{verbatim}

\section{Mathematical libraries}

The following modules constitute a mathematical library:
\begin{flushleft}
\begin{tabular}{ll}
\bf ComplexMath     & \verb'COMPLEX' mathematical functions      \\
\bf LongComplexMath & \verb'LONGCOMPLEX' mathematical functions  \\
\bf LongMath        & \verb'LONGREAL' mathematical functions     \\
\bf RealMath        & \verb'REAL' mathematical functions         \\
\end{tabular}
\end{flushleft}
\lindex{ComplexMath}
\lindex{LongComplexMath}
\lindex{LongMath}
\lindex{RealMath}

\section{Processes and Semaphores}

The following modules allows concurrent algorithms to be expressed using processes:
\begin{flushleft}
\begin{tabular}{ll}
\bf Processes  & Provides process creation and manipulation facilities. \\
\bf Semaphores & Provides mutual exclusion facilities for use by processes. \\
\end{tabular}
\end{flushleft}
\lindex{Processes}
\lindex{Semaphores}

\Example
\begin{verbatim}
MODULE Test;

IMPORT Processes, Semaphores, STextIO;

VAR
  sig : Semaphores.SEMAPHORE;
  prs : Processes.ProcessId;
  main: Processes.ProcessId;

PROCEDURE Proc;
BEGIN
  STextIO.WriteString('Proc 1'); STextIO.WriteLn;
  Semaphores.Claim(sig); (* suspend until released *)
  STextIO.WriteString('Proc 2'); STextIO.WriteLn;
  Processes.StopMe;
END Proc;

BEGIN
  STextIO.WriteString('Main 1'); STextIO.WriteLn;
  Semaphores.Create(sig,0);
  main:=Processes.Me();
  Processes.Start(Proc,40000,Processes.UrgencyOf(main)+1,NIL,prs);
  STextIO.WriteString('Main 2'); STextIO.WriteLn;
  Semaphores.Release(sig);
  STextIO.WriteString('Main 3'); STextIO.WriteLn;
  Processes.StopMe;
END Test.
\end{verbatim}

\section{Other libraries}

In this section we list the ISO modules that do not belong
to any of the above groups:
\begin{flushleft}
\begin{tabular}{ll}
\bf CharClass &  provides predicates to test a value
                 of a character type                             \\
\bf Storage &    Facilities for allocating
                 and deallocating storage                          \\
\bf Strings &    Facilities for string manipulation               \\
\bf SysClock &   Access to a system clock
\end{tabular}
\end{flushleft}
\lindex{CharClass}
\lindex{Storage}
\lindex{Strings}
\lindex{SysClock}


