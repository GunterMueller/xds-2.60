\chapter{Getting started}
\label{start}

\section{General considerations}
\label{start:general}

This section introduces some terms and principles to help you better
understand the debugger.

\subsection{Executable lines}
\label{start:general:executable}

In XD, a source line is considered {\it executable} if there are CPU
instructions corresponding to it. \See{Breakpoints}{}{dialog:breaks} may
be placed only on executable lines, only executable lines are displayed
in the \See{{\bf Mixed} mode}{}{dialog:navigating:modes}, etc.
Obviously, source lines containing comments or declarations are not executable.
However, native code XDS compilers may generate no code even for lines
containing statements, because of optimization or, for instance,
if that code would be unreachable anyway:
\begin{verbatim}
    PROCEDURE P(i : CARDINAL): CARDINAL;
    BEGIN
      CASE i OF
      |0: RETURN 1
      |1: RETURN P(i-1)
      END;
      RETURN 2  (* Warning: Unreachable code *)
    END P;
\end{verbatim}
In the above example, passing any value other than 0 or 1
as an input parameter to \verb'P' would cause the language exception
\verb'caseSelectException' to be raised. Hence, the code
corresponding to the last \verb'RETURN' statement would never
be executed and is omitted in the resulting object file.
The "Unreachable code" warning is issued during compilation.

\subsection{Procedure prologue and epilogue}
\label{start:general:epilogue}

In addition to the code correspondent to a procedure's body, a compiler
usually inserts some initialization (so called {\em prologue})
after its entry point and some cleanup ({\em epilogue})
before the \verb'RET' instruction. In particular,
the XDS compiler may generate code which
saves and restores stack frame, installs and removes exception handlers, etc.

The compiler maps a prologue to the source line containing the procedure's \verb'BEGIN',
and an epilogue to the line containing \verb'END'.

\subsection{Register variables}
\label{start:general:regvars}

Even if you turn the {\bf NOOPTIMIZE} option on, the compiler still performs
some optimizations. In particular, it may place local variables onto CPU registers.
The debugger prepends names of register variables with an exclamation mark in
\See{information windows}{}{dialog:intro:info}.
Such variables may not exist at certain points of procedure execution, so the
displayed values may be incorrect and it is not possible to modify a register
variable.

When register variables are used in \See{expressions}{Chapter }{expr},
they are treated as unsigned integers.

\subsection{Program startup}
\label{start:general:startup}

In a C program, the \verb'main()' function has the public name
\verb'main' or \verb'_main'. To ensure compatibility with third-party
debuggers and other tools, XDS compilers assign one of these names
to the main module body entry point, depending on the {\bf CC} option setting.
After the program is loaded, XD searches its symbol table for
these symbols and automatically executes the program's startup code
up to the point the located symbol refers to. If you wish to debug the startup
code, use the \See{{\bf Restart at startup}}{}{dialog:executing:startup} command
or specify the \verb'/S' option in XD command line.

If the program does not have a symbol table or there is no
\verb'main' or \verb'_main' symbol in the code segment,
the program will be stopped at the first possible point, 
as if the \See{{\bf Restart at startup}}{}{dialog:executing:startup}
command was used.

\section{Preparing a program for debugging}
\label{start:prepare}

Before invoking the debugger, ensure that your program is built with
line number and symbolic debug information. To do this, switch the
{\bf LINENO} and {\bf GENDEBUG} options on in the project file
or on the command line. You may also wish to switch on the
{\bf NOOPTIMIZE} option in order to disable the machine-independent
optimizer. Use the {\bf =ALL} submode to force all
modules which constitute your program to be recompiled with
these option settings.

Do not forget to specify a linker option (\verb'/DEBUG' for XDS Linker)  % ref ???
which would cause debugging information to be bound to the executable.
If you are taking advantage of the XDS make facility, % \extern???
it will be done automatically if you specify the {\bf GENDEBUG} compiler option,
provided that you use a template file bundled with your XDS package.
Refer to an appropriate documentation if you use a third-party linker
which is not supported in standard template files.

% ??? some words on debugging optimized code
% ??? How to do this from the IDE?
% ??? Note: it may be needed to edit the source of the samples
% ??? Write about using the redirection file (strip path option)

\subsection{Debug information formats}

The Native XDS-x86 compiler is capable to produce debug information
in two formats: CodeView (default) and HLL4.
This is controlled by the {\bf DBGFMT} compiler
option. The HLL4 format is more suitable for Modula-2/Oberon-2
programs. For instance, it allows to represent sets and enumerations,
which are not supported in CodeView. XDS Linker and XDS Debugger
accept both formats on both platforms, so use HLL4 whenever possible.

HLL4 support under Windows is probably an unique feature,
so if you use any third-party tools, CodeView is your only option.

\section{Starting XD}
\label{start:invoking}

To invoke XD from the command line, type

\verb'    xd [<options>] [<executable> [<arguments>]]'

or

\verb'    xd /b [<options>] <control file>'

and press {\bf Enter}. Available options are:

\begin{tabular}{l|l}
\tt /D          & Invoke XD in dialog mode (default) \\
\tt /B          & Invoke XD in batch mode \\
\tt /H          & Print a brief help message \\
\tt /J          & Safe log \\
\tt /L          & Create log file \\
\tt /N          & Display file names without path \\
% \tt /R={\it transport},{\it host} & Remote debug
\tt /S          & Debug startup code \\
\tt /X{\it nnn} & Set width of XD session window to {\it nnn}  \\
\tt /Y{\it nnn} & Set height of XD session window to {\it nnn}
\end{tabular}

If you invoke XD without any parameters, it will start in dialog
mode and display the {\bf Load Program} dialog.

\section{Terminating XD}

To exit the dialog mode, select {\bf Exit} from the {\bf File}
menu or press {\bf Alt-X}. If the debugger was started in the
dialog mode, it will terminate. Otherwise (the dialog mode
was activated from a control file), control file proceccing will
continue.

To interrupt XD running in the batch mode, press {\bf Ctrl+C}. % ??? Is this true.

