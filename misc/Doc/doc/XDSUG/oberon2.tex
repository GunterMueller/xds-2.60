\chapter{\XDS{} Oberon-2}\label{o2}
\index{Oberon-2}

This chapter includes the details of the \ot{} language which are
specific for this implementation.
In the standard mode\footnote{When the options \OERef{O2EXTENSIONS}
and \OERef{O2NUMEXT} are OFF.} \xds{} \ot{} is fully compatible with ETH compilers (See
{\em The Oberon-2 Report}). The last changes        % !!! extern ref
to the language are described in \ref{o2:changes}.

To provide a smooth path from \mt{} to \ot{} \XDS{} allows
all \mt{} data types to be used in \ot{} modules (See \ref{o2:usingm2}).

Several language extensions are implemented in the language according to
{\em The Oakwood Guidelines for the \ot{} Compiler Developers}\footnote{These
guidelines have been produced by a group of
\ot{} compiler developers, including ETH developers,
after a meeting at the Oakwood Hotel in Croydon, UK in June 1993.}
(See \ref{o2:oakext}). Other language extensions are described in
\ref{o2:ext}. As \xds{} is a truly multi-lingual
system, special features were introduced to provide interfacing to
foreign languages (See Chapter \ref{multilang}).

\section{The Oberon environment}\label{o2:env}
\index{Oberon environment}

The \ot{} language was originally designed for use in an
environment that provides {\em command activation, garbage
collection,} \index{garbage collection} and {\em dynamic loading}
of the modules. Not being a part of the language, these features still
contribute to the power of \ot{}.

The garbage collector and command activation are implemented in
the Oberon Run-Time Support \index{Oberon run-time support} and
can be used in any program. The dynamic loader is not
provided in the current release. See
\ref{rts:oberonRTS} for further information.

\subsection{Program structure}\label{o2:env:main}

In an \ot{} environment, any declared parameterless procedure
can be considered as a main procedure and can be called by its
name (a qualified identifier of the form
\verb|ModuleName.ProcName|).

Due to the nature of \xds{}, and its freedom from the Oberon
system, a different approach had to be found to declare
the `top level' or program modules.

The module which contains the top level of your program must
be compiled it with the \OERef{MAIN} option set.
This will generate an entry point to your program.
Only one module per program shall be compiled
with the option set. It is recommended to set it in the module
header:

\begin{verbatim}
    <*+ MAIN *>
    MODULE hello;

    IMPORT InOut;

    BEGIN
      InOut.WriteString ("Hello World!");
      InOut.WriteLn;
    END hello.
\end{verbatim}

\subsection{Creating a definition}\label{o2:env:makedef}
\index{Oberon-2!definition}
\index{definition for Oberon-2 module}

\xds{} provides two different ways to create a definition
for an \ot{} module:
\begin{itemize}
\item
        the \See{{\bf BROWSE} operation mode}{}{xc:modes:browse}
        creates a definition module from a symbol file
\item
        the \OERef{MAKEDEF} option forces the \ot{}
        compiler to generate a (pseudo) definition module
        after successful compilation of an \ot{} module.
\end{itemize}

The \OERef{MAKEDEF} option provides additional services:
the compiler will preserve the so-called {\em exported}
comments (i.e. comments which start with `(**')
if the \OERef{XCOMMENTS} option is ON.

The generated pseudo-definition module contains all exported declarations
in the order of their appearance in the source text.
All exported comments are placed at the appropriate positions.

A definition can be generated in three
{\em styles}.\index{browser style}
The \OERef{BSTYLE} equation can be used to choose one of the styles:
{\bf DEF} (default), {\bf DOC} or {\bf MOD}.

\begin{description}
\item[The DEF style] \mbox{}

This produces an ETH-style definition module.
All {\em type-bound procedures} ({\em methods})
and relative comments are shown as parts of
the corresponding record types.

This is the only style for which the
\OERef{BSREDEFINE} and \OERef{BSCLOSURE}
options are applicable.

\item[The DOC style] \mbox{}

This produces a pseudo-definition module in which methods are shown as
parts of the appropriate record types (ignoring comments) and at the
positions at which they occur in the source text.

\item[The MOD style] \mbox{}

This attempts to produce a file which can be compiled as an \ot{}
module after slight modification (i.e. the file will contain
"\verb'END procname'", etc.)
\end{description}

\section{Last changes to the language}\label{o2:changes}

\subsection{ASSERT}\label{o2:ASSERT}\index{ASSERT (Oberon-2)}
\index{Oberon-2!ASSERT}

The procedure \verb'ASSERT' checks its boolean parameter and terminates
the program if it is not \verb'TRUE'. The second optional parameter
denotes a {\em task termination code}. If omitted, a standard value
is assumed.

\verb'    PROCEDURE ASSERT(cond: BOOLEAN [; code: INTEGER]);'

A call \verb|ASSERT(expr,code)| is equivalent to

\verb'    IF NOT expr THEN HALT(code) END;'

\subsection{Underscores in identifiers}
\index{Oberon-2!identifiers}

According to the {\em Oakwood Guidelines} an underscore ("\verb'_'") may
be used in identifiers (as a letter).

\verb'    ident = ( letter | "_" ) { letter | digit | "_" }'

We recommend to use underscores with care, as it may cause
problems with software portability to other compilers. This
feature may be important for interfacing to foreign languages
(See Chapter \ref{multilang}).

\subsection{Source code directives}\label{o2:pragmas}

Source code directives (or pragmas) are used to set compilation
options in the source text and to select specific pieces of the source
text to be compiled (conditional compilation).  According to the {\em
Oakwood Guidelines} all directives are contained in ISO \mt{} style
pseudo comments using angled brackets \verb|<* ... *>|.

The additional language constructs should not be considered to be part
of the \ot{} language. They define a separate compiler control
language that coexist with \ot{}. The option \OERef{O2ISOPRAGMA} allows
pragmas to be used.

The syntax of the directives is the same for \mt{} and
\ot{}. See \ref{m2:pragmas} for further details.

\section{Oakwood numeric extensions}\label{o2:oakext}
\index{Oakwood Extensions}
\index{Oberon-2!numeric extensions}

\xds{} \ot{} supports two extensions which are of importance for
scientific programming, namely
\begin{itemize}
\item complex numbers
\item in-line exponentiation operator
\end{itemize}
The \OERef{O2NUMEXT} option should be set to use these
extensions.

\subsection{Complex numbers}\label{o2:oakext:complex}
\index{Oberon-2!complex numbers}

\onumextonly{}

Two additional types are included in the type hierarchy
if the \OERef{O2NUMEXT} option is set:
\begin{center}
\begin{tabular}{lcl}
\tt COMPLEX     & defined as & \tt (REAL,REAL)          \\
\tt LONGCOMPLEX & defined as & \tt (LONGREAL,LONGREAL)  \\
\end{tabular}
\end{center}
All numeric types form a (partial) hierarchy
$$
\mbox{\em whole types} \subset
\mbox{REAL}
\subseteq
\begin{array}{l}
\mbox{COMPLEX}  \\
\mbox{LONGREAL} \\
\end{array}
\subseteq
\mbox{LONGCOMPLEX}
$$
A common mathematical notation is used for complex number
literals:

\verb'    number = integer | real | complex'\\
\verb'    complex = real "i"'

A literal of the form \verb'5.0i' denotes a complex number with real
part equal to zero and an imaginary part equal to $5.0$. Complex
constants with a non-zero real part can be described using
arithmetic operators.

\begin{verbatim}
    CONST
      i = 1.i;
      x = 1. + 1.i;
\end{verbatim}
For the declarations
\begin{verbatim}
    VAR
      c: COMPLEX;
      l: LONGCOMPLEX;
      r: REAL;
      x: INTEGER;
\end{verbatim}
the following statements are valid:
\begin{verbatim}
      c:=i+r;
      l:=c;
      l:=c*r;
      l:=l*c;
\end{verbatim}

New conversion functions {\tt RE} and {\tt IM} can be used to
obtain a real or imaginary part of a value of a complex type. Both
functions have one parameter. If the parameter is of the \verb'COMPLEX'
type, both functions return a \verb'REAL' value; if the parameter is of
the \verb'LONGCOMPLEX' type, functions return a \verb'LONGREAL' value;
otherwise the parameter should be a complex constant and functions return
a real constant.

A complex value can be formed by applying the standard function
{\tt CMPLX} to two reals. If both {\tt CMPLX} arguments are real
constants, the result is a complex constant.

\verb'    CONST i = CMPLX(0.0,1.0);'

If both expressions are of the \verb'REAL' type, the function
returns a \verb'COMPLEX' value, otherwise it returns a
\verb'LONGCOMPLEX' value.

\subsection{In-line exponentiation}
\index{Oberon-2!in-line exponentiation}

\onumextonly{}

The exponentiation operator {\tt **} provides a convenient
notation for arithmetic expressions, which does not involve function
calls. It is an arithmetic operator which has a higher precedence
than multiplication operators.
\begin{verbatim}
    Term     =  Exponent { MulOp Exponent }.
    Exponent =  Factor { "**" Factor }.
\end{verbatim}
{\bf Note:} the operator is right-associated:
$$
  a**b**c \mbox{ is evaluated as } a**(b**c)
$$

The left operand of the exponentiation ({\tt a**b}) should be any
numeric value (including complex), while the right operand should
be of a real or integer type. The result type does not depend of
the type of right operand and is defined by the table:
\begin{center}
\begin{tabular}{ll}
\bf Left operand type & \bf Result type \\ \hline
an integer type     & \tt REAL \\
\tt REAL            & \tt REAL \\
\tt LONGREAL        & \tt LONGREAL \\
\tt COMPLEX         & \tt COMPLEX \\
\tt LONGCOMPLEX     & \tt LONGCOMPLEX \\
\end{tabular}
\end{center}

\section{Using Modula-2 features}\label{o2:usingm2}
\index{Oberon-2!using Modula-2}

All \mt{} types and corresponding operations can be used in \ot{},
including enumeration types, range types, records with variant parts,
sets, etc.

\paragraph{Important Notes:}
\begin{itemize}
\item It is not allowed to declare \mt{} types in an \ot{} module.
\item A module using \mt{} features is likely to be non-portable to other compilers.
\end{itemize}

\Example
\begin{verbatim}
    (*MODULA-2*) DEFINITION MODULE UsefulTypes;

    TYPE
      TranslationTable = ARRAY CHAR OF CHAR;
      Color  = (red,green,blue);
      Colors = SET OF Color;

    END UsefulTypes.

    (*OBERON-2*) MODULE UsingM2;

    IMPORT UsefulTypes;

    TYPE
      TranslationTable* = UsefulTypes.TranslationTable;

    VAR colors*: UsefulTypes.Color;

    BEGIN
      colors:=UsefulTypes.Colors{UsefulTypes.red};
    END UsingM2.
\end{verbatim}

\section{Language extensions}\label{o2:ext}
\index{Oberon-2!language extensions}

{\bf Warning:}  Using extensions may cause problems with the software
portability to other compilers.

In the standard mode, the \xds{} \ot{} compiler is fully
compatible with ETH compilers (See also \ref{o2:changes}).
The \OERef{O2EXTENSIONS} option enables some language extensions.
The main purposes of language extensions are
\begin{itemize}
\item to improve interfacing to other languages
     (See Chapter \ref{multilang}).
\item to provide backward compatibility with the previous versions
of \xds{}.
\end{itemize}

\Seealso
\begin{itemize}
\item         Source language directives (\ref{o2:pragmas})
\item         Oakwood numeric extensions (\ref{o2:oakext}).
\end{itemize}

\subsection{Comments}
\index{Oberon-2!comments}

\oextonly

As well as "\verb|(**)|", there is another valid format for  comments  in
source texts. The portion of a line from "\verb|--|"
to the end of line is considered a comment.

\verb'    VAR j: INTEGER; -- this is a comment'

\subsection{String concatenation}
\index{string concatenation}
\index{Oberon-2!string concatenation}

\oextonly

The symbol "+" can be used for constant string and characters
concatenation. See \ref{m2:ISO:strings} for more details.

\subsection{VAL function}
\index{Oberon-2!VAL}\index{VAL (O2)}

\oextonly

The function \verb'VAL' can be used to obtain a value of the specified
scalar type from an expression of a scalar type.
See \ref{m2:ISO:conv} for more details.

\verb'    PROCEDURE VAL(Type; expr: ScalarType): Type;'

The function can be applied to any scalar types,
including system fixed size types (See \ref{o2:fixed:size}).

\subsection{Read-only parameters}
\index{read-only parameters}
\index{Oberon-2!read-only parameters}

\oextonly

In a formal parameter section, the symbol \verb|"-"|
may appear after a name of a value parameter.
That parameter is called {\em read-only}; its value
can not be changed in the procedure's body.
Read-only parameters need not to be
copied before the procedure activation;
this enables procedures with structured parameters
to be more effective.
Read-only parameters can not be used in a procedure type declaration.

We recommend to use read-only parameters with care. The compiler
does not check that the read-only parameter is not modified via
another parameter or a global variable.

\Example
\begin{verbatim}
    PROCEDURE Foo(VAR dest: ARRAY OF CHAR;
                   source-: ARRAY OF CHAR);
    BEGIN
      dest[0]:='a';
      dest[1]:=source[0];
    END Foo;
\end{verbatim}

The call \verb|Foo(x,x)| would produce a wrong result, because
the first statement changes the value of \verb|source[0]|
(\verb|source| is not copied and points to the same location
as \verb|dest|).

\subsection{Variable number of parameters}
\index{SEQ parameters}
\index{Oberon-2!SEQ parameters}

\oextonly

Everything contained in the section \ref{m2:SEQ:param}
is applicable to \ot{}.

\subsection{Value constructors}
\index{value constructors}
\index{Oberon-2!value constructors}

\oextonly

Everything contained in the section \ref{m2:ISO:aggregates}
is applicable to \ot{}.

\section{The Oberon-2 module SYSTEM}
\label{o2:system}
\index{Oberon-2!module SYSTEM}
\index{system modules!SYSTEM (O2)}
\index{SYSTEM@{\bf SYSTEM}}

Low-level facilities are provided by the module \verb'SYSTEM'. This
module does not exist in the same sense as other library modules; it is
hard-coded into the compiler itself. However, to use the provided
facilities, it must be imported in the usual way.

Some procedures in the module \verb'SYSTEM' are generic procedures
that cannot be explicitly declared, i.e. they apply to classes of operand
types.

\xds{} \ot{} compiler implements all system features described in
{\em The \ot{} Report} (except \verb'GETREG', \verb'PUTREG', and \verb'CC') and       % !!! extern ref?
allows one to access all features, described in the \mt{} International
Standard \mt{} (See \ref{m2:ISO:system}). In this section we
describe only features specific for this implementation.
\index{SYSTEM!GETREG}
\index{SYSTEM!PUTREG}

\subsection{Compatibility with BYTE}
\index{Oberon-2!SYSTEM.BYTE}

Expressions of types \verb'CHAR', \verb'BOOLEAN',
\verb'SHORTINT' and \verb'SYSTEM.CARD8' can
be assigned to variables of type \verb'BYTE' or passed as actual
parameters to formal parameters of type \verb'BYTE'.

If a formal procedure parameter has type \verb'ARRAY OF BYTE', then the
corresponding actual parameter may be of any type, except numeric
literals.

\subsection{Whole system types}\label{o2:fixed:size}

Module \verb'SYSTEM' contains the signed types \verb'INT8', \verb'INT16',
\verb'INT32', and unsigned types \verb'CARD8', \verb'CARD16', \verb'CARD32',
which are guaranteed to contain exactly 8, 16, or 32 bits respectively.
These types were introduced to simplify consstructing the interfaces
to foreign libraries (See Chapter \ref{multilang}).
The basic types \verb'SHORTINT', \verb'INTEGER', \verb'LONGINT'
are synonyms of \verb'INT8', \verb'INT16', and \verb'INT32' respectively.

The unsigned types form a hierarchy
whereby larger types include (the values of) smaller types.
$$
\mbox{{\tt SYSTEM.CARD32}} \supseteq
\mbox{{\tt SYSTEM.CARD16}} \supseteq
\mbox{{\tt SYSTEM.CARD8}}
$$
The whole hierarchy of numeric types (See also \ref{o2:oakext:complex}):
$$
\mbox{{\tt LONGREAL}} \supseteq \mbox{{\tt REAL}} \supseteq
\left \{
  \begin{array}{l}
     \mbox{signed types}        \\
     \mbox{unsigned types}      \\
  \end{array}
\right .
$$

\subsection{NEW and DISPOSE}\label{o2:system:new}
\index{SYSTEM!NEW}
\index{SYSTEM!DISPOSE}
\index{NEW (SYSTEM, O2)}
\index{DISPOSE (SYSTEM, O2)}

The procedure {\tt SYSTEM.NEW} can be used to allocate the system memory,
i.e. memory which is not the subject of garbage collection.
{\tt SYSTEM.NEW} is a generic procedure, which is applied to pointer types
and can be used in several ways, depending on pointer's base type.

\verb'    PROCEDURE NEW(VAR p: AnyPointer [; x0,..xn: integer]);'

Let type \verb|P| be defined as \mbox{{\tt POINTER TO} $T$} and
$p$ is of type \verb'P'.
\begin{center}
\begin{tabular}{lp{9cm}}
\tt NEW($p$)       &  $T$ is a record or fixed length array type.
                  The procedure allocates a storage block
                  of {\tt SIZE($T$)} bytes
                  and assigns its address to $p$.          \\

\tt NEW($p$,$n$)   &  $T$ is a record or fixed length array type.
                  The procedure allocates a storage block of $n$ bytes
                  and assigns its address to $p$.           \\
\tt NEW($p$,$x_0$,..$x_{n-1}$) & $T$ is an $n$-dimensional open array.
                The procedure allocates an open array of lengths
                given by the expressions $x_0$,..$x_{n-1}$     \\
\end{tabular}
\end{center}

The procedure {\tt SYSTEM.DISPOSE} can be used to free a block
previously allocated by a call to {\tt SYSTEM.NEW}.
It does {\em not} immediately deallocate the block, but marks it as a free
block. The block will be deallocated by the next call of the garbage
collector.

\verb'    PROCEDURE DISPOSE(VAR p: AnyPointer; [size: integer]);'

\begin{center}
\begin{tabular}{lp{9cm}}
\tt DISPOSE($p$)     & $T$ is a record or array type.
                       The procedure deallocates the storage block $p$ points to.   \\
\tt DISPOSE($p$,$n$) & $T$ is a record or fixed length array type.
                       The procedure deallocates the storage block of $n$ bytes $p$ points to. \\
\end{tabular}
\end{center}

\subsection{M2ADR}\label{o2:system:m2adr}
\index{SYSTEM!M2ADR}
\index{M2ADR (SYSTEM, O2)}

In \ot{}, the \verb'SYSTEM.ADR' procedure returns \verb'LONGINT',   % !!! ref
which is not always very convenient. The {\tt SYSTEM.M2ADR}
procedure behaves as Modula-2 {\tt SYSTEM.ADR},   % !!! ref
returning {\tt SYSTEM.ADDRESS}:

\verb'    PROCEDURE M2ADR(VAR x: any type): ADDRESS;'



