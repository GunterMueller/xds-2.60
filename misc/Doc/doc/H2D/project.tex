\chapter{Project files}
\label{project}
\index{project files}

\section{Overview}

The most powerful and multi-purpose feature of H2D is the {\em project file},
which name may be specified at the command line after \verb'=p'.

The project file may be used:
\begin{itemize}
\item To translate a set of header files at once, using the same option
      settings:

\verb'-BACKEND=C'\\
\verb'-GENROVARS+'\\
\verb'!module <stdio.h>'\\
\verb'!module <stdlib.h>'

\item To map calling/naming convention qualifiers used by a particular C
      compiler to those supported by H2D (see \ref{rules:qualifiers}),
      or to make them ignorable:

\verb'!header <*.h>            /* define for all headers */'\\
\verb'#define _System syscall'\\
\verb'#define far'\\
\verb'!end'

\item To define macros which are predefined by a particular
      C compiler or are used in conditional compilation preprocessor
      directives (see \ref{rules:pp:conditional}):

\verb'!header <*.h>'\\
\verb'#define __WATCOM_C__'\\
\verb'#define INCL_DOS'\\
\verb'!end'

\item To declare type synonyms in order to prevent H2D from automatic
      type declarations (see \ref{rules:functions} and \ref{rules:types}):

\verb'!header <*.h>'\\
\verb'typedef char *String     /* no more PtrChar */'\\                   % !!! is this true?
\verb'!end'

\item To collect non-standard preprocessor directives in order to keep
      original header files intact (see \ref{project:directives}):

\verb'!header <string.h>'\\
\verb'#variant strlen(0) : ARRAY'\\
\verb'!end'

\end{itemize}

\subsection*{Example}

The header files {\tt a.h} and {\tt m1.h}:

{\ifonline\else\small\fi
\begin{verbatim}
/* a.h */
  #include "m1.h"
  #define constant1 0x11u
  __PASCAL function3( float * arg0, unsigned long arg1 );
/* end a.h */
\end{verbatim}

\begin{verbatim}
/* m1.h */
  #define constant2 0x111u
  __PASCAL function1( float * arg0, float * arg1 );
  function2( unsigned long arg0, unsigned long arg1 );
/* end m1.h */
\end{verbatim}

with project file {\tt p.h2d}:

\begin{verbatim}
!header "*.h"
  #define __PASCAL pascal
!end
!header "a.h"
  #merge "m1.h"
!end
!header "m1.h"
  #variant function1 (1) : VAR
  #variant constant2 : BITSET
  #variant function2 (1) : BITSET
!end
!module "a.h"
\end{verbatim}

are translated to

\begin{verbatim}
(* ************************ *)
(*           m1.h           *)
(* ************************ *)
CONST
  constant2 = {0, 4, 8};
<*- GENTYPEDEF *>
TYPE
  PtrFloat = POINTER TO REAL;
PROCEDURE ["StdCall"] function1 ( arg0: PtrFloat;
                           VAR arg1: REAL ): SYSTEM.int;
PROCEDURE function2 ( arg0: LONGCARD; arg1: BITSET ): SYSTEM.int;
(* *********************** *)
(*           a.h           *)
(* *********************** *)
CONST
  constant1 = 111H;
PROCEDURE ["StdCall"] function3 ( arg0: PtrFloat;
                            arg1: LONGCARD ): SYSTEM.int;
\end{verbatim}
} % end \small

\section{Project file contents}
\label{project:directives}

A project file may contain options settings and directives.
Option settings in a project file override settings
in the configuration file.

H2D recognizes the following directives in project files:
{\tt !header}, {\tt !module}, and {\tt !name}, which are
described is the following sections.

\subsection{!header}
\label{project:header}

\begin{verbatim}
!header ( '<' Pattern '>' | '"' Pattern '"' )
  Prologue
[
!footer
  Epilogue
]
!end
\end{verbatim}

\verb'Pattern' is a regular expression (see \ref{config:red})
representing a set of file names.
{\tt Prologue} and {\tt Epilogue} are arbitrary sequences
of C language tokens. {\tt Prologue} is inserted at the beginning
of any header file which name matches \verb'Pattern'; {\tt Epilogue} is
appended to its end. {\tt !footer} and {\tt Epilogue} may be omitted.

If a header file name matches \verb'Pattern' in more than one
\verb'!header' directive, their \verb'Prologue' and \verb'Epilogue'
sections are merged.

{\tt Prologue} usually contains:
\begin{itemize}
\item {\tt \#merge} and {\tt \#variant} directives
\item Predefined macros of a particular C compiler
\item Type synonym declarations to prevent automatic type names generation % !!! is it true
\end{itemize}

{\bf Note:} If there are \verb'#include' directives in either
\verb'Prolodue' or \verb'Epilogue' ensure that names of the
included files do not match \verb'Pattern', to avoid recursive inclusion:

\begin{verbatim}
!header <*.h&^mytypes.h>
#include <mytypes.h>
!end
\end{verbatim}

\subsection{!module}
\label{project:module}

\begin{verbatim}
!module ( <file_name> | "file_name" )
\end{verbatim}

The \verb'!module' directive is used to specify header files which
are to be translated when H2D processes the project file.

Translating more than one header at once has one more advantage.
A header file name may occur multiple times in {\tt \#include} directives.
H2D keeps information about each translated header in memory, and if an
already transtaled header file is encountered, it is not processed again.
{\bf Note:} In this case H2D requires more memory.

\subsubsection*{Example}

\begin{verbatim}
!module <ctype.h>
!module <math.h>
!module <stdio.h>
!module <stdlib.h>
!module <string.h>
\end{verbatim}

\subsection{!name}
\label{project:name}

\begin{verbatim}
!name ( '<' file_name '>' | '"' file_name '"' ) identifier
\end{verbatim}

H2D replaces {\tt file\_name} with {\tt identifier} when generating
module names. This may be useful when {\tt file\_name} contains special
characters (e.g. {\tt my-header.h}), or when there are headers with
equal names in different directories. See also the description
of the \OERef{GENLONGNAMES} option.

\subsubsection*{Example}

\verb'!name <errno.h>     errno'\\
\verb'!name <sys\errno.h> syserrno'


