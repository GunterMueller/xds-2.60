\newif\ifonline

\begin{document}
\title{XDS Family of Products}
\begin{center}
{\LARGE \bf eXtensible Development System\\}

{\Huge \bf XDS Family of Products\\}

\ref[Key features]{features}

\ref[Development activities]{development}

\ref[Credits]{credits}

\ref[Contact information]{contact} 

\\\\Copyright \copyright 1999-2011 Excelsior, LLC.
\end{center}

\chapter{Key features}
\label{features}

\begin{itemize}
\item \ref{feat:sources}
\item \ref{feat:targets}
\item \ref{feat:platforms}
\item \ref{feat:portability}
\item \ref{feat:standard}
\item \ref{feat:optimization}
\item \ref{feat:speed}
\item \ref{feat:reliability}
\item \ref{feat:OOP}
\item \ref{feat:C}
\item \ref{feat:API}
\item \ref{feat:IDE}
\item \ref{feat:debug}
\end{itemize}

\section{Two source languages}
\label{feat:sources}

   XDS contains two well intergrated compilers - {\bf Modula-2} and 
   {\bf Oberon-2}. It is possible to mix these two languages in a 
   single project. The library set may be shared as well.

\section{Multiple targets}
\label{feat:targets}
\index{Native XDS}
\index{XDS-C}

   XDS compilers generate either {\bf ANSI C text} ({\em XDS-C}) or 
   {\bf native code} for the target platform ({\em Native XDS}).
   XDS-C is a "via C" cross-compiler, the output of which is
   supposed to be supplied uindchnged to the respective C compiler.
   Although it produces readable text, it does {\bf not}
   completely preserve the source program's
   structure, identifiers and comments, and therefore may {\bf not}
   be used as a Modula-2/Oberon-2 to C converter.

   A true Modula-2 to C++ converter is avaliable from Excelsior
   on a semi-custom basis. Contact the Sales Dept. at
   {\bf sales@excelsior-usa.com} for details.

\section{Multiple platforms}
\label{feat:platforms}

   {\em Native XDS} is currently available for the Intel x86 platform
   (Windows and Linux). Support for other
   platforms may be implemented on a custom basis.
   
   {\em XDS-C} is also available for Windows and Linux, but
   allows you to target virtually any platform for 
   which an ANSI C compiler is available.

\section{Portability}
\label{feat:portability}

   All XDS implementations share the same front-end and library
   set. Thus, for programs which do not use a specific operating system API
   and do not rely on a hardware, {\bf only recompilation} is necessary 
   to port to another platform,

\section{Standard compliance}
\label{feat:standard}

   XDS implementation of Modula-2 is {\bf 100\% ISO-compliant};
   full set of ISO Modula-2 standard libraries is provided.

   XDS Oberon-2 conforms to the Niklaus Wirth's {\em Oberon-2 Report}.

   On the other hand, a number of language extensions is
   provided for conveninice. These extensions, however, have to be
   explicitly enabled using a compiler option.

\section{Highly optimized code}
\label{feat:optimization}

   Native XDS-x86 compilers produce {\bf highly optimized}, {\bf 32-bit},
   professional quality code. The resulting program perfomance competes
   with results shown by industry standard C compilers. The 
   {\em instruction scheduling} mechanism for Pentium processors
   results in a further 5-15\% perfomance gain.

\section{Good compilation speed}
\label{feat:speed}

   XDS possess a quite reasonable code quality/compilation speed ratio.
   The origin of this fact is that XDS compilers are used to build themselves.

\section{Reliability}
\label{feat:reliability}

   A set of options which affect the code generation (optimizations etc)
   was intentionally made very small in XDS. Therefore, the set of their
   combinations is apparently not large and it does not take too much
   time to {\bf test the compiler in all modes}. With many C/C++ compilers, 
   this is almost impossible.

\section{Object-oriented programming}
\label{feat:OOP}

   Oberon-2 is an {\bf object-oriented programming language} which is much easier
   to teach, learn, and use than C++. With Oberon-2, it is possible to
   concentrate on the principles of OOP, not on the way it is done using a
   particular language.

   Since Oberon-2 is a descendant of Modula-2, these languages are very similar,
   so moving from one to another can be done smoothly. And with XDS, it is
   {\bf extremely smooth}:

   \begin{itemize}
   \item There is no need to learn a new programming environment, what means no
         time loss.
   \item Programs already developed in Modula-2 may be completely {\bf re-used}
         without any modifications, just as in the case of moving from C to C++.
   \end{itemize}

\section{Interface to C}
\label{feat:C}

   Some of the most annoying things about Modula-2 are absence,
   imcompleteness, unportability and low quality of libraries.
   XDS, with its {\bf foreign languages support}, gives an opportunity
   to use the whole set of free, public domain, shareware, and
   commercial libraries from the C world and access the 
   \ref[Host OS API]{feat:API}.

   C header files can be translated into Modula-2 definition 
   modules by means of the supplied H2D utility.

\section{Host OS API support}
\label{feat:API}

   XDS packages contain {\bf definition modules for the host operating
   system base API (Win32, or POSIX and X Window/Motif},
   and therefore may be used to implement large 
   projects which use GUI, multi-threading, inter-process communication,
   and other features provided in these APIs.

\section{Integrated Development Environment}
\label{feat:IDE}

   XDS for Windows comes with an {\bf IDE},
   which greatly simplifies its usage. Although XDS is a professional 
   system with a wide set of configuration files and options, 
   the IDE hides all these details from a novice.

   The IDE may also be adopted for a third-party compiler or used as a
   plain text editor.

\section{Debug and analysis tools}
\label{feat:debug}

Native code XDS compiler for Windows comes with the
following additional tools:

\begin{itemize}
\item A full source level symbolic debugger that may be used in two modes,
      dialog and batch. The dialog mode is interactive, whereas in the 
      batch mode the debugger may automatically execute a debugging scenario 
      (or a number of scenarios) described in special control files, 
      and log the scenario execuion results. 
      This allows the debugger to be used to automate testing. 
\item A set of utilities that may help you to better understand and 
      improve the run-time performance of your program. It emphasises 
      the pieces of code which consume most of the CPU time, and, 
      hence, are the first candidates for redesign. 
\end{itemize}

\section{TopSpeed Compatibility Pak}

TopSpeed Compatiblity Pak is an add-on that may be purchased
separately. It contains a compiler with TopSpeed-like extensions and
a set of TopSpeed-like library modules.

\chapter{Development}
\label{development}

XDS development is frozen due to low demand for commercial Modula-2/Oberon-2
compilers. Technical support and maintenance releases shall continue.

\chapter{Credits}
\label{credits}

The following people helped us a lot by sending their comments, bug reports
and suggestions while beta testing, evaluating the pre-release versions or
working with our previous products:

\begin{center}
\\
Marius Ackerman \\
Alexander Antonenko \\
Werner Fouche   \\
Stefan Gruendel \\
Jim Hawkins     \\
Rob Limburg     \\
John McMonagle  \\
Stefan Metzeler \\
Peter Moylan    \\
Mike Nice       \\
Graham Stark    \\
Bernhard Treutwein 
\end{center}

Special thanks to Guenter Dotzel of Modulaware.

\chapter{How to contact us}
\label{contact}

   Send your questions and comments to:

   {\bf support@excelsior-usa.com}

   For latest information about new products, releases, updates,
   and fixes, please visit our Web page at:

   {\bf http://www.excelsior-usa.com}
  
\end{document}

