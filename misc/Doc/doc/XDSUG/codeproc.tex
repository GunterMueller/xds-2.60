\section{Code Procedures}\label{lowlevel:codeproc}
\index{Code Procedures}

Code procedures are used to implement low-level operations
and provide access to hardware.
\begin{verbatim}
CodeProcedure   = PROCEDURE "-" ident
                  [ FormalParameters] [ExpList].
\end{verbatim}

  The expression list forming the code procedure body
is restricted as follows:
\begin{itemize}
\item
this list must be non-empty and all its elements must be constant
expressions.

\item
the first expression in the list must be a constant string expression
(that is, constant expression of the type {\tt ARRAY [0..N] OF CHAR}).
This string contains the code procedure calling protocol description
(see below) in the textual form.

\item
all other expressions in the list, if any, must be numbers in the range
$[0..255]$, representing one byte of native 80386/80486 code. The code
procedure body is formed by concatenation of all these bytes into one
linear sequence.  \end{itemize}

The code procedure call is executed as follows:
\begin{enumerate}
\item
   actual parameters are evaluated according to general language
   semantics. Note that the order of parameter evaluation is undefined and
   is left to the compiler;
\item
   compiler-produced code is executed if necessary, which guarantees all calling
   protocol conditions to hold (i.e. all values are stored in the specified
   registers or memory locations, all work registers are free etc.);
\item
   code sequence of procedure body is executed. This can be done in one
   of two possible ways:
   \begin{itemize}
   \item
     if {\bf O} flag is set in compiler options (see \ref{pragma})
     then compiler just inserts the procedure body as an in-line
    code sequence in the place of the call;
   \item
     if {\bf O} flag is not set, then the procedure body is stored in the code segment
     only once and the compiler generates a call sequence in the place of the call.
     This mode reduces program code size if code procedure is called
     more than once, but the call is executed more slowly.
   \end{itemize}
\item
   compiler-produced code is executed if necessary, which
   restores the context of the call.
\end{enumerate}

\subsubsection{Calling Protocol Syntax}

\begin{verbatim}
 1. register386 = "EAX" | "EBX" | "ECX" | "EDX"
                | "EBP" | "ESI" | "EDI".
 2. register387 = "ST(" octalDigit ")".
 3. Register = register386 | register387.
 4. RegisterDiap = "ST(" octalDigit "-" octalDigit ")".
 5. RegisterPair = "(" register386 "," register386 ")".
 6. MemoryLocation = "[" integer "]".
 7. Location = Register | RegisterPair
             | MemoryLocation.
 8. ParameterName = ident.
 9. ParameterSpecifier = ParameterName ":" Location.
10. ResultSpecifier = "RETURN" ":" Register.
11. UseElement = Register | RegisterDiap
12. UseSpecifier = "USE" ":"
               "(" UseElement { "," UseElement } ")".
13. Protocol = { ParameterSpecifier } [ UseSpecifier ]
               [ ResultSpecifier ]
\end{verbatim}

Code procedure calling protocol description consists of three parts:
Parameter, Register and Result specification.

\subsection{Parameters Specification}

This part describes how a procedures parameters are passed.

{\bf Note:} each parameter must be specified explicitly, no defaults are
assumed by the compiler.

In general, parameters may be passed:
\begin{description}
\item (a) in one of 80386 general purpose registers
\item (b) in one of 80387 coprocessor stack elements
\item (c) in a pair of 80386 registers
\item (d) in memory
\end{description}

   Letters (a)-(d) will be used below to denote the corresponding parameter
   passing method.

   80386 registers are denoted by their standard names
   (note that the ESP register can not be used).

   80387 stack elements are denoted by "ST(i)", where $0\leq i\leq 7$ is
   a stack element number (see 80387 Programming Reference).

   Memory location is denoted by "[ OFFS ]", where OFFS is a positive
   integer constant which must be greater than or equal to 8.
   Parameter's address in memory is then ESP+OFFS.
   Parameters can not overlap in memory - this situation is
   detected by the compiler and reported as an error.

 Parameters are passed in a way depending on their type, kind
 and passing method chosen (See \ref{lowlevel:datarep}
 for details about value representations).
 `VAR' and `VAL' below specify whether a parameter is passed
 by reference or by value respectively.

\subsubsection{Scalar types with values represented by 1 byte}

\begin{tabular}{|c|c|p{8cm}|} \hline
\multicolumn{2}{|c|}{Method} & Action \\ \hline
(a) & VAR & 32-bit address of memory location containing parameter
            value is stored in any of seven 80386 registers.        \\
    & VAL &  parameter value is stored in the least significant
            byte of any of four 80386 registers: EAX, EBX, ECX, EDX.
            Bits 8-31 of register are undefined.                    \\ \hline
(b) && not allowed                                                  \\ \hline
(c) && not allowed                                                  \\ \hline
(d) & VAR & 32-bit address of memory location containing parameter
             value is stored in specified memory location.          \\
    & VAL & 1-byte parameter value is stored in specified memory
            location.                                               \\ \hline
\end{tabular}

\subsubsection{Scalar types with values represented by 2 bytes}

\begin{tabular}{|c|c|p{8cm}|}  \hline
\multicolumn{2}{|c|}{Method} & Action \\ \hline
(a) & VAR & 32-bit address of memory location containing parameter
            value is stored in any of seven 80386 registers.        \\
    & VAL & parameter value is stored in the least significant
            2 bytes of any of seven 80386 registers.
            Bits 16-31 of register are undefined.                   \\ \hline
(b) && not allowed                                                  \\ \hline
(c) && not allowed                                                  \\ \hline
(d) & VAR & 32-bit address of memory location containing parameter
            value is stored in specified memory location.           \\
    & VAL & 2-byte parameter value is stored in specified memory
            location.                                               \\ \hline
\end{tabular}

\subsubsection{Scalar types with values represented by 4 bytes}

\begin{tabular}{|c|c|p{8cm}|} \hline
\multicolumn{2}{|c|}{Method} & Action \\ \hline
(a) & VAR & 32-bit address of memory location containing parameter
            value is stored in any of seven 80386 registers.       \\
    & VAL & parameter value is stored in 4 bytes of any
            of seven 80386 registers.                              \\ \hline
(b) && not allowed                                                 \\ \hline
(c) && not allowed                                                 \\ \hline
(d) & VAR & 32-bit address of memory location containing parameter
         value is stored in specified memory location.             \\
    & VAL & 4-byte parameter value is stored in specified memory
            location.                                              \\ \hline
\end{tabular}

\subsubsection{REAL and LONGREAL}

\begin{tabular}{|c|c|p{8cm}|} \hline
\multicolumn{2}{|c|}{Method} & Action \\ \hline
(a) & VAR & 32-bit address of memory location containing parameter
            value is stored in any of seven 80386 registers.       \\
    & VAL & not allowed                                            \\ \hline
(b) & VAR &  not allowed                                           \\
    & VAL & value is stored in the specified 80387 stack element
            in 80-bit internal 80387 format                        \\ \hline
(c) & & not allowed                                                \\ \hline
(d) & VAR & 32-bit address of memory location containing parameter
         value is stored in specified memory location.             \\
    & VAL & 4-byte (8-byte resp.) parameter value is stored
         in specified memory location.                             \\ \hline
\end{tabular}

   The current code generator uses coprocessor registers in stack
 mode only. Because of this, the set of stack element numbers specified
 for parameter passing is limited to be one of:
\begin{verbatim}
   {}, {0}, {0..1}, {0..2} ... {0..7}
\end{verbatim}

   i.e. parameters to be passed in coprocessor registers are
 pushed onto the 80387 stack in the specified order.

\subsubsection{Records}

\begin{tabular}{|c|c|p{8cm}|} \hline
\multicolumn{2}{|c|}{Method} & Action \\ \hline
(a) &&
         32-bit address of the first (least significant) byte
         of the record is stored in any of seven 80386 registers. \\ \hline
(b) &&   not allowed                                              \\ \hline
(c) & VAR & allowed for Oberon-2 records only.
    In that case:
      32-bit address of the first (least significant) byte
      of the record is stored in the first pair element;
      32-bit address of Oberon-2 record type run-time descriptor
      is stored in the second pair element.
      Pair could consist of any two of seven 80386 registers,
      pair elements must be different.                             \\
    & VAL & not allowed                                            \\ \hline
(d) &&
         32-bit address of the first (least significant) byte
         of the record is stored in specified memory location.     \\ \hline
\end{tabular}

\subsubsection{Static arrays}

\begin{tabular}{|c|p{8.5cm}|} \hline
Method & Action \\ \hline
(a) &
         32-bit address of the first (least significant) byte
         of the array is stored in any of seven 80386 registers. \\ \hline
(b) & not allowed                                                \\ \hline
(c) & not allowed                                                \\ \hline
(d) &
         32-bit address of the first (least significant) byte
         of the array is stored in specified memory location.    \\ \hline
\end{tabular}

\subsection{Open arrays}

\begin{tabular}{|c|p{8.5cm}|} \hline
Method & Action \\ \hline
(a) &
         32-bit address of the first (least significant) byte
         of the array proper is stored in any of seven 80386 registers. \\ \hline
(b) & not allowed                                                \\ \hline
(c) & allowed for one-dimensional arrays only.
    In this case:
      32-bit address of the first (least significant) byte
      of the array is stored in the first pair element;
      32-bit signed integer value equal to
      (number of array elements - 1) is stored in the second
      pair element.
      Pair could consist of any two of seven 80386 registers,
      pair elements must be different.                           \\ \hline
(d) &
         open array descriptor (see \ref{lowlevel:opendesc})
         is stored in the specified memory location.             \\ \hline
\end{tabular}

\subsection{Register Usage Specification}

   This part defines the set of 80386/80387 registers to be used as
 working registers in the code procedure body. The compiler uses this set
 to correctly adjust the coprocessor stack and to save all intermediate
 values, which are possibly stored in 80386 registers by the code preceeding
 the call.

   80386 registers used for parameter passing are considered to be working
 registers also. They may or may not be specified in the USE part of the
 protocol description.

\subsubsection{Using 80387 stack}\label{lowlevel:stack387}

   Specification "ST(i-j)" is a shorter form of "ST(i),...,ST(j)".
 The set of stack element numbers, specified to be work stack elements,
 is limited to be one of:
\begin{verbatim}
   {}, {0}, {0..1}, {0..2} ... {0..7}
\end{verbatim}
 and is understood as follows:
\begin{tabbing}
   let \= $ \{0..N_{\mbox{par}}\}$ \= be the set of stack elements  \\
       \>                       \> used for parameters passing;  \\
\>        $\{0..N_{\mbox{use}}\}$   \> be the set of work stack elements \\
\end{tabbing}
It is guaranteed by the compiler that after all parameters
are pushed onto the coprocessor stack, there are at least $N$ free elements left
on the stack; that is, you can push $N$ more elements onto it (making no pop's)
without getting a STACK OVERFLOW exception, where $N$ is defined as:
$$ N = \left \{
        \begin{array}{ll}
           N_{\mbox{use}} - N_{\mbox{par}} & \mbox{ if } N_{\mbox{use}}\geq N_{\mbox{par}} \\
           0                         & \mbox{ otherwise}
        \end{array}
       \right .
$$

The stack must be cleared explicitly in the code procedure body.
The compiler expects that the stack state immediately after the procedure body
execution is that which would be formed by the stack in the state immediately
prior to the procedure body execution had the following operations
performed over it :

\begin{itemize}
\item   All parameters were popped off the stack
\item   The function result (if any) was pushed onto the stack
\end{itemize}

\subsection{Result Specification}

This part of the calling protocol description is necessary for code
functions only. It describes the register in which
the function result is returned.
The register content is interpreted in accordance with the type of result
(for value representations see \ref{lowlevel:datarep}).

\subsubsection{Values represented by 1 byte}

The result may be returned in any of four 80386 registers:
EAX, ECX, EDX, EBX. The compiler assumes the result to be the least
significant byte of the register.

\subsubsection{Values represented by 2 bytes}

      The result may be returned in any of six 80386 registers:
    EAX, ECX, EDX, EBX, ESI, EDI. The compiler assumes the result to be the least
    significant 2 bytes of the register.

\subsubsection{Values represented by 4 bytes (except REAL)}

      The result may be returned in any of six 80386 registers:
    EAX, ECX, EDX, EBX, ESI, EDI.

\subsubsection{Real values}

      The result may be returned on the top of the 80387 stack only. So the result
    specification should be "RETURN: ST(0)".

 No other type is allowed to be the type of function result.

   The register specified for the function result is considered to be a work
 register as well. It may be specified in the USE part of the protocol
 description, but this is not compulsory.

   For code functions returning {\tt REAL} or
{\tt LONGREAL} values,  (see \ref{lowlevel:stack387}),
at least one 80387 stack element is guaranteed to be free before
the function body execution, even if no stack elements are defined
as being used in the USE part of the protocol description.


\subsection{Code Procedure Examples}

All examples are formed as follows:
\begin{itemize}
\item code procedure heading
\item calling protocol description
\item assembler code in the comments\footnote{ignored by the compiler}
\item a sequence of code bytes
\item description of the way it works, which contains:
\begin{itemize}
        \item pre-action, generated by the compiler
        \item code procedure itself
        \item post-action, generated by the compiler
\end{itemize}
\end{itemize}

\subsubsection{Convert CHAR to BITSET}

\begin{verbatim}
PROCEDURE - MakeSet(s: CHAR): BITSET
  "s: EAX  RETURN: EAX",

(*begin
        movzx   eax,al
end*)

00FH,0B6H,0C0H;
\end{verbatim}

\begin{description}
\item[compiler] \mbox{}
   \begin{itemize}
     \item saves EAX if necessary
     \item places value of {\tt s} into AL
   \end{itemize}
\item[code procedure] \mbox{}
   \begin{itemize}
   \item extends AL to EAX with zeros
   \end{itemize}
\item[compiler] \mbox{}
   \begin{itemize}
   \item uses EAX as procedure call result
   \end{itemize}
\end{description}

\subsubsection{16-bit DIV/MOD}

\begin{verbatim}
PROCEDURE - Divide(val,div: SYSTEM.CARD16;
                   VAR res: SYSTEM.CARD16;
                   VAR rem: SYSTEM.CARD16);
  "val: EAX div: ECX res: ESI rem: EDI USE: (EDX)",

(*begin
        xor     dx,dx
        div     cx
        mov     [esi],ax
        mov     [edi],dx
end*)

066H,033H,0D2H,066H,0F7H,0F1H,
066H,089H,006H,066H,089H,017H;
\end{verbatim}

\begin{description}
\item[compiler] \mbox{}
   \begin{itemize}
     \item saves EAX, ECX, ESI, EDI, EDX if necessary
     \item puts value of {\tt val} into AX, value of {\tt div} into CX,
           address of {\tt res} into ESI, address of {\tt rem} into EDI
   \end{itemize}
\item[code procedure] \mbox{}
   \begin{itemize}
   \item performs the division and moves the result to {\tt res}
         and remainder to {\tt rem}
   \end{itemize}
\item[compiler] \mbox{}
   \begin{itemize}
   \item  restores context if necessary
   \end{itemize}
\end{description}

\subsubsection{String Copy 0}\label{StrCopy0}

\begin{verbatim}
PROCEDURE - StrCopy0(VAR d: ARRAY OF CHAR;
                         s: ARRAY OF CHAR)
  " d: (EDI,ECX) s: (ESI,EDX) ",

(*begin
        cmp     ecx,edx
        jl      l0
        mov     ecx,edx
l0:
        inc     ecx
        cld
        rep     movsb
end*)

03BH,0CAH,07CH,006H,090H,090H,090H,090H,
08BH,0CAH,041H,0FCH,0F3H,0A4H;
\end{verbatim}

\begin{description}
\item[compiler] \mbox{}
   \begin{itemize}
     \item saves ESI, EDI, ECX and EDX if necessary
     \item places address of {\tt d} in EDI, address of {\tt s} - in ESI,
           {\tt HIGH(d)} - in ECX, {\tt HIGH(s)} - in EDX
   \end{itemize}
\item[code procedure] \mbox{}
   \begin{itemize}
   \item moves {\tt MIN(HIGH(d),HIGH(s))+1} bytes from {\tt s} to {\tt d}
   \end{itemize}
\item[compiler] \mbox{}
   \begin{itemize}
   \item restores context if necessary
   \end{itemize}
\end{description}

\subsubsection{String Copy 1}

\begin{verbatim}
PROCEDURE - StrCopy1(VAR d: ARRAY OF CHAR;
                         s: ARRAY OF CHAR)
  " d: [8] s: [16] " ...
\end{verbatim}

The compiler will detect an error: parameters overlap in memory.


\subsubsection{String Copy 2}

\begin{verbatim}
PROCEDURE - StrCopy2(VAR d: ARRAY OF CHAR;
                         s: ARRAY OF CHAR)
  " d: [8] s: [20] USE: (ESI,EDI,ECX,EDX) " ,

(*begin
        mov     esi,[esp+20]
        mov     edx,[esp+28]
        mov     edi,[esp+8]
        mov     ecx,[esp+16]
        cmp     ecx,edx
        jl      l0
        mov     ecx,edx
l0:
        inc     ecx
        cld
        rep     movsb
end*)

08BH,074H,024H,014H,08BH,054H,
024H,01CH,08BH,07CH,024H,008H,
08BH,04CH,024H,010H,03BH,0CAH,
07CH,006H,090H,090H,090H,090H,
08BH,0CAH,041H,0FCH,0F3H,0A4H;
\end{verbatim}

\begin{description}
\item[compiler] \mbox{}
   \begin{itemize}
     \item  allocates enough place on the program stack
            and places {\tt d} and {\tt s}
            array descriptors on it with specified offsets from ESP
     \item saves ESI, EDI, ECX and EDX if necessary
   \end{itemize}
\item[code procedure] \mbox{}
   \begin{itemize}
   \item performs the same string copying as in example \ref{StrCopy0}
   \end{itemize}
\item[compiler] \mbox{}
   \begin{itemize}
   \item restores the program stack and registers
   \end{itemize}
\end{description}

\subsubsection{Copy Bytes}

\begin{verbatim}
PROCEDURE - CopyBytes(from: ARRAY OF SYSTEM.BYTE;
                    VAR to: ARRAY OF SYSTEM.BYTE;
                       len: LONGINT);
  " from: ESI  to: EDI  len: ECX ",

(*begin
        cld
        rep     movsb
end*)

0FCH,0F3H,0A4H;
\end{verbatim}

\begin{description}
\item[compiler] \mbox{}
   \begin{itemize}
     \item saves ESI, EDI, ECX if necessary
     \item places address of {\tt from} in ESI,
           address of {\tt to} in EDI,
           value of {\tt len} in ECX
   \end{itemize}
\item[code procedure] \mbox{}
   \begin{itemize}
   \item  moves {\tt len} bytes
   \end{itemize}
\item[compiler] \mbox{}
   \begin{itemize}
   \item  restores context if necessary
   \end{itemize}
\end{description}

\subsubsection{Using 80387 coprocessor (F1)}

The following function calculates $(p_0+1)p_1\pi$.

\begin{verbatim}
PROCEDURE - F1(p0,p1: LONGREAL): LONGREAL
  "p0: ST(0) p1: ST(1) USE: (ST(0-2)) RETURN: ST(0)",

(*begin
        fld1
        fadd
        fmul
        fldpi
        fmul
end*)

0D9H,0E8H,0DEH,0C1H,0DEH,0C9H,0D9H,0EBH,0DEH,0C9H;
\end{verbatim}

\begin{description}
\item[compiler] \mbox{}
   \begin{itemize}
     \item makes sure that 3 FPP stack elements are available
     \item pushes two operands onto FPP stack ({\tt p1}, then {\tt p0})
   \end{itemize}
\item[code procedure] \mbox{}
   \begin{itemize}
   \item performs the computations on stack, using one more stack element
   \end{itemize}
\item[compiler] \mbox{}
   \begin{itemize}
   \item  uses the top of FPP stack as function call result
   \end{itemize}
\end{description}

\subsubsection{Using 80387 coprocessor (F2)}

The following procedure calculates $r:=\sqrt{r\pi}$.

\begin{verbatim}
PROCEDURE - F2(VAR r: LONGREAL);
  "r: EAX  USE: (ST(0),ST(1)) ",

(*begin
        fld     qword [eax]
        fldpi
        fmul
        fsqrt
        fstp    qword [eax]
end*)

0DDH,040H,008H,0D9H,0EBH,0DEH,
0C9H,0D9H,0FAH,0DDH,058H,008H;
\end{verbatim}

\begin{description}
\item[compiler] \mbox{}
   \begin{itemize}
     \item makes sure that 2 FPP stack elements are available
     \item saves EAX if necessary
     \item places address of {\tt res} in EAX
   \end{itemize}
\item[code procedure] \mbox{}
   \begin{itemize}
   \item performs the computations, writing result to memory
         and cleaning FPP stack
   \end{itemize}
\item[compiler] \mbox{}
   \begin{itemize}
   \item restores context if necessary
   \end{itemize}
\end{description}
